% Do zrobienia:
% 1. dodać cytowania do ogólnych stwierdzeń we wstępie.
% 2. dodać wpisy do skorowidzu
% 3. nieco skrócić

\chapter{Wstęp}
\label{chp:intro}

\paragraph{}
\label{par:intro:01}
Analiza własności kwantowych układów złożonych pociąga za sobą niejednokrotnie
konieczność rozpatrywania szczegółów zjawiska
\index{idx}{splątanie kwantowe} \emph{splątania kwantowego},
polegającego w ogólności na występowaniu szczególnych korelacji w pomiarach 
wielkości obserwowalnych pomiędzy poszczególnymi podukładami. 
Jedną z cech charakterystycznych teorii kwantowej,
w odróżnieniu od opisu klasycznego,
jest włączenie do opisu możliwych stanów układów złożonych
przejawiających tę na pierwszy rzut oka paradoksalną własność.
Zjawisko splątania, odkryte jeszcze w latach trzydziestych XIX\,w.,
stanowi jeden z najważniejszych i najintensywniej badanych działów nowoczesnej
teorii kwantowej.
Zapowiadające w przyszłości niezliczone praktyczne zastosowania,
koncentrujące się w głównej mierze na potencjalnych korzyściach płynących
z wykorzystania splątania w procesach obliczeniowych i przesyłu informacji,
zjawisko to pozostaje z punktu widzenia badań podstawowych wciąż jeszcze jednym
z najbardziej zagadkowych elementów teorii.

Podstawowym problemem w zajmującej nas dziedzinie badań,
czekającym na definitywne rozstrzygnięcie,
jest podanie tzw. kryterium splątania,
a więc procedury,
która pozwoliłaby w sposób jednoznaczny odpowiedzieć na pytanie,
czy zadany stan kwantowego układu złożonego jest splątany,
czy też wprost przeciwnie -- jest to 
\index{idx}{stan!separowalny} \emph{stan separowalny}.
Jednym z pierwszych prób znalezienia wyczerpującej odpowiedzi na tak postawione
pytanie było twierdzenie zaprezentowane w swojej szczególnej postaci 
przez A.\,Peresa
\cite{peres1996separability}
i w tym samym czasie nieco bardziej ogólnie przez
R.,\,P.,\,M.\,Horodeckich
\cite{horodecki1996separability},
do którego będziemy się poniżej odnosić jako do 
\index{idx}{Peresa-Horodeckich kryterium} \emph{kryterium Peresa-Horodeckich}.
Odkładając precyzyjne zdefiniowanie niezbędnych pojęć do następnego rozdziału,
przypomnijmy jedynie, że twierdzenie głosi,
iż stan mieszany $\rho$ złożonego układu kwantowego o skończonej ilości
stopni swobody,
traktowany jako macierz gęstości na skończenie wymiarowej przestrzeni 
Hilberta $\mathcal{H}_{1} \otimes \mathcal{H}_{2}$,
gdzie przestrzenie $\mathcal{H}_{1}$ i $\mathcal{H}_{2}$ odpowiadają dwóm podukładom,
jest separowalny wtedy i tylko wtedy,
gdy dla każdego dodatniego odwzorowania liniowego
$S \!: \mathcal{H}_{1} \rightarrow \mathcal{H}_{2}$
i odpowiadającej mu macierzy $F(S)$ na przestrzeni liniowej
$\mathcal{H}_{1} \otimes \mathcal{H}_{2}$,
otrzymanej w izomorficznym obrazie znanym jako izomorfizm Choi-Jamiołkowskiego,
zachodzi
\begin{equation}
    \label{eq:PHcriterionOrig}
        % \nonumber
    \text{tr} \, \rho \, F(S) \: \geq \: 0
\end{equation}


Ponadto, dzięki pełnej charakterystyce odwzorowań dodatnich na dwuwymiarowej
przestrzeni Hilberta,
tj. gdy $\mathcal{H}_{1} = \mathcal{H}_{2} = \mathbb{C}^{2}$,
można omawiane kryterium znacznie uprościć,
jeśli ograniczymy się do najprostszego nietrywialnego układu dwóch qubitów,
i powiedzieć,
że stan $\rho$ jest separowalny wtedy i tylko wtedy,
gdy $\rho^{PT}$, tzw. częściowa transpozycja macierzy $\rho$,
jest macierzą dodatnio określoną.
W takiej postaci kryterium jest niezwykle użyteczne jako efektywne pod względem
obliczeniowym i stanowi definitywną odpowiedź na postawiony problem.
Ze względu na doniosłość omawianego kryterium zadaniem wartym dalszej uwagi
byłaby z jednej strony próba uogólnienia twierdzenia,
tak aby analogiczna teza obowiązywała dla jak najszerszej grupy układów kwantowych;
z drugiej zaś -- przeprowadzenie szczegółowej analizy kolejnych,
bardziej skomplikowanych przypadków,
poczynając od układu złożonego z dwóch podukładów o trzech stopniach swobody,
to jest opisywanych za pomocą trójwymiarowej przestrzeni Hilberta.
Nawet w tej,
wydawałoby się niezbyt zawiłej sytuacji,
niewiele wiadomo w ogólności na temat dodatnich odwzorowań liniowych,
a co za tym idzie -- zastosowanie kryterium Peresa-Horodeckich do konkretnych
obliczeń stanowi utrudnione zadanie.
W niniejszej pracy postaramy się przedstawić rezultaty badań prowadzących
do rozwiązania tych właśnie problemów.

\paragraph{}
\label{par:intro:02}
Dostępne pomiarom wielkości obserwowalne układu kwantowego o nieskończonej
ilości stopni swobody tworzą w naturalny sposób strukturę określaną mianem
algebry operatorów na ośrodkowej przestrzeni Hilberta $\mathcal{H}$,
jednak z reguły nie jest to pełna algebra operatorów ograniczonych
$B(\mathcal{H})$,
a jej domknięta podalgebra $\mathfrak{M}$ spełniająca aksjomaty tzw.
algebry von Neumanna.
Teoria algebr operatorowych,
w szczególności algebr von Neumanna i C$^{*}$-algebr
daje matematyczne narzędzie do opisu w zasadzie wszystkich rozpatrywanych
w fizyce matematycznej przykładów układów fizycznych:
począwszy od układów skończenie wymiarowych i klasycznych,
opisywanych za pomocą algebr przemiennych,
poprzez nieskończone łańcuchy spinowe i algebry Araki-Woodsa reprezentujące
nierelatywistyczny gaz bozonów,
a skończywszy na lokalnych algebrach obserwabli w kwantowej teorii pola.
Należy zaznaczyć,
że mimo iż formalnie wszystkie z tych algebr można przedstawić jako
podalgebry pełnej przestrzeni $B(\mathcal{H})$ dla ośrodkowej przestrzeni Hilberta,
to jednak własności topologiczne każdego z wymienionych przykładów
różnią się od siebie w sposób diametralny.
Dla przykładu, lokalne algebry obserwabli w kwantowej teorii pola,
to tzw. faktory typu III,
które nie zawierają żadnych niezerowych skończonych projektorów,
tworząc strukturę w wysokim stopniu odmienną od struktury algebry $B(\mathcal{H})$.
Ponieważ w swoim oryginalnym sformułowaniu kryterium Peresa-Horodeckich
dotyczy jedynie przestrzeni skończenie wymiarowych i odwzorowań na operatorach
na tych przestrzeniach,
czyli \emph{de facto} macierzy,
wydaje się zasadne,
aby osiągnąć możliwie daleko idące uogólnienie tamtego faktu,
rozszerzając go na szeroką klasę algebr von Neumanna,
mających zastosowanie do opisu rzeczywistych układów kwantowych.
A zatem jednym z dwóch głównych nurtów niniejszej pracy będzie sformułowanie
i dowód twierdzenia o możliwie najogólniejszych założeniach,
które wciąż mogłoby nosić miano kryterium Peresa-Horodeckich.
Potrzeba uogólnienia kryterium znajduje swoje uzasadnienie również 
w rosnącym zainteresowaniu na polu badań nad splątaniem układów o ciągłym
spektrum stopni swobody (\emph{CV-entanglement})
\cite{adesso2007entanglement}.
Ponadto pytanie o uogólnienie omawianego twierdzenia na stany na algebrach
von Neumanna pojawiło się w pracy E.\,St{\o}rmera
\cite{stormer2008separable},
który udowodnił tezę dla C*-algebr, spośród których jednak jest tak zwaną
UHF-algebrą 
(topologicznym domknięciem sumy mnogościowej algebr skończenie wymiarowych).
W naszej pracy zakładany wynik otrzymaliśmy dla iloczynu tensorowego
$\mathfrak{M} \bar{\otimes} \mathfrak{N}$ 
algebr von Neumanna $\mathfrak{M}$, $\mathfrak{N}$,
spośród których jedna jest tzw. algebrą iniektywną;
tak sformułowane twierdzenie pokrywa w zasadzie wszystkie znane
przypadki złożonych układów kwantowych.

Najważniejszym obiektem matematycznym występującym w sformułowaniu
kryterium Peresa-Horodeckich są odwzorowania dodatnie na *-algebrach,
w szczególności na algebrach operatorów,
które pomimo intuicyjnej definicji oraz wielu ważnych zastosowań,
stanowią obiekt stosunkowo słabo poznany.
Teoria tych odwzorowań, zapoczątkowana w pracach Kadisona
\cite{kadison1952generalized}
i Stinespringa
\cite{stinespring1955positive},
jako uogólnienie z jednej strony teorii dodatnich funkcjonałów,
z drugiej zaś izomorfizmów tych algebr,
osiągnęła swój pierwszy ważny wynik w publikacjach Størmera i Woronowicza
\cite{stormer1963positive,woronowicz1976positive}
wraz z podaniem ich pełnej charakteryzacji w niskowymiarowym przypadku
odwzorowań pomiędzy algebrami $B(\mathbb{C}^{2})$ i $B(\mathbb{C}^{3})$
(p. str. \ref{thm:StormerWoronowicz})
Należy w tym miejscu wspomnieć również o osiągnięciach M.-D.\,Choi
\cite{choi1975positive,choi1975completely,choi1977extremal},
który w swych pracach dotyczących odwzorowań kompletnie dodatnich,
kład podwaliny późniejszych wyników.

Na kryterium Peresa-Horodeckich można zatem spojrzeć jak na łącznik pomiędzy
matematyczną teorią odwzorowań dodatnich,
rozwijaną niezależnie od osiągnięć mechaniki kwantowej,
a fizyczną interpretacją w zastosowaniu do teorii splątania:
Ze wzoru \eqref{eq:PHcriterionOrig} możemy wnioskować,
że jeśli stan kwantowy $\rho$ jest splątany,
to istnieje przynajmniej jedno odwzorowanie dodatnie $S$,
które nb. nie jest kompletnie dodatnie,
takie że $\text{tr} \rho \, F(S) < 0$.
Macierz $F(S)$ nosi w tym przypadku nazwę
\emph{świadka splątania} (\emph{entanglement witness}).
Istnieje więc,
przynajmniej w skończenie wymiarowym przypadku,
ścisła odpowiedniość pomiędzy stanami splątanymi,
a dodatnimi odwzorowaniami na algebrach obserwabli.
Ten zaskakujący i głęboki związek pozostaje główną motywacją naszej pracy,
która koncentrować się będzie na zastosowaniu metod algebr operatorowych
do analizy odwzorowań dodatnich,
a co za tym idzie -- do kwantowej teorii informacji,
dziedziny skupiającej się wokół badań nad zjawiskiem splątania kwantowego.

Poza wspomnianymi wcześniej szczegółowymi wynikami St{\o}rmera i Woronowicza,
ogólna charakterystyka odwzorowań dodatnich pozostaje nieznana.
Niewiele wiadomo już nawet w najprostszym nierozwiązanym przypadku odwzorowań
na algebrze macierzy $B(\mathbb{C}^{3})$.
W naszej pracy,
pamiętając o fizycznej motywacji badań w tej dziedzinie,
skupimy się właśnie na tym przypadku.
Spróbujemy wprowadzić nowe przykłady odwzorowań,
,,wykrywających'' nieznane wcześniej stany splątane układu złożonego
z dwóch ,,qutritów'', tj. podukładów o trzech stopniach swobody.
Punktem wyjścia będzie położenie nacisku na geometryczny aspekt znanych
już zależności,
co pozwoli na przedstawienie uproszczonego dowodu twierdzenia St{\o}rmera
dla odwzorowań dodatnich na $B(\mathbb{C}^{2})$.
W dalszej części skupimy się na wyżej wymiarowych odwzorowaniach,
prezentując oryginalne wyniki,
uzupełniające teorię odwzorowań dodatnich o kilka nowych geometrycznych metod.
Potrzeba skupienia wysiłków na przypadku nisko wymiarowym podyktowana jest
z jednej strony brakiem ogólnej wiedzy na temat odwzorowań dodatnich,
z drugiej zaś możliwościami zastosowań do układów aktualnie interesujących
społeczność fizyków pracujących nad zagadnieniami kwantowej teorii informacji,
wykorzystującej układy kwantowe o ograniczonej liczbie stopni swobody
jako modelowe w badaniach nad kwantowym podejściem do problemów obliczeniowych,
w szczególności kryptografii i przesyłu informacji.

W kolejnych rozdziałach przedstawimy główne wyniki pracy,
w całości pochodzące z opublikowanych wcześniej artykułów naukowych,
przygotowanych przez autora i promotora tej pracy
\cite{miller2014horodeckis,miller2015stable,miller2015topology}.
Poczynając od rozdziału poświęconego przygotowaniu matematycznemu i ustaleniu
obowiązującej notacji (rozdz. \ref{chp:preliminaries}),
gdzie spróbujemy przybliżyć najważniejsze fakty dotyczące algebr operatorów,
w szczególności algebr von Neumanna,
oraz odwzorowań dodatnich i kompletnie dodatnich,
włączając w to geometryczne aspekty teorii,
takie jak teoria zbiorów wypukłych w przestrzeniach rzeczywistych,
punkty ekstremalne, ,,wystające'' (\emph{exposed}) i ich własności.
Następnie,
w rodz. \ref{chp:PHcrit},
przejdziemy do pełnego sformułowania i udowodnienia twierdzenia
uogólniającego kryterium Peresa-Horodeckich na przypadek algebr von Neumanna.
W kolejnym rozdziale zajmiemy się analizą struktury odwzorowań dodatnich na
nisko wymiarowych algebrach macierzy.
Poczynając od pełnego scharakteryzowania odwzorowań na algebrze
$B(\mathbb{C}^{2})$ za pomocą geometrycznych metod znanych z analizy
operatorów liniowych na stożkach w przestrzeniach rzeczywistych,
przejdziemy do wyżej wymiarowego przypadku,
próbując uogólnić wcześniejsze wyniki.
Poza rozwiniętymi już metodami wykorzystamy podejście do badania odwzorowań
liniowych uwzględniające asymptotykę półgrup odwzorowań będących
kolejnymi potęgami tego wyjściowego.
Podobnie jak w pracy
\cite{olkiewicz1999environment},
zajmującej się asymptotycznymi własnościami generatorów kwantowych półgrup
dynamicznych,
wykorzystamy tzw. stabilne podprzestrzenie tych półgrup do uzyskania
wstępnej klasyfikacji ekstremalnych odwzorowań dodatnich 
zachowujących ślad i operator identyczności.
Jak się okazuję,
można powiedzieć nie troszcząc się póki co o pełną ścisłość sformułowań,
że dla tego rodzaju odwzorowań taka stabilna podprzestrzeń musi ograniczać się
tylko do kilku możliwych przykładów,
a jej wymiar pozostaje swego rodzaju niezmiennikiem na operacje składania
z automorfizmami Jordana,
będącymi również podstawowymi operacjami zachowującymi separowalność stanów
kwantowych.
Otrzymany rezultat posłuży nam do zaprezentowania oryginalnego przykładu
ekstremalnego odwzorowania dodatniego niebadanego wcześniej,
który stanowi swego rodzaju nowy element teorii,
być może użyteczny w dalszych zastosowaniach,
i który już znalazł swoje uogólnienie w pracach innych fizyków
\cite{rutkowski2015class}.
Pracę kończy rozdział podsumowujący otrzymane rezultaty i nakreślający
perspektywy dalszych badań w ramach zaprezentowanej tematyki.

Bez wątpienia wyniki przytoczone w niniejszej pracy wpisują się w szerszy nurt
aktualnych badań nad zjawiskiem splątania.
Wysiłki badaczy koncentrują się także na podaniu alternatywnych kryteriów
orzekających o splątaniu stanów mieszanych,
wykorzystujących do tego celu podejścia takie jak korelacje kwantowe,
miary splątania etc. (cytaty).
Korzyścią, która płynie z alternatywnych podejść,
jest nierzadko mniejsza złożoność obliczeniowa,
dostępność algorytmów ułatwiających otrzymanie statystycznie istotnych wyników,
a także możliwość uniknięcia największej trudności związanej z kryterium
Peresa-Horodeckich,
mianowicie potrzeby znajomości ogólnej struktury dodatnich odwzorowań
na zadanej algebrze operatorów.
Jednak twierdzenie podane przez Peresa i Horodeckich oraz 
jego uogólnienie udowodnione w następnych rozdziałach
jest jedynym znanym warunkiem separowalności nie tylko koniecznym, ale i wystarczającym.
Jasne jest,
że im głębsza wiedza na temat odwzorowań dodatnich,
tym łatwiejsze staje się zastosowanie kryterium do detekcji splątania
rzeczywistych układów kwantowych.
Jak zobaczymy poniżej,
każde wejrzenie w matematyczną strukturę analizowanych tu obiektów,
pozostającą po większej części wciąż niezbadaną,
przekłada się niemal natychmiast na potencjalne zastosowania w postaci
obserwabli wykrywających splątanie pewnych określonych stanów układów złożonych
(\emph{entanglement witness}).
Ta fascynująca i nieoczywista zależność pomiędzy z jednej strony abstrakcyjną
teorią matematyczną,
z drugiej zaś kwantową teorią informacji,
bez wątpienia w przyszłości odgrywać będzie wciąż ważną rolę,
inspirującą badaczy do dalszych poszukiwań w tej dziedzinie.

%%%%% **** %%%%%


























