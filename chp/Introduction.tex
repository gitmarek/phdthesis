\chapter*{Wstęp}
\addcontentsline{toc}{chapter}{Wstęp}
\label{chp:intro}

\paragraph{}
\label{par:intro:01}
Analiza własności kwantowych układów złożonych pociąga za sobą niejednokrotnie
konieczność rozpatrywania zjawiska
\index{idx}{splątanie kwantowe} \emph{splątania kwantowego},
polegającego na występowaniu korelacji w pomiarach
wielkości obserwowalnych pomiędzy poszczególnymi podukładami.
Jedną z cech charakterystycznych teorii kwantowej,
w odróżnieniu od opisu klasycznego,
jest występowanie stanów układów złożonych przejawiających tego typu korelacje.
Zjawisko splątania, odkryte jeszcze w latach trzydziestych XX\,w.
\cite{einstein1935can,schrodinger1935gegenwartige},
stanowi jeden z najważniejszych i najintensywniej badanych działów nowoczesnej
teorii kwantowej:
budzące nadzieję na niezliczone praktyczne zastosowania w przyszłości,
w głównej mierze opartych na potencjalnych korzyściach płynących
z wykorzystania splątania w procesach obliczeniowych i przesyłu informacji.
Pomimo szerokiej dyskusji,
zjawisko to pozostaje z punktu widzenia badań podstawowych wciąż jeszcze jednym
z najbardziej zagadkowych elementów
\mbox{teorii \cite{horodecki2009quantum}}.

Podstawowym problemem zajmującej nas dziedziny badań,
czekającym na definitywne rozstrzygnięcie,
jest podanie tzw. \emph{kryterium splątania},
a więc procedury lub twierdzenia,
które pozwoliłoby w sposób jednoznaczny odpowiedzieć na pytanie,
czy zadany stan kwantowego układu złożonego jest splątany,
czy też wprost przeciwnie -- jest to
\index{idx}{stan!separowalny} \emph{stan separowalny}.
Jednym z pierwszych prób znalezienia wyczerpującej odpowiedzi na tak postawione
pytanie było twierdzenie zaprezentowane w swojej szczególnej postaci
przez A.\,Peresa
\cite{peres1996separability}
i w tym samym czasie przedstawione nieco bardziej ogólnie przez
R.,\,P.,\,M.\,Horodeckich
\cite{horodecki1996separability},
do którego będę się odnosić poniżej jako do
\index{idx}{Peresa-Horodeckich kryterium} \emph{kryterium Peresa-Horodeckich}.
Odkładając precyzyjne zdefiniowanie niezbędnych pojęć do następnego rozdziału,
przypomnijmy jedynie, że twierdzenie głosi,
iż stan mieszany $\rho$ złożonego układu kwantowego o skończonej liczbie
stopni swobody,
traktowany jako macierz gęstości na skończenie wymiarowej przestrzeni
Hilberta $\mathcal{H}_{1} \! \otimes \! \mathcal{H}_{2}$,
gdzie przestrzenie $\mathcal{H}_{1}$ i $\mathcal{H}_{2}$ odpowiadają dwóm podukładom,
jest separowalny wtedy i tylko wtedy,
gdy dla każdego dodatniego odwzorowania liniowego
$S \!: \mathcal{H}_{1} \rightarrow \mathcal{H}_{2}$
i odpowiadającej mu macierzy $F(S)$ na przestrzeni liniowej
$\mathcal{H}_{1} \! \otimes \! \mathcal{H}_{2}$,
otrzymanej w izomorficznym obrazie znanym jako izomorfizm Choi-Jamiołkowskiego
\cite{choi1975completely, jamiolkowski1972linear},
zachodzi
\begin{linenomath*}
 \begin{equation}
    \nonumber
    \label{eq:PHcriterionOrig}
        % \nonumber
    \text{Tr} \, \rho \, F(S) \: \geq \: 0.
 \end{equation}
\end{linenomath*}
Ponadto, dzięki znajomości pełnej charakterystyki odwzorowań dodatnich
na dwuwymiarowej zespolonej przestrzeni Hilberta,
tj. gdy $\mathcal{H}_{1} = \mathcal{H}_{2} = \mathbb{C}^{2}$,
można omawiane kryterium znacznie uprościć.
Jeśli ograniczymy się do najprostszego nietrywialnego układu dwóch qubitów,
powiemy,
że stan $\rho$ jest separowalny wtedy i tylko wtedy,
gdy $\rho^{PT}$, tzw. częściowa transpozycja macierzy $\rho$,
jest macierzą dodatnio określoną.
W takiej postaci twierdzenie jest niezwykle użyteczne, efektywne pod względem
obliczeniowym i stanowiące definitywną odpowiedź na postawione pytanie.
Ze względu na doniosłość omawianego kryterium,
zadaniem wartym dalszej uwagi
byłaby z jednej strony próba uogólnienia twierdzenia,
tak aby analogiczna teza obowiązywała dla jak najszerszej grupy układów kwantowych;
z drugiej zaś -- przeprowadzenie szczegółowej analizy kolejnych,
bardziej skomplikowanych przypadków,
poczynając od układu złożonego z dwóch podukładów
opisywanych za pomocą trójwymiarowej przestrzeni Hilberta,
czyli tak zwanych ,,qutritów''.
Nawet w tej,
wydawałoby się niezbyt skomplikowanej,
sytuacji,
niewiele wiadomo w ogólności na temat dodatnich odwzorowań liniowych,
a co za tym idzie,
podanie efektywnego kryterium separowalności stanów złożonych stanowi
ciągle nierozwiązany problem.
W niniejszej rozprawie postaram się przedstawić rezultaty badań
rzucających nieco światła na te zagadnienia.

\paragraph{}
\label{par:intro:02}
Dostępne pomiarom wielkości obserwowalne układu kwantowego o nieskończonej
liczbie stopni swobody tworzą w naturalny sposób,
wraz ze swoim algebraicznym domknięciem,
strukturę określaną mianem algebry operatorów na przestrzeni Hilberta
$\mathcal{H}$.
Z reguły jednak nie jest to pełna algebra operatorów ograniczonych
$\mathcal{B}(\mathcal{H})$,
a jej domknięta podalgebra $\mathfrak{M}$ spełniająca aksjomaty tzw.
algebry von Neumanna
\cite{Takesaki1}.
Teoria algebr operatorowych,
w szczególności algebr von Neumanna i \mbox{C$^{*}$-algebr},
daje narzędzie do opisu w zasadzie wszystkich rozpatrywanych
w fizyce matematycznej przykładów układów fizycznych:
począwszy od układów skończenie wymiarowych i klasycznych,
reprezentowanych za pomocą algebr przemiennych,
poprzez nieskończone łańcuchy spinowe i algebry Araki-Woodsa reprezentujące
nierelatywistyczny gaz bozonów
\cite{araki1968classification},
a skończywszy na lokalnych algebrach obserwabli w kwantowej teorii pola
\cite{Nielsen2010}.
Należy zaznaczyć,
że pomimo formalnej możliwości przedstawienia wszystkich z tych algebr jako
podalgebr pełnej przestrzeni $\mathcal{B}(\mathcal{H})$,
%  dla ośrodkowej przestrzeni Hilberta,
własności topologiczne każdego z wymienionych przypadków
różnią się jednak od siebie diametralnie.
Dla przykładu, lokalne algebry obserwabli w kwantowej teorii pola
to tzw. faktory typu III,
które nie zawierają żadnych niezerowych skończonych projektorów,
tworząc strukturę w wysokim stopniu odmienną od struktury algebry $\mathcal{B}(\mathcal{H})$.
Ponieważ w swoim oryginalnym sformułowaniu kryterium Peresa-Horodeckich
dotyczy jedynie przestrzeni skończenie wymiarowych i odwzorowań operatorów
na tych przestrzeniach,
czyli \emph{de facto} macierzy,
wydaje się zasadne,
aby pokusić się o możliwie daleko idące uogólnienie,
rozszerzając kryterium na szeroką klasę algebr von Neumanna,
mających zastosowanie do opisu rzeczywistych układów kwantowych.
A zatem jednym z dwóch głównych nurtów niniejszej pracy będzie sformułowanie
i dowód twierdzenia stanowiącego możliwie najogólniejszą wersją
kryterium Peresa-Horodeckich.
Potrzeba uogólnienia kryterium znajduje swoje uzasadnienie również
w rosnącym zainteresowaniu na polu badań nad splątaniem układów o ciągłym
spektrum stopni swobody (\emph{CV-entanglement})
\cite{adesso2007entanglement}.
Pytanie o możliwość uogólnienia, którego dowodzę w niniejszej pracy,
pojawiło się już wcześniej w publikacji E.\,St{\o}rmera
\cite{stormer2008separable},
który udowodnił tezę dla C*-algebr,
spośród których jedna jest tak zwaną UHF-algebrą,
czyli topologicznym domknięciem sumy mnogościowej algebr skończenie wymiarowych.
W naszej pracy zakładany wynik otrzymałem dla iloczynu tensorowego
$\mathfrak{M} \bar{\otimes} \mathfrak{N}$
algebr von Neumanna $\mathfrak{M}$ i $\mathfrak{N}$,
spośród których jedna jest algebrą iniektywną.
Tak sformułowane twierdzenie obejmuje wszystkie znane
przypadki złożonych układów kwantowych.

Najważniejszym obiektem matematycznym występującym w sformułowaniu
kryterium Peresa-Horodeckich są odwzorowania dodatnie na $^{*}$-algebrach operatorów,
które pomimo intuicyjnej definicji oraz wielu ważnych zastosowań,
stanowią obiekty stosunkowo słabo
\mbox{poznane \cite{Stormer2013}}.
Teoria tych odwzorowań, zapoczątkowana w pracach
Kadisona \cite{kadison1952generalized}
i Stinespringa
\cite{stinespring1955positive},
jako uogólnienie z jednej strony teorii dodatnich funkcjonałów liniowych,
z drugiej zaś *-homomorfizmów algebr,
osiągnęła swój pierwszy ważny rezultat w publikacjach Størmera i Woronowicza
\cite{stormer1963positive,woronowicz1976positive}
wraz z podaniem pełnej charakteryzacji w nisko wymiarowym przypadku
odwzorowań pomiędzy algebrami $\mathcal{B}(\mathbb{C}^{2})$ i $\mathcal{B}(\mathbb{C}^{3})$
(por. str. \pageref{thm:PositiveMapsOnM2})
Należy w tym miejscu wspomnieć również o osiągnięciach M.-D.\,Choi
\cite{choi1975completely,choi1975positive,choi1980some,choi1977extremal},
który w swych pracach dotyczących odwzorowań kompletnie dodatnich
kładł podwaliny późniejszych wyników.

Na kryterium Peresa-Horodeckich można zatem spojrzeć jak na łącznik pomiędzy
matematyczną teorią odwzorowań dodatnich,
rozwijaną niezależnie od osiągnięć mechaniki kwantowej,
a fizyczną interpretacją w zastosowaniu do teorii splątania.
Z nierówności przytoczonej powyżej %\eqref{eq:PHcriterionOrig}
możemy wnioskować,
że jeśli stan kwantowy $\rho$ jest splątany,
to istnieje przynajmniej jedno odwzorowanie dodatnie $S$,
takie że $\text{Tr} \rho \, F(S) < 0$.
Macierz $F(S)$ nosi w tym przypadku nazwę
\emph{świadka splątania} (\emph{entanglement witness})
\cite{bourennane2004experimental}.
Istnieje więc,
na tym etapie przynajmniej w skończenie wymiarowym przypadku,
ścisła odpowiedniość pomiędzy stanami splątanymi,
a dodatnimi odwzorowaniami na algebrach obserwabli.
Ten zaskakujący i głęboki związek pozostaje główną motywacją naszej pracy,
która koncentrować się będzie na zastosowaniu metod algebr operatorowych
do analizy odwzorowań dodatnich,
a więc również do teorii splątania kwantowego.

Poza wspomnianymi szczegółowymi wynikami St{\o}rmera i Woronowicza,
ogólna charakterystyka odwzorowań dodatnich pozostaje nieznana.
Niewiele wiadomo nawet w najprostszym nierozwiązanym przypadku odwzorowań
na algebrze macierzy $\mathcal{B}(\mathbb{C}^{3})$.
W niniejszej pracy,
pamiętając o fizycznej motywacji badań w tej dziedzinie,
skupię się właśnie na tym przypadku.
Spróbuję wprowadzić nowe przykłady odwzorowań
,,wykrywających'' nieznane wcześniej stany splątane układu złożonego
z dwóch qutritów.
Punktem wyjścia będzie położenie nacisku na geometryczny aspekt znanych
już zależności,
co pozwoli na przedstawienie uproszczonego dowodu twierdzenia St{\o}rmera
dla odwzorowań dodatnich na $\mathcal{B}(\mathbb{C}^{2})$.
W dalszej części zaprezentuję oryginalne wyniki,
uzupełniające teorię odwzorowań dodatnich o kilka nowych geometrycznych metod.
Potrzeba skupienia wysiłków na przypadku nisko wymiarowym podyktowana jest
z jednej strony brakiem ogólnej wiedzy na temat odwzorowań dodatnich,
z drugiej zaś możliwościami zastosowań do układów aktualnie w polu zainteresowania
społeczności fizyków pracujących nad zagadnieniami kwantowej teorii informacji,
wykorzystującej układy kwantowe o ograniczonej liczbie stopni swobody
jako modelowe w badaniach nad kwantowym podejściem do problemów obliczeniowych,
w szczególności kryptografii i przesyłu informacji.

W kolejnych rozdziałach przedstawię główne wyniki pracy,
w całości pochodzące z opublikowanych wcześniej artykułów naukowych,
przygotowanych przez autora i promotora niniejszej rozprawy
\cite{miller2015extremal,miller2014horodeckis,miller2015stable,miller2015topology}.
Zaczynając od rozdziału \ref{chp:preliminaries}.,
%poświęconego przygotowaniu
%matematycznemu i ustaleniu obowiązującej notacji,
spróbuję przybliżyć najważniejsze fakty dotyczące algebr operatorów,
w szczególności algebr von Neumanna,
oraz odwzorowań dodatnich i kompletnie dodatnich,
włączając w to geometryczne aspekty teorii,
takie jak teoria zbiorów wypukłych w przestrzeniach rzeczywistych,
punkty ekstremalne, ,,wystające'' (\emph{exposed}) i ich własności.
Następnie,
w rozdziale \ref{chp:PHcrit}.,
przejdę do pełnego sformułowania i udowodnienia twierdzenia
uogólniającego kryterium Peresa-Horodeckich na przypadek algebr von Neumanna
i C*-algebr.
W kolejnym rozdziale zajmę się analizą struktury odwzorowań dodatnich na
nisko wymiarowych algebrach macierzy.
Poczynając od pełnego scharakteryzowania odwzorowań na algebrze
$\mathcal{B}(\mathbb{C}^{2})$ za pomocą geometrycznych metod znanych z analizy
operatorów liniowych na stożkach w przestrzeniach rzeczywistych,
przejdę do wyżej wymiarowego przypadku,
próbując uogólnić wcześniejsze wyniki.
Poza rozwiniętymi już metodami,
wykorzystam podejście do badania odwzorowań
liniowych uwzględniające asymptotykę półgrup odwzorowań.
Podobnie jak w teorii
zajmującej się asymptotycznymi własnościami generatorów kwantowych półgrup
dynamicznych \cite{olkiewicz1999environment},
wykorzystane zostaną tzw. stabilne podprzestrzenie tych półgrup do uzyskania
wstępnej klasyfikacji ekstremalnych odwzorowań dodatnich
zachowujących ślad i operator identyczności,
czyli tzw. odwzorowań \emph{bistochastycznych}.
Nie uściślając w pełni sformułowań na tym etapie,
można powiedzieć,
że dla tego rodzaju odwzorowań stabilna podprzestrzeń musi ograniczać się
tylko do kilku możliwych przypadków.
Otrzymany rezultat posłuży do zaprezentowania oryginalnego przykładu
ekstremalnego odwzorowania dodatniego, niebadanego wcześniej,
który stanowi swego rodzaju nowy element teorii,
być może użyteczny w dalszych zastosowaniach,
i który już znalazł swoje uogólnienie w pracach innych fizyków
\cite{rutkowski2015class}.
Pracę kończy rozdział podsumowujący otrzymane rezultaty i nakreślający
perspektywy dalszych badań w ramach przedstawionej tematyki.

Bez wątpienia wyniki przytoczone w niniejszej pracy wpisują się w szerszy nurt
aktualnych badań nad zjawiskiem splątania.
Wysiłki badaczy koncentrują się także na podaniu alternatywnych kryteriów
orzekających o splątaniu stanów mieszanych,
wykorzystujących do tego celu podejścia takie jak korelacje kwantowe,
miary splątania itd.
\cite{modi2010unified,ollivier2001quantum}.
Korzyścią, która płynie z alternatywnych podejść,
jest nierzadko mniejsza złożoność obliczeniowa,
dostępność algorytmów ułatwiających otrzymanie statystycznie istotnych wyników,
a także możliwość uniknięcia największej trudności związanej z kryterium
Peresa-Horodeckich,
mianowicie potrzeby znajomości ogólnej struktury dodatnich odwzorowań
na zadanej algebrze operatorów.
Jednak twierdzenie podane przez Peresa i Horodeckich oraz
jego uogólnienie udowodnione w następnych rozdziałach
jest jedynym znanym warunkiem separowalności
koniecznym i wystarczającym zarazem.
Jasne jest,
że im głębsza wiedza na temat odwzorowań dodatnich,
tym łatwiejsze staje się zastosowanie kryterium do detekcji splątania
rzeczywistych układów kwantowych.
Jak zobaczymy poniżej,
każde wejrzenie w matematyczną strukturę analizowanych tu obiektów,
pozostającą po większej części wciąż niezbadaną,
przekłada się niemal natychmiast na potencjalne zastosowania w postaci
obserwabli wykrywających splątanie szerokiej klasy stanów układów złożonych.
Ta fascynująca i nieoczywista zależność pomiędzy,
z jednej strony abstrakcyjną teorią matematyczną,
z drugiej zaś kwantową teorią informacji,
bez wątpienia w przyszłości wciąż odgrywać będzie ważną rolę,
inspirującą badaczy do dalszych poszukiwań w tej dziedzinie.

\vspace{0.5cm}
Korzystając ze sposobności,
chciałbym podziękować promotorowi niniejszej rozprawy,
prof. dr. hab. Robertowi Olkiewiczowi,
za mentorską opiekę w czasie moich studiów fizycznych,
przekazaną mi wiedzę i pomoc w stawianiu pierwszych kroków
w samodzielnej pracy naukowej.

%%%%% **** %%%%%
