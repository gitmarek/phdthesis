\thispagestyle{empty}

\begin{center}
{ \large \bfseries
    Mathematical methods of operator algebras
    in the analysis of an entanglement criterion
    of composite quantum systems.
}\\[1.5cm]

% Author and supervisor
{
\begin{tabular}{lcr}
\emph{Author:} & \hspace{3cm} & \emph{Supervisor:}\\
Marek \textsc{Miller} & & prof. dr hab. \\
    & & Robert \textsc{Olkiewicz}
\end{tabular}
}
\\[1cm]
Abstract\\[0.5cm]
\end{center}

{
This work presents results concerning the structure of
positive maps of operator algebras, particularly matrix algebras,
with respect to applications to the quantum information theory,
especially the analysis of an entanglement criterion of composite
quantum systems.
A straightforward connection between the considered criterion
and the theory of positive maps is the main physical motivation
behind the outlined attempt to specify new examples of extreme
positive maps, as well as to investigate further the complicated
structure of these objects.

Starting from a sound definition of entangled states on a tensor product of
von Neumann or C*-algebras,
one proves a theorem, analogous to the criterion of entanglement
existing previously in the literature,
although this time for the most general and still physically meaningful
setting of injective von Neumann algebras,
as well as nuclear C*-algebras.
Furthermore, the structure of positive maps acting on low dimensional
matrix algebras has been investigated.
The first step was to fully characterise the maps of the algebra
$\mathcal{B}(\mathbb{C}^{2})$ by means of geometrical methods known
from the theory of linear operators preserving positive real cones.
Next, so called stable subspaces of semigroups generated by a positive map
of $\mathcal{B}(\mathbb{C}^{3})$ have been employed
to obtain a preliminary classification of extreme positive maps
that preserve both trace and the identity element (bistochastic maps).
This result allows to present a genuine example of an extreme positive
map acting on the matrix algebra $\mathcal{B}(\mathbb{C}^{3})$,
that constitutes a new and important addition to the theory.
This work ends with the chapter outlining the primary role of
idempotent elements of the semigroup of bistochastic maps
and specifying the future program of research.
}
\vfill
