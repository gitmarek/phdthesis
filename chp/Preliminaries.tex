\chapter{Przygotowanie}
\label{chp:preliminaries}

\paragraph{}
Metody matematyczne,
niezbędne do badania kryterium splątania nieskończenie wymiarowych złożonych
układów kwantowych, koncentrują się wokół zastosowań algebr operatorowych,
w szczególności algebr von Neumanna,
oraz analizy struktury tzw. dodatnich odwzorowań na tych algebrach.

W tym rozdziale zajmiemy się przedstawieniem podstawowych informacji dotyczących
wykorzystywanego w dalszym wywodzie aparatu matematycznego.
Poczynając od prezentacji najważniejszych wyników z teorii algebr operatorowych,
w tym algebr von Neumanna oraz algebr Jordana,
przejdziemy
w dalszej kolejności do omówienia podstawowych własności
odwzorowań dodatnich na tych algebrach.
% Na zakończenie rozdziału pokusimy się o przedstawienie podstawowych zastosowań
% prezentowanych metod do kwantowej teorii informacji.
Rozdział ten ma służyć również ustaleniu spójnej notacji obowiązującej poniżej.

\paragraph{}
Niech $n \in \mathbb{N}$ będzie liczbą naturalną.
Niech także $\mathbb{C}^{n}$ oznacza zespoloną $n$-wymiarową przestrzeń liniową,
a $\mathbb{R}^{n}$ przestrzeń rzeczywistą --
wyposażoną w naturalny iloczyn skalarny.
Dla ogólnej przestrzeni Hilberta,
niekoniecznie skończenie wymiarowej,
zarezerwujmy symbol $\mathcal{H}$.
Wektory należące do $\mathcal{H}$ oznaczamy
$\eta, \xi, \upsilon$ itd.
Dla odróżnienia od wektorów należących do innej przestrzeni,
będziemy czasem używać strzałek: $\vec{\eta}, \vec{\xi}, \vec{\upsilon}$ itd.
Dla podprzestrzeni $\mathbb{C}^{n}$,
$\eta, \xi$ itd. oznaczają zawsze wektory-kolumny;
wektory-wiersze będziemy zapisywać
$\eta^{t}, \xi^{t}$ itd.
Kreska nad symbolem wektora: $\bar{\eta}, \bar{\xi}$ itd.,
oznaczać będzie zawsze zespolone sprzężenie poszczególnych współrzędnych wektora.
Naturalny iloczyn skalarny w $\mathbb{C}^{n}$ oznaczamy
$\langle \eta , \xi \rangle = \eta^{*} \xi$,
gdzie $\eta^{*} = \bar{\eta}^{t}$.
Operatory liniowe na $\mathbb{C}^{n}$ lub $\mathbb{R}^{n}$,
a więc macierze,
będziemy oznaczać zwykle $A, B, C$ itd.
Przestrzeń Banacha wszystkich operatorów liniowych,
ograniczonych w standardowej normie operatorowej,
będziemy oznaczać $\mathcal{B}(\mathcal{H})$.
Dla $A \in \mathcal{B}(\mathcal{H})$,
symbol $A^{*}$ oznacza sprzężenie Hermitowskie operatora.
Przestrzeń $\mathcal{B}(\mathcal{H})$ z antyliniową operacją
$A \mapsto A^{*}$ oraz z operacją składania operatorów
tworzy \mbox{$^{*}$-algebrę} Banacha.
Szczególną algebrę macierzy kwadratowych rozmiaru $n$ nad ciałem liczb
rzeczywistych lub zespolonych,
a więc gdy $\mathcal{H} = \mathbb{C}^{n}$ lub 
$\mathcal{H} = \mathbb{R}^{n}$,
będziemy dodatkowo oznaczać symbolem $\mathcal{M}_{n}(\mathbb{R})$ lub
$\mathcal{M}_{n}(\mathbb{C})$; 
zawsze $\mathcal{M}_{n} = \mathcal{M}_{n}(\mathbb{C})$.
Gdy $A \in \mathcal{M}_{n}$, 
$A^{t}$ oznacza transpozycję macierzy,
a $\text{tr} A$ jej ślad.
Dla ogólnych przestrzeni liniowych $V$ i $W$,
których bazy to odpowiednio
$\{ \eta_{i} \}_{i \in I}$ i
$\{ \xi_{j} \}_{j \in J}$,
algebraiczny iloczyn tensorowy
jest przestrzenią liniową $V \! \otimes \! W$, generowaną przez wektory
$\{\eta_{i} \otimes \xi_{j}\}_{i\in I, j \in J}$.
Iloczyn tensorowy przestrzeni Hilberta $\mathcal{H}_{1}$ i $\mathcal{H}_{2}$
oznaczany  przez $\mathcal{H}_{1} \overline{\otimes} \mathcal{H}_{2}$,
jest przestrzenią Hilberta będącą domknięciem przestrzeni liniowej
$\mathcal{H}_{1} \! \otimes \! \mathcal{H}_{2}$ w topologii indukowanej
przez iloczyn skalarny
$\langle \eta_{1} \otimes \xi_{1}, \eta_{2} \otimes \xi_{2} \rangle =
\langle \eta_{1} , \eta_{2} \rangle \, \langle \xi_{1}, \xi_{2} \rangle$,
$\eta_{1}, \eta_{2} \in \mathcal{H}_{1}$,
$\xi_{1}, \xi_{2} \in \mathcal{H}_{2}$.
Istnieje wiele norm,
w które można wyposażyć iloczyn tensorowy ogólnych przestrzeni Banacha $X$ i $Y$,
tak aby uzyskać również przestrzeń Banacha.
Tę wieloznaczność można jednak wyeliminować w przypadku niektórych algebr operatorów.




















\section{Algebry operatorowe}


\section{Odwzorowania dodatnie}


\paragraph{}
W następnym rozdziale wykorzystamy przedstawione powyżej podstawowe fakty
matematyczne do precyzyjnego sformułowania i udowodnienia wyników związanych
z uogólnieniem kryterium Peresa-Horodeckich dla stanów złożonych układów
kwantowych opisywanych w języku algebr von Neumanna.
Zobaczymy,
w jaki sposób znajomość struktury tych algebr pozwoli na przeprowadzenie
pełnego dowodu w kilku krokach, dla obieków o coraz większym stopniu ogólności:
poczynając od algebr skończenie wymiarowych,
a kończąc na egzotycznych algebrach typu III.

% ### Plan szczegółowy rozdziału:
% 1. Wstęp fizyczny cz. 1: uzasadnienie opisu złożonych układów fizycznych w algebraicznym języku alg. vN.
% 2. Wstęp fizyczny cz.2: badanie splątania nierozerwalnie związane z badaniem dodatnich odwzorowań na algebrach macierzy, kryterium P.-H.
% 3. Wstęp mat: algebry vN; alg. Jordana;
% 4. Odwzorowania dodatnie, kompletnie dodatnie, kompletnie ograniczone itd.; algebry operatorowe;
% 5. Wstęp fizyczny cz.3: dalej o kryt. splątania: *entanglement wittness*, izomorfizm Choi-Jamiołkowski (odniesienie do kolejnych rozdziałów); znów zwrócenie uwagi na to, w jakim kierunku będzie zmierzać prezentacja naszych wyników.

% Na początek rozdziału 3:
% Jak już zostało wspomniane powyżej,
% podstawowym narzędziem, dzięki któremu możemy rozstrzygnąć,
% czy dany stan złożonego układu kwantowego jest splątany,
% jest kryterium odkryte przez Peresa i Horodeckich
% \cite{peres1996separability, horodecki1996separability},
% stwierdzające, że stan kwnatowy $\phi$,
% 


\paragraph{}

