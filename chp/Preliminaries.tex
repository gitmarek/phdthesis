\chapter{Matematyczne metody analizy stanów splątanych}
\label{chp:preliminaries}

\paragraph{}
Metody matematyczne,
niezbędne do badania kryterium splątania nieskończenie wymiarowych złożonych
układów kwantowych, koncentrują się wokół zastosowań algebr operatorowych,
w szczególności algebr von Neumanna,
oraz analizy struktury tzw. dodatnich odwzorowań na tych algebrach.

W tym rozdziale zajmiemy się przedstawieniem podstawowych informacji dotyczących
wykorzystywanego w dalszym wywodzie aparatu matematycznego.
Poczynając od prezentacji najważniejszych wyników z teorii algebr operatorowych,
przejdziemy w dalszej kolejności do omówienia podstawowych własności
odwzorowań dodatnich na tych algebrach.
Rozdział ten ma służyć również ustaleniu spójnej
notacji obowiązującej w całej pracy.

\paragraph{}
Niech $n \in \mathbb{N}$ będzie liczbą naturalną.
Niech $\mathbb{C}^{n}$ oznacza zespoloną $n$-wymiarową przestrzeń liniową,
a $\mathbb{R}^{n}$ przestrzeń rzeczywistą,
obie wyposażone w naturalny iloczyn skalarny.
Dla ogólnej przestrzeni Hilberta,
niekoniecznie skończenie wymiarowej,
zarezerwujmy symbol $\mathcal{H}$.
Wektory należące do $\mathcal{H}$ oznaczamy:
$\eta, \xi, \upsilon$ itd.
Dla odróżnienia od wektorów należących do innej przestrzeni,
będziemy czasem używać strzałek: $\vec{\eta}, \vec{\xi}, \vec{\upsilon}$ itd.
Dla podprzestrzeni $\mathbb{C}^{n}$,
$\eta, \xi$ itd. oznaczają zawsze wektory-kolumny;
wektory-wiersze będziemy zapisywać
$\eta^{t}, \xi^{t}$ itd.
Kreska nad symbolem wektora: $\bar{\eta}, \bar{\xi}$ itd.,
oznaczać będzie zawsze zespolone sprzężenie poszczególnych współrzędnych wektora.
Naturalny iloczyn skalarny w $\mathbb{C}^{n}$ oznaczamy
$\langle \eta , \xi \rangle = \eta^{*} \xi$,
gdzie $\eta^{*} = \bar{\eta}^{t}$.
Operatory liniowe na $\mathbb{C}^{n}$ lub $\mathbb{R}^{n}$,
a więc macierze,
będziemy oznaczać zwykle $A, B, C$ itd.
Przestrzeń Banacha wszystkich operatorów liniowych,
ograniczonych w standardowej normie operatorowej
\eqref{eq:operatorNorm},
będziemy oznaczać $\mathcal{B}(\mathcal{H})$.
Dla $A \in \mathcal{B}(\mathcal{H})$,
symbol $A^{*}$ oznacza sprzężenie Hermitowskie operatora.
Przestrzeń $\mathcal{B}(\mathcal{H})$ z antyliniową operacją
$A \mapsto A^{*}$ oraz z operacją składania operatorów
tworzy szczególny przykład \mbox{$^{*}$-algebry} Banacha (p. niżej).
Algebrę macierzy kwadratowych rozmiaru $n$ nad ciałem liczb
rzeczywistych lub zespolonych,
a więc gdy $\mathcal{H} = \mathbb{C}^{n}$ lub
$\mathcal{H} = \mathbb{R}^{n}$,
będziemy dodatkowo oznaczać symbolem $M_{n}(\mathbb{R})$ lub
$M_{n}(\mathbb{C})$;
zawsze $M_{n} = M_{n}(\mathbb{C})$.
Gdy $A \in M_{n}$,
$A^{t}$ oznacza transpozycję macierzy,
a $\text{Tr} A$ jej ślad.
Dla ogólnych przestrzeni liniowych $V$ i $W$,
których bazy to odpowiednio
$\{ \eta_{i} \}_{i \in I}$ i
$\{ \xi_{j} \}_{j \in J}$,
algebraiczny iloczyn tensorowy
jest przestrzenią liniową $V \! \otimes \! W$, generowaną przez wektory
$\{\eta_{i} \otimes \xi_{j}\}_{i\in I, j \in J}$.
Iloczyn tensorowy przestrzeni Hilberta $\mathcal{H}_{1}$ i $\mathcal{H}_{2}$,
oznaczany przez $\mathcal{H}_{1} \bar{\otimes} \mathcal{H}_{2}$,
jest przestrzenią Hilberta będącą domknięciem przestrzeni liniowej
$\mathcal{H}_{1} \! \otimes \! \mathcal{H}_{2}$ w topologii indukowanej
przez iloczyn skalarny
$\langle \eta_{1} \otimes \xi_{1}, \eta_{2} \otimes \xi_{2} \rangle =
\langle \eta_{1} , \eta_{2} \rangle \, \langle \xi_{1}, \xi_{2} \rangle$,
$\eta_{1}, \eta_{2} \in \mathcal{H}_{1}$,
$\xi_{1}, \xi_{2} \in \mathcal{H}_{2}$.


\section{Algebry operatorowe}
W niniejszym podrozdziale,
w celu przytoczenia ogólnie przyjętych definicji
i znanych twierdzeń z teorii algebr operatorów,
posłużymy się kilkoma wybranymi publikacjami, między innymi
M.\,Takesaki, \emph{Theory of operator algebras}, t.\,1. \cite{Takesaki1} oraz
O.\,Bratteli, \emph{Operator Algebras and Quantum Statistical Mechanics}, t.\,1
\cite{Bratteli2003}.

\paragraph{C*-algebry.}
Struktura C*-algebry i algebry von Neumanna ma kluczowe znaczenie z punktu widzenia
zastosowań do opisu układów kwantowych.
Przytoczmy więc podstawowe definicje i własności tych obiektów.
Algebrą Banacha nazywamy przestrzeń Banacha $\mathcal{A}$,
tj. unormowaną przestrzeń liniową zupełną \cite{Rudin1991},
dla której mnożenie spełnia warunek
\begin{linenomath*}
 \begin{equation}
 \label{eq:algBanachMultiplication}
|| A B || \leq ||A|| \, ||B||,
 \end{equation}
\end{linenomath*}
dla wszystkich $A, B \in \mathcal{A}$.
Odwzorowanie $A \in \mathcal{A} \rightarrow A^{*} \in \mathcal{A}$ nazywamy
\emph{inwolucją} \index{inwolucja} lub
operacją sprzężenia na $\mathcal{A}$, jeśli spełnione są następujące warunki:
\begin{enumerate}
 \item $A^{**} = A$;
 \item $(AB)^{*} = B^{*} A^{*}$;
 \item $(\alpha A + \beta B)^{*} = %
  \overline{\alpha} A^{*} + \overline{\beta} B^{*}$
\end{enumerate}
Algebrę Banacha z inwolucją, dla której
$||A^{*}|| = ||A||$ dla każdego elementu
algebry, nazywamy *-algebrą Banacha.


\begin{Definition}
 \label{def:c*alg}
  \emph{C*-algebrą} nazywamy *-algebrę Banacha $\mathcal{A}$, dla której
\begin{linenomath*}
 \begin{equation}
 || A^{*} A || = ||A||^{2},
 \end{equation}
\end{linenomath*}
dla każdego $A \in \mathcal{A}$.
\end{Definition}

Definiując elementy dodatnie C*-algebry $\mathcal{A}$,
możemy wprowadzić relację częściowego porządku w $\mathcal{A}$.

\begin{Definition}
 Niech $\mathcal{A}$ będzie C*-algebrą. Element $A \in \mathcal{A}$ jest z
definicji \emph{dodatni}, co oznaczamy $A \geq 0$, jeżeli jest postaci
$A = B^{*}B$ dla pewnego $B \in \mathcal{A}$.
\end{Definition}

Opierając się na powyższej definicji, powiemy, że dla $A, B \in \mathcal{A}$,
$A \geq B$ wtedy i tylko wtedy, gdy $A-B \geq 0$. Ponadto jeżeli $-A
\geq 0$, to piszemy $A \leq 0$.

W przestrzeni $\mathcal{A}^{*}$ wszystkich ciągłych funkcjonałów na
C*-algebrze $\mathcal{A}$ możemy wyróżnić unormowane funkcjonały dodatnie,
czyli \emph{stany}.

\begin{Definition}
 \label{def:stateOnCalg}
  Funkcjonał liniowy $\omega$ na C*-algebrze $\mathcal{A}$ nazywamy
\emph{dodatnim}, jeżeli
\begin{linenomath*}
 \begin{equation}
 \label{eq:stateOnCalg}
  \omega(A^{*}A) \geq 0
 \end{equation}
\end{linenomath*}
dla wszystkich $A \in \mathcal{A}$.
Jeżeli dodatkowo $||\omega|| = 1$, to $\omega$ nazywamy \emph{stanem}.
Zbiór stanów algebry $\mathcal{A}$ oznaczamy przez $\mathcal{S}(\mathcal{A})$.
\end{Definition}

\paragraph{Topologie algebr operatorów.}
Niech $\mathcal{H}$ oznacza ośrodkową przestrzeń Hilberta, a $\mathcal{B}(\mathcal{H})$ przestrzeń
Banacha ograniczonych operatorów liniowych na $\mathcal{H}$.
Naturalną topologią na $\mathcal{B}(\mathcal{H})$ jest \emph{topologia jednostajna}, zadana
przez normę operatorową
\begin{linenomath*}
 \begin{equation}
 \label{eq:operatorNorm}
  ||A|| = %
  \sup \{ ||A\xi||; \, \, \xi \in \mathcal{H}, %
	    ||\xi|| = 1 \}, \quad A \in \mathcal{B}(\mathcal{H}).
 \end{equation}
\end{linenomath*}

Jeżeli $\xi \in \mathcal{H}$,
to odwzorowanie $A \rightarrow ||A\xi||$ zadaje
półnormę na $\mathcal{B}(\mathcal{H})$ dla każdego $\xi$.
Lokalnie wypukła topologia zdefiniowana za
pomocą rodziny tych półnorm nosi nazwę \emph{mocnej topologii}
(o definiowaniu
topologii przez rodzinę półnorm, zob. \cite{Rudin1991}).
Odpowiadająca topologia \emph{$\sigma$-mocna} zadana jest przez
rodzinę półnorm
$A \rightarrow \left( \sum \limits_{n=1}^{\infty} ||A \xi_{n}||^{2} \right)^{1/2}$
dla każdego ciągu $(\xi_{n})$,
$n \in \mathbb{N}$, takiego, że
$\sum \limits_{n=1}^{\infty} ||\xi_{n}||^{2} < \infty$.

Analogicznie do topologii mocnych, można za pomocą odpowiedniej rodziny
półnorm
zdefiniować na $\mathcal{B}(\mathcal{H})$ topologie \emph{słabą} i \emph{$\sigma$-słabą}.
I tak topologię słabą
wyznaczać będzie rodzina półnorm $A \rightarrow |\langle \xi, A \eta \rangle|$,
$\xi, \eta
\in \mathcal{H}$. Z kolei topologia $\sigma$-słaba jest wyznaczona
przez półnormy
$A \rightarrow \left| \sum \limits_{n=1}^{\infty} \langle \xi_{n}, A
\eta_{n}\rangle
\right|$ dla ciągów $(\xi_{n})$ i $(\eta_{n})$, takich że
$\sum \limits_{n=1}^{\infty} ||\xi_{n}||^{2} < \infty$ i
$\sum \limits_{n=1}^{\infty} ||\eta_{n}||^{2} < \infty$.

\paragraph{Algebry von Neumanna.}
Szczególną klasą C*-algebr są \emph{algebry von Neumanna},
dla których prosta definicja niesie za sobą niezwykle bogatą strukturę
(por. M.\,Takesaki, \emph{Theory of operator algebras}, t.1-3
\cite{Takesaki1, Takesaki2, Takesaki3}).

Dla dowolnej przestrzeni Hilberta $\mathcal{H}$,
niech $\mathfrak{M} \in \mathcal{B}(\mathcal{H})$.
Przez $\mathfrak{M}^{\prime}$ oznaczmy \emph{komutant} zbioru $\mathfrak{M}$,
tj. zbiór tych wszystkich $y \in \mathcal{B}(\mathcal{H})$,
dla których $x y = y x$ dla każdego $x \in \mathfrak{M}$.

\begin{Definition}
    \label{def:vNalgebras}
    Algebrą von Neumanna na $\mathcal{H}$ nazywamy $^{*}$-podalgebrę
    $\mathcal{B}(\mathcal{H})$, dla której
    \begin{linenomath*}
 \begin{equation}
        \mathfrak{M} = \mathfrak{M}^{\prime \prime}.
     \end{equation}
\end{linenomath*}
    \emph{Faktorem} nazwiemy taką algebrę von Neumanna, która posiada
    trywialne \emph{centrum}, tzn.
    $\mathfrak{M} \cap \mathfrak{M}^{\prime} = \mathbb{C} \mathbf{1}$.
\end{Definition}

Istnieje abstrakcyjna definicja algebr von Neumanna, bez odwoływania się
do istnienia reprezentacji jako algebry operatorów na przestrzeni Hilberta.
W literaturze zwykle tak zdefiniowane algebry noszą miano W*-algebr.
Powiemy, że C*-algebra $\mathfrak{M}$ posiada dodatkowo strukturę W*-algebry,
jeśli istnieje przestrzeń Banacha $X$, taka że
$X^{*} = \mathfrak{M}$,
gdzie $X^{*}$ oznacza przestrzeń funkcjonałów liniowych na $X$.
Prawdą jest, że dla każdej W*-algebry $\mathfrak{M}$ istnieje wierna,
$\sigma$-słabo ciągła reprezentacja $\mathfrak{M}$ jako pewnej algebry von Neumanna
(por. twierdzenie 3.5 \cite{Takesaki1}).
Zarówno dla W*-algebr, jak i algebr von Neumanna,
przestrzeń $X$ nazwiemy przestrzenią \emph{predualną} $\mathfrak{M}$,
i będziemy oznaczać przez $\mathfrak{M}_{*}$.
Jest to przestrzeń tych funkcjonałów liniowych na $\mathfrak{M}$, które
są dodatkowo ciągłe w $\sigma$-słabej topologii i dla których zarezerwujemy
nazwę funkcjonałów \emph{normalnych}.
W ogólności, zdefiniujmy normalne odwzorowania liniowe.
\begin{Definition}
    \label{def:normalMap}
    Niech $\mathfrak{M}, \mathfrak{N}$ będą algebrami von Neumanna.
    Odwzorowanie liniowe $S: \mathfrak{M} \rightarrow \mathfrak{N}$
    nazywamy \emph{normalnym}, jeśli
    jest ono ciągłe w $\sigma$-słabej topologii na $\mathfrak{M}$ i $\mathfrak{N}$.
\end{Definition}
Dla odwzorowania normalnego $S: \mathfrak{M} \rightarrow \mathfrak{N}$
istnieje odwzorowanie liniowe $S_{*}: \mathfrak{N}_{*} \rightarrow \mathfrak{M}_{*}$,
takie że $(S_{*} \varphi) (x) = \varphi(Sx)$ dla każdego
$x \in \mathfrak{M}$ i $\varphi \in \mathfrak{N}_*$.
Odwzorowanie $S_{*}$ nazwiemy \emph{predualnym} do $S$.

Ważnym obiektem w teorii algebr von Neumanna są projektory; powiemy, że
$p \in \mathfrak{M}$ jest projektorem, jeśli $p = p^{*} = p^{2}$.
Projektor $p$ jest \emph{centralny}, jeśli
$p$ należy do centrum algebry $\mathfrak{M}$, tzn.
$p \in \mathfrak{M} \cap \mathfrak{M}^{\prime}$.
W oparciu o analizę własności projektorów centralnych,
udało się scharakteryzować algebry von Neumanna.
Ta wiedza okaże się dla nas niezwykle przydatna, gdy przystąpimy do próby
uogólnienia kryterium Peresa-Horodeckich na szeroką klasę tzw. algebr iniektywnych
(p. definicja \ref{def:injectivevNalgebras}).

\begin{Definition}
Dwa projektory $p, q \in \mathfrak{M}$ są równoważne,
jeśli istnieje $u \in \mathfrak{M}$, takie że
$p = u^{*} u$ oraz $ q = u u^{*}$
Wówczas piszemy $p \sim q$.
\end{Definition}

Następujący ciąg definicji i twierdzeń, podanych tutaj bez dowodu,
pochodzi z podręcznika M.\,Takesaki, \emph{Theory of operator algebras}, t.\,1,
\cite{Takesaki1} (p. rozdz. V).

\begin{Definition}
    \label{def:projectionsInvN}
    Projektor $p \in \mathfrak{M}$ nazywamy \emph{skończonym}, jeśli
    dla każdego projektora $q$, takiego że $p \sim q \leq p$, mamy $p = q$.
    W przeciwnym wypadku projektor nazwiemy \emph{nieskończonym}.
    Jeżeli algebra $p \mathfrak{M} p$ jest przemienna,
    projektor $p$ również nazwiemy \emph{przemiennym}.
    Projektor $p$ jest \emph{minimalny}, jeśli $p \mathfrak{M} p = \mathbb{C} p$.
\end{Definition}

Jest jasne, że projektor przemienny jest skończony, a projektor minimalny jest przemienny.

\begin{Definition}
    \label{def:typesofvNalg}
    Algebra von Neumanna jest \emph{typu I}, jeżeli każdy niezerowy centralny
    projektor w $\mathfrak{M}$ majoryzuje pewien niezerowy projektor abelowy
    w $\mathfrak{M}$.
    Jeżeli dla algebry $\mathfrak{M}$ nie istnieją niezerowe skończone projektory,
    wówczas $\mathfrak{M}$ jest \emph{typu III}.
    Jeżeli $\mathfrak{M}$ nie posiada niezerowych przemiennych projektorów
    i jeśli każdy niezerowy centralny projektor w $\mathfrak{M}$ majoryzuje
    pewien niezerowy skończony projektor w $\mathfrak{M}$, wówczas
    $\mathfrak{M}$ jest \emph{typu II}.
    Jeżeli $\mathbf{1} \in \mathfrak{M}$ jest projektorem skończonym i
    $\mathfrak{M}$ jest typu II, wówczas mówimy, że $\mathfrak{M}$ jest
    \mbox{\emph{typu II$_{1}$}}.
    Jeżeli $\mathfrak{M}$ jest typu II i nie posiada niezerowych centralnych skończonych
    projektorów, wówczas mówimy, że $\mathfrak{M}$ jest \emph{typu II$_{\infty}$}.
\end{Definition}

\begin{Theorem}
    \label{thm:decompositionofvNalg}
    Każda algebra von Neumanna $\mathfrak{M}$ rozkłada się w sposób
    jednoznaczny na sumę prostą algebr typu I, II$_{1}$, II$_{\infty}$ lub typu III
    (składnik każdego z typów może występować co najwyżej raz).
    Jeżeli $\mathfrak{M}$ jest faktorem, wówczas $\mathfrak{M}$ jest dokładnie
    jednym z typów: I, II$_{1}$, II$_{\infty}$ albo III.
\end{Theorem}

\begin{Definition}
    \label{def:semifinitevNalg}
    Algebrę von Neumanna $\mathfrak{M}$ nazywamy \emph{skończoną}, jeżeli
    \mbox{$\mathbf{1} \in \mathfrak{M}$} jest projektorem skończonym.
    Jeżeli w rozkładzie na sumę prostą, jak w twierdzeniu \ref{thm:decompositionofvNalg},
    nie występuje składnik sumy typu III,
    wówczas algebrę $\mathfrak{M}$ nazywamy \emph{półskończoną}.
\end{Definition}

Posługując się przedstawionymi powyżej definicjami i twierdzeniami klasyfikującymi
algebry von Neumanna
względem własności ich projektorów,
będziemy mogli przeprowadzić dowód uogólnienia kryterium Peresa-Horodeckich
od przypadku algebr skończenie wymiarowych,
aż do egzotycznych faktorów typu III,
w kilku krokach -- za każdym razem w pewnym sensie
sprowadzając kolejny stopień ogólności do poprzedniego.
Pomocne w przeprowadzeniu tego typu dowodu okaże się założenie iniektywności
algebry von Neumanna, zapewniające możliwość przybliżania odwzorowań
elementów algebry przez odwzorowania skończonego rzędu
(p. str. \pageref{def:injectivevNalg}).

\paragraph{Iloczyny tensorowe algebr operatorów.}
Dla dwóch zespolonych przestrzeni liniowych $V$ i $W$,
przez $V \! \otimes \! W$ oznaczmy algebraiczny iloczyn tensorowy tych przestrzeni,
rozpięty przez skończone kombinacje liniowe wektorów postaci
$v \otimes w$, $v \in V$, $w \in W$.

W nieskończenie wymiarowych zupełnych przestrzeniach unormowanych,
jakimi są w ogólności algebry operatorów,
pojęcie iloczynu tensorowego musi zostać doprecyzowane, tak aby
był on na nowo unormowaną algebrą operatorów.
W przypadku C*-algebr nie można mówić o jedynym tego typu rozszerzeniu,
dopóki nie założy się czegoś więcej o przynajmniej jednym składniku iloczynu.
Definicję iloczynu tensorowego C*-algebr odłóżmy zatem do momentu,
kiedy będziemy próbowali przenieść na ten grunt wyniki udowodnione już
wcześniej dla algebr von Neumanna (p. str. \pageref{sec:HorCstar}).

Z kolei algebry von Neumann mają tę własność, że ich iloczyn
tensorowy można zdefiniować jednoznacznie,
co czyni je wygodnym narzędziem do opisu złożonych układów kwantowych
o nieskończonej ilości stopni swobody.
Niech $\mathfrak{M}$, $\mathfrak{N}$ będą algebrami von Neumanna,
a $\mathfrak{M}_{*}$, $\mathfrak{N}_{*}$ ich przestrzeniami predualnymi.
Wówczas istnieje dokładnie jedna algebra von Neumanna $\mathcal{A}$,
dla której
$\mathcal{A}_{*} = \mathfrak{M}_{*} \bar{\otimes} \mathfrak{N}_{*}$,
gdzie przestrzeń
$\mathfrak{M}_{*} \bar{\otimes} \mathfrak{N}_{*}$ jest domknięciem
algebraicznego iloczynu tensorowego przestrzeni
$\mathfrak{M}_{*}$, $\mathfrak{N}_{*}$ w naturalnej normie na iloczynie
tensorowym $\mathfrak{M}^{*} \otimes \mathfrak{N}^{*}$,
dokładniej w normie sprzężonej do normy iniektywnej
(por. \cite{Takesaki1}, definicja 5.1, str. 220, oraz niżej,
str.\,\pageref{def:projectiveoperatornorm}).

\begin{Definition}
 \label{def:TensorProductOfvN}
    Algebrę von Neumanna $\mathcal{A}$, dla której
    $\mathcal{A}_{*} = \mathfrak{M}_{*} \bar{\otimes} \mathfrak{N}_{*}$,
    nazywamy iloczynem tensorowym algebr $\mathfrak{M}$, $\mathfrak{N}$
    i oznaczamy symbolem
    $\mathfrak{M} \bar{\otimes} \mathfrak{N}$.
\end{Definition}

Dzięki definicyjnej własności iloczynu tensorowego algebr von Neumanna
możliwe będzie wprowadzenie ścisłej definicji zbioru stanów separowalnych
(por. str. \pageref{def:SeparableStates234}).

\section{Odwzorowania dodatnie}
Analiza odwzorowań dodatnich na algebrach operatorów stanowi główny przedmiot
tej rozprawy.
Jak już zostało wspomniane powyżej, znaczenie tego typu odwzorowań dla kwantowej
teorii informacji kryje się w nieoczekiwanym związku pomiędzy,
z jednej strony odwzorowaniami dodatnimi,
z drugiej zaś obserwablami umożliwiającymi niejako wykrycie splątania w
danym stanie kwantowego układu złożonego.
Precyzyjne sformułowanie tej zależności nazwiemy (uogólnionym)
\emph{kryterium Peresa-Horodeckich} i sformułujemy je następująco.
Jeżeli $\varphi \in (\mathfrak{M} \bar{\otimes} \mathfrak{N})_{*}$ jest stanem
na iloczynie tensorowym algebr von Neumanna
$\mathfrak{M}$ i $\mathfrak{N}$,
spośród których przynajmniej jedna algebra jest iniektywna,
wówczas $\varphi$ jest stanem separowalnym wtedy i tylko wtedy,
gdy $\varphi \circ (\mathbf{1} \otimes S)$ jest dodatnim funkcjonałem
dla każdego dodatniego i normalnego odwzorowania
$S: \mathfrak{M} \rightarrow \mathfrak{N}$
skończonego rzędu.
Dowód tego faktu zostanie podany w następnym rozdziale
(por. twierdzenie \ref{thm:PHcrit}).
W tym miejscu przytoczymy najważniejsze informacje dotyczące dodatnich
odwzorowań, niezbędne w dalszej części wywodu.

\begin{Definition}
Niech $\mathfrak{M}$, $\mathfrak{N}$ będą algebrami von Neumanna.
Powiemy, że normalne odwzorowanie liniowe
$S: \mathfrak{M} \rightarrow \mathfrak{N}$
jest \emph{dodatnie},
jeśli dla każdego $z \in \mathfrak{M}$,
takiego że $z \geq 0$, mamy
$S(z) \geq 0$.
\end{Definition}

Dla $k \in \mathbb{N}$, dodatnie odwzorowanie $S$,
jest z definicji \emph{k-dodatnie}, jeżeli
odwzorowanie
$\mathbf{1}_{k} \otimes S$
jest dodatnie na $M_{k} \bar{\otimes} \mathfrak{M}$.
Oczywiście, $S$ jest dodatnie wtedy i tylko wtedy, gdy jest $1$-dodatnie.
Jeżeli $S$ jest $k$-dodatnie dla każdego $k \in \mathbb{N}$,
wówczas powiemy, że $S$ jest \emph{kompletnie} dodatnie.
W skończenie wymiarowym przypadku
odwzorowania kompletnie dodatnie zostały scharakteryzowane przez M.-D.\,Choi
w pracy \emph{Completely positive linear maps on complex matrices}
\cite{choi1975completely}.
Jeżeli $S$ jest odwzorowaniem kompletnie dodatnim pomiędzy algebrami macierzy,
$S: M_{m} \rightarrow M_{n}$, wówczas istnieją macierze prostokątne
$V_{i} \in M_{m \times n}$ rozmiaru ,,$m$ na $n$'', dla $i = 1,2,\ldots,N$,
takie że
\begin{linenomath*}
 \begin{equation}
    S(A) = \sum \limits_{i = 1}^{N} V_{i}^{*} A V_{i}.
 \end{equation}
\end{linenomath*}
Dowód tego faktu wykorzystuje istnienie izomorfizmu,
znanego jako izomorfizm Choi-Jamiołkowskiego
\cite{choi1975completely, jamiolkowski1974effective}:
\begin{Theorem}
    Niech $S: M_{m} \rightarrow M_{n}$ będzie odwzorowaniem liniowym
    oraz niech $E_{ij}$, $i,j = 1,2,\ldots,m$ będą macierzami, których
    elementy zadane są przez deltę Kroneckera:
    $(E_{ij})_{kl} = \delta_{ik} \delta_{jl}$.
    Wówczas $S$ jest odwzorowaniem kompletnie dodatnim,
    wtedy i tylko wtedy, gdy macierz
    \begin{linenomath*}
 \begin{equation}
        \sum \limits_{i,j=1}^{m} (\mathbf{1}_{m} \otimes S)
            (E_{ij}^{*} \otimes E_{ij}) \in M_{m} \otimes M_{n}
     \end{equation}
\end{linenomath*}
    jest macierzą dodatnią.
\end{Theorem}
%W następnym rozdziale zobaczymy, jak twierdzenie analogiczne do powyższego
%uogólnia się na odwzorowania liniowe na algebrach von Neumanna
%(zob. lemat \ref{lem:netininjectivefactor}).

Powiemy, że odwzorowanie dodatnie $S$ jest \emph{rozkładalne}, jeżeli
 $S(A) = \Lambda_{1}(A) + \Lambda_{2} (A^{t})$,
 dla każdego $A$,
gdzie $\Lambda_{1}, \Lambda_{2}$ to odwzorowania kompletnie dodatnie
(które mogą być zerowe).
Wiadomo, że wszystkie odwzorowania dodatnie algebry $M_{2}$
oraz pomiędzy algebrami $M_{2}$ i $M_{3}$ są rozkładalne
\cite{Stormer2013}.
Jednak już w przypadku odwzorowań $M_{3}$ można podać przykłady takich,
które nie są rozkładalne.

W części poświęconej odwzorowaniom dodatnim na algebrach macierzy (p. rozdz. 3.)
będziemy zajmować się analizą ekstremalnych odwzorowań dodatnich.
Dla uproszczenia,
skupmy uwagę na odwzorowaniach na jednej tylko algebrze macierzowej $M_{n}$.
Powiemy, że dodatnie odwzorowanie $S: M_{n} \rightarrow M_{n}$ jest \emph{ekstremalne},
jeżeli dla każdego dodatniego $T: M_{n} \rightarrow M_{n}$, takiego że
$S - T \geq 0$, tzn. $0 \leq T \leq S$,
mamy $T = \alpha S$ oraz $0 \leq \alpha \leq 1$.
Jest prawdą, że każde dodatnie odwzorowanie na algebrze macierzy można zapisać
jako skończoną kombinację wypukłą odwzorowań ekstremalnych.
Jest to konsekwencja słynnego twierdzenia Kreina-Milmana
(por. twierdzenie 3.23 w książce W.\,Rudina \emph{Functional analysis}
\cite{Rudin1991})
Warto zaznaczyć, że zbiór odwzorowań dodatnich tworzy stożek,
a punkty ekstremalne tego stożka należą do jego brzegu.
Dla $n \geq 3$ zbiór odwzorowań ekstremalnych okazuje się mieć bardzo
skomplikowaną strukturę.
Dlatego wprowadźmy dodatkową definicję odwzorowań ,,wystających''
(ang. \emph{exposed}).
Odwzorowanie dodatnie $S: M_{n} \rightarrow M_{n}$ jest wystające,
jeżeli dla każdego dodatniego odwzorowania $T: M_{n} \rightarrow M_{n}$,
takiego że $\text{Tr}\, P_{\xi} \, T(P_{\eta}) = 0$,
gdzie $\mathbb{C}^{n} \ni \eta, \xi \neq 0$ jest parą wektorów,
dla których projektory ortogonalne $P_{\xi}, P_{\eta}$ na jednowymiarowe
podprzestrzenie rozpięte przez $\xi$ i $\eta$ dają
$\text{Tr}\, P_{\xi} \, S(P_{\eta}) = 0$;
wówczas mamy $T = \alpha S$, $\alpha \geq 0$.
Dzięki ogólnemu twierdzeniu udowodnionemu przez Straszewicza
\cite{straszewicz1935exponierte},
zbiór odwzorowań wystających jest gęstym podzbiorem zbioru odwzorowań
ekstremalnych $M_{n}$.

\paragraph{}
W następnym rozdziale wykorzystamy przedstawione powyżej podstawowe fakty
matematyczne do precyzyjnego sformułowania i udowodnienia wyników związanych
z uogólnieniem kryterium Peresa-Horodeckich dla stanów złożonych układów
kwantowych opisywanych w języku algebr von Neumanna.
Zobaczymy,
w jaki sposób znajomość struktury tych algebr pozwoli na przeprowadzenie
pełnego dowodu w kilku krokach, dla obiektów o coraz większym stopniu ogólności:
poczynając od algebr skończenie wymiarowych,
a kończąc na algebrach typu III.
