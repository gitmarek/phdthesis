\chapter{Przygotowanie}
\label{chp:preliminaries}

\paragraph{}
Metody matematyczne,
niezbędne do badania kryterium splątania nieskończenie wymiarowych złożonych
układów kwantowych, koncentrują się wokół zastosowań algebr operatorowych,
w szczególności algebr von Neumanna,
oraz analizy struktury tzw. dodatnich odwzorowań na tych algebrach.

W tym rozdziale zajmiemy się przedstawieniem podstawowych informacji dotyczących
wykorzystywanego w dalszym wywodzie aparatu matematycznego.
Poczynając od prezentacji najważniejszych wyników z teorii algebr operatorowych,
w tym algebr von Neumanna oraz algebr Jordana,
przejdziemy
w dalszej kolejności do omówienia podstawowych własności
odwzorowań dodatnich na tych algebrach.
% Na zakończenie rozdziału pokusimy się o przedstawienie podstawowych zastosowań
% prezentowanych metod do kwantowej teorii informacji.
Rozdział ten ma służyć również ustaleniu spójnej notacji obowiązującej poniżej.

\paragraph{}
Niech $n \in \mathbb{N}$ będzie liczbą naturalną.
Niech także $\mathbb{C}^{n}$ oznacza zespoloną $n$-wymiarową przestrzeń liniową,
a $\mathbb{R}^{n}$ przestrzeń rzeczywistą --
wyposażoną w naturalny iloczyn skalarny.
Dla ogólnej przestrzeni Hilberta,
niekoniecznie skończenie wymiarowej,
zarezerwujmy symbol $\mathcal{H}$.
Wektory należące do $\mathcal{H}$ oznaczamy
$\eta, \xi, \upsilon$ itd.
Dla odróżnienia od wektorów należących do innej przestrzeni,
będziemy czasem używać strzałek: $\vec{\eta}, \vec{\xi}, \vec{\upsilon}$ itd.
Dla podprzestrzeni $\mathbb{C}^{n}$,
$\eta, \xi$ itd. oznaczają zawsze wektory-kolumny;
wektory-wiersze będziemy zapisywać
$\eta^{t}, \xi^{t}$ itd.
Kreska nad symbolem wektora: $\bar{\eta}, \bar{\xi}$ itd.,
oznaczać będzie zawsze zespolone sprzężenie poszczególnych współrzędnych wektora.
Naturalny iloczyn skalarny w $\mathbb{C}^{n}$ oznaczamy
$\langle \eta , \xi \rangle = \eta^{*} \xi$,
gdzie $\eta^{*} = \bar{\eta}^{t}$.
Operatory liniowe na $\mathbb{C}^{n}$ lub $\mathbb{R}^{n}$,
a więc macierze,
będziemy oznaczać zwykle $A, B, C$ itd.
Przestrzeń Banacha wszystkich operatorów liniowych,
ograniczonych w standardowej normie operatorowej,
będziemy oznaczać $\mathcal{B}(\mathcal{H})$.
Dla $A \in \mathcal{B}(\mathcal{H})$,
symbol $A^{*}$ oznacza sprzężenie Hermitowskie operatora.
Przestrzeń $\mathcal{B}(\mathcal{H})$ z antyliniową operacją
$A \mapsto A^{*}$ oraz z operacją składania operatorów
tworzy szczególny przykład \mbox{$^{*}$-algebry} Banacha.
Algebrę macierzy kwadratowych rozmiaru $n$ nad ciałem liczb
rzeczywistych lub zespolonych,
a więc gdy $\mathcal{H} = \mathbb{C}^{n}$ lub
$\mathcal{H} = \mathbb{R}^{n}$,
będziemy dodatkowo oznaczać symbolem $\mathcal{M}_{n}(\mathbb{R})$ lub
$\mathcal{M}_{n}(\mathbb{C})$;
zawsze $\mathcal{M}_{n} = \mathcal{M}_{n}(\mathbb{C})$.
Gdy $A \in \mathcal{M}_{n}$,
$A^{t}$ oznacza transpozycję macierzy,
a $\text{tr} A$ jej ślad.
Dla ogólnych przestrzeni liniowych $V$ i $W$,
których bazy to odpowiednio
$\{ \eta_{i} \}_{i \in I}$ i
$\{ \xi_{j} \}_{j \in J}$,
algebraiczny iloczyn tensorowy
jest przestrzenią liniową $V \! \otimes \! W$, generowaną przez wektory
$\{\eta_{i} \otimes \xi_{j}\}_{i\in I, j \in J}$.
Iloczyn tensorowy przestrzeni Hilberta $\mathcal{H}_{1}$ i $\mathcal{H}_{2}$
oznaczany  przez $\mathcal{H}_{1} \overline{\otimes} \mathcal{H}_{2}$,
jest przestrzenią Hilberta będącą domknięciem przestrzeni liniowej
$\mathcal{H}_{1} \! \otimes \! \mathcal{H}_{2}$ w topologii indukowanej
przez iloczyn skalarny
$\langle \eta_{1} \otimes \xi_{1}, \eta_{2} \otimes \xi_{2} \rangle =
\langle \eta_{1} , \eta_{2} \rangle \, \langle \xi_{1}, \xi_{2} \rangle$,
$\eta_{1}, \eta_{2} \in \mathcal{H}_{1}$,
$\xi_{1}, \xi_{2} \in \mathcal{H}_{2}$.

Dla ogólnych przestrzeni Banacha $X$ i $Y$, przez
$\mathcal{L}(X,Y)$
oznaczmy przestrzeń Banacha wszystkich ograniczonych odwzorowań
liniowych pomiędzy z $X$ w $Y$.

% Istnieje wiele norm,
% w które można wyposażyć iloczyn tensorowy ogólnych przestrzeni Banacha $X$ i $Y$,
% tak aby uzyskać również przestrzeń Banacha.
% Tę wieloznaczność można jednak wyeliminować w przypadku niektórych algebr operatorów.
%



%%%%%%%%%%%%%%%%%%%%%%%%%%%%%%%%%%%
%%  DO PRZYGOTOWANIA MAT %%%%%%%%%%
%%%%%%%%%%%%%%%%%%%%%%%%%%%%%%%%%%%
%
% The pairing between $x \in X$ and $\varphi \in X^{*}$ is denoted by
% $\langle  \varphi , x \rangle$. We say that a net $(T_{\lambda})$,
% $T_{\lambda} \in \mathcal{L}(X, Y^{*})$, converges
% to $T \in \mathcal{L}(X, Y^{*})$ in the \emph{weak* operator topology} \label{page:weakstaroperatortop}, if
% $\langle T_{\lambda} x, y \rangle \rightarrow \langle Tx, y \rangle$
% for any $x \in X$ and $y \in Y$. If $X$ and $Y$ are also operator spaces,
% we denote by $\mathcal{CB}(X,Y)$ the operator space
% of completely bounded linear maps equipped with the norm
% $|| \cdot ||_{\mathcal{CB}}$. It is true that the space $\mathcal{F}(X,Y)$
% of finite rank maps from $X$ into $Y$, i.e. the maps with their image being finite dimensional
% subspaces of $Y$, is a subspace of $\mathcal{CB}(X,Y)$.
% By $\mathcal{CB}(X,Y)_{+}$ we denote the cone of completely bounded positive maps from $X$ and $Y$,
% and by $\mathcal{CP}(X,Y)$ the cone of completely positive maps.
% It is known that $\mathcal{CP}(X,Y) \subset \mathcal{CB}(X,Y)_{+}$,
% (see the book by Effros and Ruan \cite{Effros2000} for a general outline of the theory of completely bounded and completely positive maps).
%
%
%
% Let $\mathfrak{M}$, $\mathfrak{N}$ be arbitrary von Neumann algebras. Without any loss of generality we may assume
% that $\mathfrak{M}\subset B(\mathcal{H})$ and $\mathfrak{N}\subset B(\mathcal{K})$, where $\mathcal{H},\mathcal{K}$
% are Hilbert spaces, not necessarily separable. Their state spaces, i.e. convex sets of normal, positive and
% normalized functionals, we denote by $\mathcal{S}(\mathfrak{M})$ and $\mathcal{S}(\mathfrak{N})$, respectively.
% The algebraic tensor product
% is denoted by $\mathfrak{M}\otimes\mathfrak{N}$; whereas the spatial tensor product of
% $\mathfrak{M}$ and $\mathfrak{N}$ by $\tmn$, i.e. $\tmn=(\mathfrak{M}\otimes\mathfrak{N})''$.
% Notice that $(\tmn)_*$ is isometrically isomorphic with $\mathfrak{M}_*\hat{\otimes}\mathfrak{N}_*$, the completion
% of $\mathfrak{M}_*\otimes\mathfrak{N}_*$ with respect to the projective operator norm $\|\cdot\|_\wedge$,
% and so $(\mathfrak{M}_*\hat{\otimes}\mathfrak{N}_*)^*=\tmn$, with the operator norm
% on $\tmn$ being the dual norm on the dual space
% $(\mathfrak{M}_*\hat{\otimes}\mathfrak{N}_*)^*$
% (see again the book by Effros and Ruan \cite{Effros2000}, Theorem 7.2.4;
% and also Corollary 7.1.5 for the proof of the next theorem).
% Moreover, one has the following.
% \begin{theorem}
% \label{thm:isometry}
% The map $F: \mathcal{CB}(\mathfrak{M}_{*}, \mathfrak{N})
% \rightarrow \mathfrak{M} \bar{\otimes} \mathfrak{N}$,
% given by
% \begin{equation}
% \langle F(T), \omega \otimes \varphi \rangle =\langle T\omega, \varphi \rangle,
% \end{equation}
% where $\omega \in \mathfrak{M}_{*}$, $\varphi \in \mathfrak{N}_{*}$,
% extends to a completely isometric isomorphism between operator spaces.
% \end{theorem}
% Let us recall that a density matrix acting on $\mathcal{H}\otimes\mathcal{K}$, where $\mathcal{H}$ and $\mathcal{K}$
% are finite dimensional Hilbert spaces, is separable, if it is a convex combination of tensor products of density
% matrices in $\mathcal{H}$ and $\mathcal{K}$. The definition was next generalized by Werner \cite{Werner} to infinite
% dimensional Hilbert spaces by taking the trace norm limits of such convex combinations. In the same spirit, we define
% a convex set $C_1$ of separable states in $\mathcal{S}(\tmn)$ by
% \begin{equation}
% C_{1} = \overline{{\rm conv}}^{||\cdot||_{\wedge}}
% \left \{\omega \otimes \varphi: \, \,\omega \in \mathcal{S}(\mathfrak{M}),
% \varphi \in \mathcal{S}(\mathfrak{N})\right \}.
% \end{equation}
% By $C_{1}^{*}$ we denote the dual cone: $C_{1}^{*} = \{ x \in
% \mathfrak{M} \bar{\otimes} \mathfrak{N}:
% \langle x , \varphi \rangle \geq 0 \,\, \forall \varphi \in C_{1} \}$, and
% by $C_{0}^{*}$ the subcone of $C_{1}^{*}$: $C_{0}^{*} = \{ x \in
% \mathfrak{M} \otimes \mathfrak{N}:\langle x , C_{1} \rangle \geq 0 \}$. It follows easily from Theorem \ref{thm:isometry} that
% $F(\mathcal{CB}(\mathfrak{M}_{*}, \mathfrak{N})_{+}) = C_{1}^{*}$ and
% $F(\mathcal{CP}(\mathfrak{M}_{*}, \mathfrak{N})) = (\mathfrak{M} \bar{\otimes} \mathfrak{N})_{+}$,
% the cone of positive elements.
%
% For any von Neumann algebra $\mathfrak{M}$,
% there is also a natural completely isometric embedding
% \begin{equation}
% \theta: \mathfrak{M} \check{\otimes} \mathfrak{M}_{*}\rightarrow
% \mathcal{CB}(\mathfrak{M}_{*}, \mathfrak{M}_{*}),
% \end{equation}
% where $\check{\otimes}$ stands for injective operator space tensor product
% (see again \cite{Effros2000}, Proposition 8.1.2).
% For instance, if
% $u \in \mathfrak{M} \otimes \mathfrak{M}_{*}$, $u = \sum_{i = 1}^{n} a_{i} \otimes \omega_{i}$,
% then for $\omega \in \mathfrak{M}_{*}$,
% \begin{equation}
% \theta(u)\omega = \sum_{i=1}^{n} \langle a_{i} , \omega \rangle \, \omega_{i}.
% \end{equation}
% Using the map $F$, we show the following.
% \begin{proposition}
% \label{prop:34523}
% Let $\omega\in \mathcal{S}(\tmn)$. Then $\langle F(A),\omega\rangle\geq 0$ $\forall A\in\cbms_+$, if and only if $\omega$ is separable.
% \end{proposition}
% {\bf Proof.} $\Rightarrow$ Suppose that $\omega\notin C_1$. Because $C_1$ is norm closed and
% convex, so there exists $x\in\tmn$ such that $\langle x,
% \omega\rangle<0$ and $\langle x,C_1\rangle\geq 0$. Therefore, $x\in C_1^*$ and so there exists $A\in\cbms_+$
% such that $\langle F(A),\omega\rangle<0$, a contradiction.\\
% $\Leftarrow$ This part is clear. $\Box$\\
% \begin{proposition}
% Let $E\subset\cbms_+$ and let $B_{1}$ be a unit ball of $\cbms$ (with respect to the $\mathcal{CB}$-norm). Let also $E \cap B_{1}$  be dense in  $\cbms_+ \cap B_{1}$ in the
% weak$^*$ operator topology. Suppose that $\omega_0\in \mathcal{S}(\tmn)$. If $\forall A\in E$
% $\langle F(A),\omega_0\rangle \geq 0$, then $\omega_0\in C_1$.
% \end{proposition}
% {\bf Proof.} At first, we show that $F(E)$ is weak$^*$ dense in $C_1^*$. Let $x\in C_1^*$. Then $x=F(A)$, $A\in\cbms_+$.
% By the assumption, there exists a net $(A_{\lambda})$, $A_{\lambda}\in E$, $\sup_{\lambda}\|A_{\lambda}\|_{\mathcal{CB}}\leq
% \|A\|_{\mathcal{CB}}$ such that $A_{\lambda}\to A$ in the weak$^*$ operator topology (see p. \pageref{page:weakstaroperatortop}).
% The map $F$ is continuous with respect to that topology because of Theorem \ref{thm:isometry}.
% Therefore,  for any $\omega\in
% \mathfrak{M}_*\otimes\mathfrak{N}_*$ $\langle F(A_{\lambda}),\omega\rangle\to\langle x,\omega\rangle$. Because
% $\|F(A_{\lambda})\|=\|A_{\lambda}\|_{\mathcal{CB}}$ is uniformly bounded so $F(A_{\lambda})$ converges to $x$ in the
% weak$^*$ topology of von Neumann algebra $\tmn$ . Hence $\langle x,\omega_0\rangle\geq 0$ for any $x\in C_1^*$, and so $\omega_0\in C_1$. $\Box$\\
% It follows that a kind of positive uniform approximation in $\mathfrak{M}$ or $\mathfrak{N}$ is necessary to accomplish our task of generalising the Horodeckis' theorem.
% This leads to the assumption about injectivity of one (say $\mathfrak{M}$) of the algebras. Let us recall that a von Neumann
% algebra $\mathfrak{M}\subset B(\mathcal{H})$ is called injective if there is a norm-one Banach space projection (not necessarily
% normal) $\pi :B(\mathcal{H})\to\mathfrak{M}$ onto $\mathfrak{M}$. In such an algebra, there is a net $(A_{\lambda})$ of completely
% positive finite rank contractions on $\mathfrak{M}_*$ converging strongly to the identity map (see the Takesaki's textbook \cite{Takesaki3}, vol III, Theorem XV.3.1):
% \begin{equation}
% \label{RandomLabel:513684}
%     || A_{\lambda} \omega - \omega || \stackrel{\lambda}{\longrightarrow} 0, \quad
% \end{equation}
% or any $\omega \in \mathfrak{M}_{*}$.
% This property of injective von Neumann algebras will prove itself of particular
% use in the following.
% %%%%%%%%%%%%%%%%%%%%%%%%%%%%%%%%%%%%%%%%%%%%


\section{Algebry operatorowe}

% Wykorzystane Definicje i twierdzenia:
%
% 1. Algebry operatorów, przestrzenie operatorowe.
% 2. Algebry C*
% 3. Algebry von Neumanna,
% 4. przestrzeń predualna, funkcjonały i odwzorowania normalne i predualne
% 5. topologie: silna słaba, *-słaba, $\sigma$-topologie
% 6. iloczyny tensorowe C*-algebr
% 7. iloczyny tensorowe alg. vN.
% 8. projektory w alg vN., centrum alg.
% 9. klasyfikacja alg. vN.,  algebry skończone (nie: skończenie wymiarowe), półskończone, typu III
% 10. algebry iniektywne,
% 11. C0, C1 stożki
% 12. stany separowalne, def.
% 13. centrum algebry
%

\section{Odwzorowania dodatnie}

% Wykorzystane Definicje i twierdzenia:
%
% 1. definicja
% 2. odwzorowania kompletnie dodatnie
% 3. odwz. kompletnie ograniczone
% 3. odwz. ekstremalne, typu exposed
% 4. izomorfizm Choi-Jamiołkowskiego
% 5. definicja zbiorów $C_{0}$ itd.
% 6. stożki dualne
% 7. Przygotowanie do dowodu twierdzenia w rozdz. M3notes (w tym rozdziale)
% 8. JB* algebry
% 9. Odwzorowania rozkładalne

\paragraph{}
W następnym rozdziale wykorzystamy przedstawione powyżej podstawowe fakty
matematyczne do precyzyjnego sformułowania i udowodnienia wyników związanych
z uogólnieniem kryterium Peresa-Horodeckich dla stanów złożonych układów
kwantowych opisywanych w języku algebr von Neumanna.
Zobaczymy,
w jaki sposób znajomość struktury tych algebr pozwoli na przeprowadzenie
pełnego dowodu w kilku krokach, dla obieków o coraz większym stopniu ogólności:
poczynając od algebr skończenie wymiarowych,
a kończąc na egzotycznych algebrach typu III.

% ### Plan szczegółowy rozdziału:
% 1. Wstęp fizyczny cz. 1: uzasadnienie opisu złożonych układów fizycznych w algebraicznym języku alg. vN.
% 2. Wstęp fizyczny cz.2: badanie splątania nierozerwalnie związane z badaniem dodatnich odwzorowań na algebrach macierzy, kryterium P.-H.
% 3. Wstęp mat: algebry vN; alg. Jordana;
% 4. Odwzorowania dodatnie, kompletnie dodatnie, kompletnie ograniczone itd.; algebry operatorowe;
% 5. Wstęp fizyczny cz.3: dalej o kryt. splątania: *entanglement wittness*, izomorfizm Choi-Jamiołkowski (odniesienie do kolejnych rozdziałów); znów zwrócenie uwagi na to, w jakim kierunku będzie zmierzać prezentacja naszych wyników.

% Na początek rozdziału 3:
% Jak już zostało wspomniane powyżej,
% podstawowym narzędziem, dzięki któremu możemy rozstrzygnąć,
% czy dany stan złożonego układu kwantowego jest splątany,
% jest kryterium odkryte przez Peresa i Horodeckich
% \cite{peres1996separability, horodecki1996separability},
% stwierdzające, że stan kwnatowy $\phi$,
%


\paragraph{}
