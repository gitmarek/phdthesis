\documentclass[12pt]{article}

\usepackage[utf8]{inputenc}


\usepackage{amsmath,amsthm,amssymb}
\usepackage{mathrsfs}
\usepackage{authblk}

\theoremstyle{plain}
\newtheorem{theorem}{Theorem}
\newtheorem{lemma}{Lemma}
\newtheorem{proposition}{Proposition}
\theoremstyle{definition}
\newtheorem{definition}{Definition}
\newtheorem{corollary}{Corollary}
\theoremstyle{remark}
\newtheorem{remark}{Remark}
\newtheorem*{remark*}{Remark}

\numberwithin{equation}{section}

\usepackage[margin=25mm]{geometry}

\usepackage{fancyhdr}
\rfoot{\thepage}

\begin{document}

\title{Extremal positive maps on $M_{3}(\mathbb{C})$ \\ and idempotent matrices}

\author[1]{Marek Miller\thanks{marek.miller@ift.uni.wroc.pl}}
\author[1]{Robert Olkiewicz\thanks{robert.olkiewicz@ift.uni.wroc.pl}}
\affil[1]{Institute of Theoretical Physics, Uniwersytet Wroc{\l}awski, Poland}

\date{\today}

\maketitle

% Abstract and Introduction
\abstract{
A new method of analysing positive bistochastic maps on the algebra of complex
matrices $M_{3}$ has been proposed.
By identifying the set of such maps with a convex set of linear operators on
$\mathbb{R}^{8}$, one can employ techniques from the theory of 
compact affine semigroups to obtain results concerning asymptotic properties
of positive maps.
It turns out that the idempotent elements play a crucial role
in classifying the convex set into subsets, in which representations of
extremal positive maps are to be found.
It has been show that all extremal positive bistochastic maps that are not
Jordan isomorphisms of $M_3$ are represented by matrices that fall into two possible
categories, determined by the simplest idempotent matrices:
one by the zero matrix, and the other by a one dimensional orthogonal projection.
Some norm conditions for matrices representing possible extremal maps
have been specified and examples of maps from both categories have been
brought up, based on the results published previously.

\vspace{0.1cm}
{\scriptsize \noindent \textbf{Keywords:}
positive maps, extremal, idempotent, semigroup.}

%{\scriptsize \noindent \textbf{MSC2010 codes.}
%47H07  15B48  81P40}
}


\section*{Introduction}
\label{sec:Introduction}

\paragraph{}
Positive maps of operator algebras constitute a rich area of research, directly
connected to the theory of quantum entanglement.
In 1990s,
A.\,Peres and P.\,M.\,R. {Horo-deckis} pointed out at the intrinsic relation
between separable states of composite quantum systems
and positive maps of algebras of observables
\cite{peres1996separability,horodecki1996separability}.
The well established criterion of separability, proposed in the mentioned papers,
reveals a one-to-one correspondence between
positive maps and entanglement witnesses
\cite{chruscinski2014entanglement}.
% It is necessary to mention that their work was based on the previous
% results obtained by \mbox{M.-D.\,Choi}
% \mbox{\cite{choi1975completely,choi1980some,choi1977extremal}}.
This \emph{Peres-Horodecki criterion},
originally proposed for maps on algebras $M_{n}$ of square complex matrices of size $n$,
which holds true even in the most general setting of
injective von Neumann algebras \cite{miller2014horodeckis},
is computationally feasible as long as the structure of general positive
maps on operator algebras representing a composite quantum system in 
question is known.
To this day, the complete characterisation 
of positive maps have been obtained only for the algebra $M_{2}$
and the maps between $M_{2}$ and $M_{3}$
\cite{stormer1963positive,woronowicz1976positive}.

To analyse the structure of positive maps in the simplest, still unresolved
case of maps on $M_{3}$,
we propose a continuation of the reasoning conducted in our previous paper
\cite{miller2015stable}.
We have established a connection between positive maps that preserve both
the trace of matrices and the identity matrix, the so-called bistochastic
maps, and their stable subspaces that have the structure of Jordan algebras.
Here, we go one step further and explore the relation between those stable
subspaces and the idempotent real matrices that represent the conditional
expectations projecting onto the spaces.
To this end, we employ mostly geometrical techniques that allowed us before to
establish the structure theorem for maps on the algebra $M_{2}$
\cite{miller2015topology},
as well as the methods from the theory of compact affine semigroups
\cite{schwarz1955hausdorff,chow1975compact}.

The main result of this paper makes possible to outline a program,
suggesting where the extremal positive maps on $M_{3}$ are to be found
with respect to their connection to associated idempotents
(see Theorem \ref{thm:LowerIndices} below).
We have shown that all extremal positive bistochastic maps that are not
Jordan isomorphisms of $M_3$ are represented by matrices that fall into two possible
categories, determined by the simplest idempotent matrices:
one by the zero matrix, and the other by a one dimensional orthogonal projection.
As a corollary, we specify some norm conditions for matrices representing possible
extremal maps.
The structure of the paper concentrates on building a mathematical framework
necessary to prove the final result.
We start with a brief listing of necessary notation and definitions.

%%%%%%%%%%%%%%%%%%%%%
%%% Preliminaries %%%
%%%%%%%%%%%%%%%%%%%%%

\section{Preliminaries}
\label{sec:Preliminaries}

\paragraph{}
Let $n, k \in \mathbb{N}$ be grater than 0.
Let $M_{n} = M_{n}(\mathbb{C})$ denote the algebra of complex square
matrices of size $n$.
The algebra of real matrices will always be denoted explicitly by
$M_{n}(\mathbb{R})$.
For $A \in M_{n}$, the norm $||A||$ is understood to be the standard operator
norm, i.e. the maximal singular value of the matrix $A$.
For the Hilbert-Schmidt norm (HS-norm) of $A$,
we reserve the symbol
$||A||_{HS} = \sqrt{\text{tr} A^{*} A}$,
where $\text{tr}$ denotes the trace operation,
and $A^{*}$ is the conjugate-transpose of $A$.
The HS-norm of $A$ can be computed as the sum of squares of
singular values of $A$.
The Hilbert-Schmidt inner product, induced by the HS-norm,
is defined as
$\langle A, B \rangle_{HS} = \text{tr} A^{*} B$,
for $A, B \in M_{n}$.
The identity matrix of $M_{n}$, we denote by $\mathbf{1}_{n}$,
or simply $\mathbf{1}$, and the null matrix by $\hat{0}$.

We say that a matrix $A \in M_{n}$ is positive-semidefinite,
or simply \emph{positive},
if the inner product $\langle \eta, A \eta \rangle \geq 0$,
for any vector $\eta \in \mathbb{C}^{n}$
(i.e. $A = A^{*}$ and $A$ has a non-negative spectrum).
A linear map $S\!: M_{n} \rightarrow M_{n}$ is said to be positive,
indicated: $S \geq 0$,
if for any $A \in M_{n}$ such that $A \geq 0$,
we have $S(A) \geq 0$. 
For a positive map $S$,
its operator norm is given by $||S|| = S(\mathbf{1})$.
Any positive map is Hermitian,
i.e.  $S(A^{*}) = S(A)^{*}$, for all $A \in M_{n}$.
The identity map of $M_{n}$ is labelled $I_{n}$,
or simply $I$.
The convex cone of all positive maps of $M_{n}$ is denoted
by $\mathcal{P}(M_{n})$.
A positive map $S$ is extremal,
if for any positive map $T: M_{n} \rightarrow M_{n}$ such that
$S - T \in \mathcal{P}(M_{n})$,
i.e. $0 \leq T \leq S$,
we have $T = \alpha S$ for some number $0 \leq \alpha \leq 1$.
It is true that every positive map
can be written as a convex combination of extremal ones.
If  $S\!: M_{n} \rightarrow M_{n}$ is a positive map such that
$S(\mathbf{1}) = \mathbf{1}$ and
$\text{tr} S(A) = \text{tr} A$,
for any $A \in M_{n}$,
then we call it \emph{bistochastic},
or \emph{doubly stochastic}.

From now on,
let us fix $n = 3$.
We choose the set of normalised Gell-Mann matrices,
$\left \{ \lambda_{\mu} \right \}_{\mu=0}^{8}$, by taking:
\begin{equation}
\nonumber
    \lambda_{1} = \frac{1}{\sqrt{2}} \begin{pmatrix}
            0 & 1 & 0 \\
            1 & 0 & 0 \\
            0 & 0 & 0
        \end{pmatrix}, \quad
    \lambda_{2} = \frac{1}{\sqrt{2}} \begin{pmatrix}
            0 & -i & 0 \\
            i & 0 & 0 \\
            0 & 0 & 0
        \end{pmatrix}, \quad
    \lambda_{3} = \frac{1}{\sqrt{2}} \begin{pmatrix}
            1 & 0 & 0 \\
            0 & -1 & 0 \\
            0 & 0 & 0
        \end{pmatrix},
\end{equation}
\begin{equation}
\nonumber
    \lambda_{4} = \frac{1}{\sqrt{2}} \begin{pmatrix}
            0 & 0 & 1 \\
            0 & 0 & 0 \\
            1 & 0 & 0
        \end{pmatrix}, \quad
    \lambda_{5} = \frac{1}{\sqrt{2}} \begin{pmatrix}
            0 & 0 & -i \\
            0 & 0 & 0 \\
            i & 0 & 0
        \end{pmatrix}, \quad
    \lambda_{6} = \frac{1}{\sqrt{2}} \begin{pmatrix}
            0 & 0 & 0 \\
            0 & 0 & 1 \\
            0 & 1 & 0
        \end{pmatrix},
\end{equation}
\begin{equation}
\nonumber
    \lambda_{7} = \frac{1}{\sqrt{2}} \begin{pmatrix}
            0 & 0 & 0 \\
            0 & 0 & -i \\
            0 & i & 0
        \end{pmatrix}, \quad
    \lambda_{8} = \frac{1}{\sqrt{6}} \begin{pmatrix}
            1 & 0 & 0 \\
            0 & 1 & 0 \\
            0 & 0 & -2
        \end{pmatrix}
\end{equation}
and $\lambda_{0} = \frac{1}{\sqrt{3}} \mathbf{1}_{3}$;
which is an orthonormal basis for $M_{3}$ with respect to the HS inner product.
Any self-adjoint matrix $A = A^{*} \in M_{3}$ can be written as
$A = \lambda(a)$, where
$\lambda(a) = \sum_{\mu=0}^{8} a_{\mu} \lambda_{mu}$,
and 
$a = (a_{0}, a_{1}, \ldots, a_{8}) \in \mathbb{R}^{9}$.
Sometimes, we will use a simplified notation:
$a = (a_{0}, \vec{a})$,
where $\vec{a} = (a_{1},\ldots,a_{8}) \in \mathbb{R}^{8}$,
and $\lambda(a) = \lambda(a_{0}, \vec{a}) = a_{0} \lambda_{0} + \vec{a} \cdot \vec{\lambda}$.
For a bistochastic map $S$,
let us define a matrix $x \in M_{8}(\mathbb{R})$ by
$x_{ij} = \text{tr} \lambda_{i} S(\lambda_{j})$,
$i,j = 1,2,\ldots,8$.
Then it is easy to see that $S$ acts on a self-adjoint matrix by
\begin{equation}
\label{eq:isomoprhism}
S(\lambda(a_{0}, \vec{a})) = \lambda(a_{0}, x \vec{a}).
\end{equation}
For a matrix $x \in M_{8}(\mathbb{R})$,
we denote a linear map on $M_{3}$ that preserves both the identity and trace,
defined by the relation \eqref{eq:isomoprhism},
by $S_{x}$.
Let $\Lambda \subset \mathbb{R}^{8}$ be a set of those real matrices $x$,
for which $S_{x}$ is a bistochastic map:
$\Lambda = \left \{ x \in M_{8}(\mathbb{R}) \: |  \: S_{x} \geq 0 \right \}$.
Therefore, there is a one-to-one correspondence between $\Lambda$
and the set of bistochastic maps on $M_{3}$.
The structure of the set $\Lambda$ is fairly complicated. 
First, it is a closed convex set, i.e. for any $x,y \in \Lambda$,
$\lambda x + (1-\lambda)y \in \Lambda$,
for $0 \leq \lambda \leq 1$.
Moreover, is is a compact affine topological semigroup
(see e.g. \cite{schwarz1955hausdorff,chow1975compact} for the definitions)
with involution being the matrix transposition: $A \mapsto A^{t}$.
Obviously,
$S_{\mathbf{1}_{8}} = I$.
The structure of the set analogous to $\Lambda$,
but representing maps on the algebra $M_{2}$,
has been studied using geometrical methods in \cite{miller2015topology}.
Because there is no such geometrical identification for $n = 3$,
this time we exploit the semigroup aspect of the set $\Lambda$.

\begin{proposition}
\label{prop:LambdaAndBalls}
    Let $\overline{K}_{r}(\hat{0})$ denote a closed
    ball in $M_{8}(\mathbb{R})$ with respect to the operator norm,
    centred around $\hat{0}$, with radius $r > 0$.
    Then
    \begin{equation}
        \overline{K}_{\frac{1}{2}}(\hat{0}) \subset \Lambda \subset \overline{K}_{1}(\hat{0}).
    \end{equation}
\end{proposition}
\begin{proof}
    Suppose that $x \in M_{8}(\mathbb{R})$ and $||x|| \leq \frac{1}{2}$.
    Let $\vec{m}, \vec{n} \in \mathbb{R}^{8}$ be such that
    $P_{\vec{m}} = \lambda(\frac{1}{\sqrt{3}}, \vec{n})$ and
    $P_{\vec{n}} = \lambda(\frac{1}{\sqrt{3}}, \vec{m})$
    are orthogonal projections in $M_{3}$.
    It is easy to check that in that case
    $|| \vec{m} ||_{2} = || \vec{n} ||_{2} = \sqrt{\frac{2}{3}}$,
    where $|| \cdot||_{2}$ denotes
    the standard Euclidean norm.
    Because $||x|| \leq \frac{1}{2}$, we have that
    \begin{equation}
        \text{tr} \, P_{\vec{m}} S_{x}(P_{\vec{n}}) = 
        \frac{1}{3} + \langle \vec{m}, x \vec{n} \rangle \geq
        \frac{1}{3} - |\langle \vec{m}, x \vec{n} \rangle| \geq 0.
        %\frac{1}{3} - \frac{1}{2} \frac{2}{3} = 0.       
    \end{equation}
    Hence $S_{x}$ is positive, which means that $x \in \Lambda$.
    
    Let now $x \in \Lambda$.
    Then for any $A = A^{*} \in M_{3}$, $S_{x}$ fullfils the
    \emph{Kadison-Schwarz inequality} 
    \cite{choi1980some}:
    \begin{equation}
     \label{eq:KSineq}
        S(A)^{2} \leq S(A^{2}).
    \end{equation}
    Let $\vec{n} \in \mathbb{R}^{8}$; since $\lambda(\frac{1}{\sqrt{3}}, \vec{n})$
    is self-adjoint, by \eqref{eq:KSineq}, we have that
    \begin{equation}
        \frac{1}{\sqrt{3}} + || x \vec{n} ||_{2}^{2} = \text{tr}\, 
        \left ( S_{x}(\lambda(\frac{1}{\sqrt{3}}, \vec{n}) \right)^{2} \leq
        \text{tr}\, \lambda(\frac{1}{\sqrt{3}}, \vec{n})^{2} =
        \frac{1}{\sqrt{3}} + ||\vec{n}||_{2}^{2},
    \end{equation}
i.e. $||x\vec{n}||_{2} \leq ||\vec{n}||_{2}$ for any $\vec{n} \in \mathbb{R}^{8}$,
which means that $x \in \overline{K}_{1}(\hat{0})$.
\end{proof}

Let $P_{8} \in M_{8}(\mathbb{R})$ denote a diagonal matrix
$P_{8} = \text{diag}(0,0,0,0,0,0,0,1)$;
and so on for other sets of indices:
$P_{38} = \text{diag}(0,0,1,0,0,0,0,1)$, etc.
It is true that all matrices
$P_{8}, P_{38}, P_{138}, P_{1238}, P_{13468}$ are idempotent elements 
belonging to $\Lambda$.
In addition, there are no idempotent elements in $\Lambda$ of rank 6 or 7,
and the only idempotent of rank 8 is the identity $\mathbf{1}_{8}$.
Let also $G_{3} = \text{Ad} \, \text{SU}(3) \subset \Lambda$ denotes
the group of those matrices $g \in \Lambda$ such that
$S_{g}$ is an automorphism: $S_{g}(A) = U A U^{*}$,
for a unitary matrix $U \in \text{SU}(3)$ and for any $A \in M_{3}$.
It is evident that $G_{3} \subset \text{SO}(8)$,
the special group of orthogonal matrices.
The set of those $x$ such that $S_{x}$ is an extremal map in the set
of all positive maps on $M_{3}$ (not necessarily bistochastic) is denoted by
$\text{Ext}_{0}(\Lambda)$,
whereas the set of extremal points of the convex set $\Lambda$ is labelled as
$\text{Ext}(\Lambda)$.
The matrix $x \in \text{Ext}(\Lambda)$,
if and only if for all 
$y_{1}, y_{2} \in \Lambda$ and $0 \leq \lambda \leq 1$,
if $x = \lambda y_{1} + (1-\lambda) y_{2}$,
then $y_{1} = y_{2} = x$.
It is true that $\text{Ext}_{0}(\Lambda) \subset \text{Ext}(\Lambda)$.
The next fact follows from Proposition \ref{prop:LambdaAndBalls}.

\begin{proposition}
    \label{prop:oneHalfofOrthogonal}
    Let $x \in \mathrm{Ext}(\Lambda)$.
    Suppose that $||x|| = \frac{1}{2}$;
    then $ x = \frac{1}{2} R$,  for $R \in \mathrm{O}(8)$.
\end{proposition}
\begin{proof}
    Let $x \in \mathrm{Ext}(\Lambda)$ and $||x|| = \frac{1}{2}$.
    Let $x = R |x|$ be the polar decomposition of $x$,
    $R \in \text{O}(8)$, $|x| = \sqrt{x^{t} x}$.
    Supose that $|x| \neq \frac{1}{2} \mathbf{1}_{8}$.
    Let $y_{1} = \frac{1}{2} \mathbf{1}_{8}$ and
    $y_{2} = 2 |x| - \frac{1}{2} \mathbf{1}_{8}$.
    Then $- \frac{1}{2} \mathbf{1}_{8} \leq y_{2} \leq \frac{1}{2} \mathbf{1}_{8}$,
    so $|| y_{2} || \leq \frac{1}{2}$
    and by Proposition \ref{prop:LambdaAndBalls},
    we have that both $R y_{1}, R y_{2} \in \Lambda$.
    Then $x = \frac{1}{2} R y_{1} + \frac{1}{2} R y_{2}$,
    and thus it cannot be extremal, a contradiction.
    Hence, $x = \frac{1}{2} R$.
\end{proof}


\begin{remark}
    By Proposition \ref{prop:LambdaAndBalls},
    we see that $x \in \text{Ext}(\Lambda)$ implies that
    $||x|| \geq \frac{1}{2}$.
    Therefore, $x = \frac{1}{2} R$, $R \in \mathrm{O}(8)$,
    are the only possible elements in $\text{Ext}(\Lambda)$,
    and so in $\text{Ext}_{0}(\Lambda)$,
    with the norm $||x|| = \frac{1}{2}$.
\end{remark}

%%%%%%%%%%%%%%%%%%%%
%%% Main results %%%
%%%%%%%%%%%%%%%%%%%%

\section{Idempotent and extremal elements of $\Lambda$}
\label{sec:main}

Let $x \in \Lambda$ and let 
$\langle x \rangle \subset \Lambda$ be the semigroup generated by $x$:
$\langle x \rangle = \left \{ x^{k} \: | \: k \in \mathbb{N}, k \geq 1 \right \}$.
By $\overline{\langle x \rangle}$,
we denote the closure of $\langle x \rangle$ in $M_{8}(\mathbb{R})$.
The proof of the following proposition is presented in
    \cite[Lemma 3]{schwarz1955hausdorff}.

\begin{proposition}
    \label{prop:UniqeClusterPoint}
    Since $\Lambda$ is closed, $\overline{\langle x \rangle} \subset \Lambda$.
    The set $\overline{\langle x \rangle}$ contains a unique idempotent,
        denoted by $e_{x}$.
\end{proposition}

\begin{definition}
    For the set $\Lambda$, we define the following subsets:
    \begin{enumerate}
       \item the set of idempotents of $\Lambda$:
            $\mathcal{E}(\Lambda) = \left \{ e \in \Lambda \: |  \: e^{2} = e \right \}$;
       \item the set of nilpotent elements:
            $\mathcal{N}(\Lambda) = \left \{ x \in \Lambda \: |  \: \lim \limits_{n \rightarrow \infty} x^{n} = \hat{0} \right \}$;     
       \item the group of invertible elements:
            $\mathcal{G}(\Lambda) = \left \{ x \in \Lambda \: |  \: \exists \, y \in \Lambda, xy = yx = \mathbf{1}_{8} \right \}$.       
    \end{enumerate}
   For an idempotent element $e \in \mathcal{E}(\Lambda)$,
   we define the following subsets of $\Lambda$:
   \begin{enumerate}
        \item let $H(e)$ be the maximal subgroup of $\Lambda$ 
            containing $e \in \mathcal{E}(\Lambda)$;
        \item $Q(e) = \left \{ x \in \Lambda \: | \: e_{x} = e \right \}$,
        where $e_{x}$ is the unique idempotent element of $\overline{\langle x \rangle}$.
   \end{enumerate}
\end{definition}

\begin{remark}
\label{rem:GOfLambda}
It is a known fact from the semigroup theory that for each idempotent
element of a semigroup, there is exactly one maximal subgroup containing it.
In particular, in our case of the semigroup $\Lambda$,
$H(\hat{0}) = \left \{ \hat{0} \right \}$,
$H(\mathbf{1}_{8}) = \mathcal{G}(\Lambda)$,
obviously,
and moreover $H(P_{8}) = \left \{ P_{8} \right \}$.
Indeed,
if $x = (x_{ij})_{i,j=1}^{8} \in H(P_{8})$,
then
$x = x P_{8} = P_{8} x P_{8} = x_{88} P_{8}$.
Because there is a sequence of natural numbers $n_{k}$, $k=1,2,3,\ldots$,
such that $x^{n_{k}} \overset{k}{\rightarrow} P_{8}$, so
$|x_{88}| = 1$. But $- P_{8} \notin \Lambda$, and thus $x = P_{8}$.
\end{remark}

\begin{remark}
    \label{rem:Qof1}
It is true that $Q(\hat{0}) = \mathcal{N}(\Lambda)$.
Indeed, let $x \in Q(\hat{0})$,
then by definition, there is a sequence of natural numbers $n_{k}$ such that
$x^{n_{k}} \overset{k}{\rightarrow} \hat{0}$.
For any $l \in \mathbb{N}$, there is $k(l)$ such that
$l \leq n_{k(l)}$, and since 
$||x|| \leq 1 $, by Proposition \ref{prop:LambdaAndBalls}, one has
$||y^{l}|| \leq ||y^{l - n_{k(l)}}|| \, ||y^{n_{k(l)}}|| \leq ||y^{n_{k(l)}}||
\overset{l}{\rightarrow} 0$.
Hence $y^{l} \overset{l}{\rightarrow} \hat{0}$
and $Q(\hat{0}) \subset \mathcal{N}(\Lambda)$.
The reverse inclusion is obvious.
Moreover, since we have that $H(\mathbf{1}_{8}) = \mathcal{G}(\Lambda)$,
    by \cite[Theorem 8]{schwarz1955hausdorff},
    $H(\mathbf{1}_{8}) = Q(\mathbf{1}_{8}) \cdot \mathbf{1}_{8}$,
    and hence $Q(\mathbf{1}_{8}) = \mathcal{G}(\Lambda)$.
\end{remark}

Our task is to analyse the family of sets
$\left \{ Q(e), e \in \mathcal{E}(\Lambda) \right \}$,
and in particular,
to show, where among this family the elements of
$\text{Ext}_{0}(\Lambda)$ are to be found.
To this end, we present the following series of results.    

\begin{proposition}
    For $e_{1}, e_{2} \in \mathcal{E}(\Lambda)$,
    $e_{1} \neq e_{2}$,
    the sets $Q(e_{1})$ and $Q(e_{2})$ are disjoint.
    Moreover,
    $\Lambda = \! \bigcup \limits_{e \in \mathcal{E}(\Lambda)} \! Q(e)$.
\end{proposition}
\begin{proof}
    If $e_{1} \neq e_{2}$, by Proposition \ref{prop:UniqeClusterPoint},
    the sets $Q(e_{1})$ and $Q(e_{2})$ must be disjoint.
    For $x \in \Lambda$, we have that $x \in Q(e_{x}) \subset \Lambda$,
    and the assertion follows.
\end{proof}

\begin{lemma}
    \label{lem:equivClassesOfQe}
    Let $e \in \mathcal{E}(\Lambda)$ and $g \in \mathcal{G}(\Lambda)$.
    Then $Q(g e g^{t}) = g Q(e) g^{t} = 
    \left \{ g x g^{t} \: | \: x \in Q(e) \right \}$.
\end{lemma}
\begin{proof}
    It is clear that $g \langle x \rangle g^{t} = \langle g x g^{t} \rangle$.
    Because the mapping $x \mapsto g x g^{t}$ is a homeomorphism,
    so $g \overline{\langle x \rangle} g^{t} = \overline{\langle g x g^{t} \rangle}$.
    Let $x \in Q(g e g^{t})$.
    Then $g e g^{t} \in \overline{\langle x \rangle}$.
    Hence $e \in g^{t} \overline{\langle x \rangle} g = \overline{\langle g^{t} x g \rangle}$.
    This in turn means that $g^{t} x g \in Q(e)$, i.e.
    $x \in g Q(e) g^{t}$, which establishes that
    $Q(g e g^{t}) \subset g Q(e) g^{t}$.
    The reverse inclusion follows from an analogous reasoning.
\end{proof}

Next, we describe the structure of the set of idempotents
in $\Lambda$.
For $e \in \mathcal{E}(\Lambda)$,
let $\mathcal{E}_{G_{3}}(e) = \left \{ g e g^{t} \: | \: g \in G_{3} \right \}$.
At first, in the following lemma, we recall a known fact
that an idempotent contractive operator on a Hilbert space 
is an orthogonal projection 
(see e.g. \cite[Problem 5.3.14]{abramovich2002problems}).
\begin{lemma}
    \label{lem:eIsProj}
    Let $e \in \mathcal{E}(\Lambda)$, then $e^{t} = e$, and hence,
    $e$ is an orthogonal projection in $M_{8}(\mathbb{R})$.
\end{lemma}
\begin{proof}
    Because $e^{2} = e$, so $||e|| = ||e^{2}|| \leq ||e||^{2}$,
    hence $||e|| \geq 1$.
    Since $e \in \Lambda$, by Proposition \ref{prop:LambdaAndBalls}, $||e|| =1$.
    Suppose that $\vec{n} \in \ker e$ and $\vec{m} \in \mathcal{R}(e)$,
    the range of the operator $e$.
    Let $\alpha \in \mathbb{R}$.
    Then 
\begin{equation}
    || \alpha \vec{m} ||^{2} = || P (\vec{n} + \alpha \vec{m}) ||^{2} \leq
    || \vec{n} + \alpha \vec{m}||^{2} \leq ||\vec{n}||^{2} + 
        2 \alpha \langle \vec{n}, \vec{m} \rangle + || \alpha \vec{m}||^{2},
\end{equation}
i.e. $||\vec{n}||^{2} + 2 \alpha  \langle \vec{n}, \vec{m} \rangle \geq 0$,
for any $\alpha \in \mathbb{R}$.
It means that 
$\langle \vec{n}, \vec{m} \rangle = 0$,
for every  $\vec{n} \in \ker e$ and $\vec{m} \in \mathcal{R}(e)$,
i.e.
$\ker e \perp \mathcal{R}(e)$, which proves that $e = e^{t}$.
\end{proof}

\begin{proposition}
\label{prop:htranspose}
Let $e \in \mathcal{E}(\Lambda)$ and $h \in H(e)$.
Then $h^{t} \in H(e)$, and $h^{t} h = h h^{t} = e$.
\end{proposition}
\begin{proof}
    By \cite[Theorem 8]{schwarz1955hausdorff},
    $H(e) = Q(e) e = e \, Q(e)$.
    If $x \in Q(e)$, then $x^{t} \in Q(e)$,
    by Lemma \ref{lem:eIsProj}.
    Hence, $h^{t} \in H(e)$.
    Since $e$ is a projection, $e \leq \mathbf{1}_{8}$;
    moreover $h^{t} h \leq \mathbf{1}_{8}$, because $h \in \Lambda$.
    Thus, $\hat{0} \leq (h^{t} h)^{k} \leq h^{t} h = e h^{t} h e \leq e$,
    for any $k \in \mathbb{N}$.
    If $n_{k}$ is a sequence of natural numbers such that
    $(h^{t} h)^{n_{k}} \overset{k}{\rightarrow} e$,
    we obtain that $h^{t} h = e$.
    By the same reasoning, we have also: $h h^{t} = e$.
\end{proof}

Let us recall that for a bistochastic map $S$, by 
$K_{S}$ we denote the stable subspace of $S$ defnied by
\begin{equation}
    K_{S} = \left \{ x \in M_{3} \:|\:
            \forall k \in \mathbb{N} \,\,
            || S^{k} x ||_{HS} = || S^{*k} x ||_{HS} =  ||x||_{HS}
    \right \},
\end{equation}
where $S^{*}$ is the adjoint map given by
$\text{tr} S^{*}(A) B = \text{tr} A S(B)$ for all $A, B \in M_{3}$
(see \cite[Eq. (3.3)]{miller2015stable}).
The fact that $H(\mathbf{1}_{8}) = \mathcal{G}(\Lambda)$
can be generalised to the following result.
\begin{theorem}
    Let $e \in \mathcal{E}(\Lambda)$ and
    $K_{S_{e}}$ be the stable subspace of the map $S_{e}$.
    Then $K_{S_{e}}$ is a Jordan algebra and
    $H(e) \cong \text{Aut}_{J} \, K_{S_{e}}$,
    the group of Jordan automorphisms of $K_{S_{e}}$.
\end{theorem}
\begin{proof}
    Because $e$ is idempotent, by
    \cite[Corollary 3]{miller2015stable},
    the space $K_{S_{e}} = S_{e}(M_{3})$ is a Jordan algebra.
    The map $S_{e}$ is in fact the conditional expectation onto $K_{S_{e}}$.
    Let $h \in H(e)$.
    By Proposition \ref{prop:htranspose}, for any $k\in \mathbb{N}$,
    $S_{h}^{* k} S_{h}^{k} = S_{h^{t}}^{k} S_{h}^{k} =  S_{(h^{t})^{k} h^{k}} = S_{e}$,
    and the same for $S_{h}^{k} S_{h}^{* k}= S_{e}$.
    Thus, the stable algebra $K_{S_{h}} = K_{S_{e}}$
    (compare \cite{miller2015stable}, eq. (3.3) and below).   
    Again, by \cite[Corollary 3]{miller2015stable},
    $\varphi_{h} = S_{h} \big |_{K_{S_{e}}}$
    is a Jordan automorphism of the Jordan algebra $K_{S_{e}}$.

    On the other hand, if
    $\varphi$ is an arbitrary Jordan automorphism of $K_{S_{e}} \subset M_{3}$,
    then it could be extended to a bistochastic map $S_{h}$ on $M_{3}$,
    for some $h \in \Lambda$,
    by $S_{h} = \varphi \circ S_{e}$.
    Then
    $S_{h} S_{e} = S_{h} = S_{e} S_{h}$,
    because $S_{e}$ acts as the identity map on $K_{S_{e}}$.
    Hence, $he = eh = h$.
    In addition, since $\varphi$ is invertible on $K_{S_{e}}$,
    by extending $\varphi^{-1}$ to another bistochastic map
    $S_{h'}$,  $h' \in \Lambda$,
    we show that $h' \in H(e)$ and $h' h = h h' = e$,
    which proves that $h \in H(e)$.
\end{proof}

\begin{remark}
From the above theorem,
since $\mathcal{G}(\Lambda) = H(\mathbf{1}_{8})$,
 we infer that
the group $\mathcal{G}(\Lambda)$ consists of those matrices that represent
Jordan isomorphism on $M_{3}$,
i.e. for any $g \in \mathcal{G}(\Lambda)$,
there is a unitary matrix $U \in \text{SU}(3)$ such that
either $S_{g}(A) = U A U^{*}$ or $S_{g}(A) = U A^{t} U^{*}$
for all $A \in M_{3}$.
% It is evident that $G_{3} \subset \mathcal{G}(\Lambda)$.
% The elements of $G_{3}$ represent precisely those Jordan isomorphism,
% that are also ordinary isomorphisms;
% and for every $g \in \mathcal{G}(\Lambda) \backslash G_{3}$ there is
% $g' \in G_{3}$ such that $g = g' \tau$,
% where $\tau \in \mathcal{G}(\Lambda)$ is such that
% $S_{\tau}(A) = A^{t}$ for every $A \in M_{3}$,
% i.e.
% $\tau = P_{13468} - P_{257}$.
\end{remark}

\begin{theorem}
\label{thm:Idempotents}
The set $\mathcal{E}(\Lambda)$
    is a sum of seven disjoint subsets:
    \begin{equation}
        \mathcal{E}(\Lambda) = \bigcup \limits_{e_{0} \in J} \mathcal{E}_{G_{3}}(e_{0}),
    \end{equation}
    where $J = \left \{ \hat{0}, P_{8}, P_{38}, P_{138}, P_{1238}, P_{13468}, \mathbf{1}_{8} \right \}$.
\end{theorem}
\begin{proof}
    If $e \in \mathcal{E}(\Lambda)$, then as above, $S_{e}$ is a conditional
    expectation map onto the Jordan algebra $K_{S_{e}}$.
    By Theorems 5.3.8 and 6.2.3 of \cite{hanche1984jordan},
    all Jordan subalgebras of $M_{3}$ are isomorphic (unitary equivalent)
    to one of the following:
    $\mathbb{C}\mathbf{1}$, 
    $\mathbb{C} E_{12} \oplus \mathbb{C} E_{3}$,
    $\mathbb{C} E_{1} \oplus \mathbb{C} E_{2} \oplus \mathbb{C} E_{3}$,
    $M_{2}^{s} \oplus \mathbb{C} E_{3}$, 
    $M_{2} \oplus \mathbb{C} E_{3}$,
    $M_{3}^{s}$,
    and
    $M_{3}$ itself;
    where $M_{k}^{s}$ is the Jordan algebra of symmetric matrices of 
    size $k$: $M_{k}^{s} = \{ A \in M_{k}: A = A^{t} \}$;
    $E_{i}, i = 1,2,3$, are matrix units with 1 at \emph{i}th diagonal entry
    and 0 elsewhere, and $E_{12} = E_{1} + E_{2}$.
    Hence, there is $g \in G_{3}$ such that
    $e = g e_{0} g ^{t}$, and $e_{0}$ is the orthogonal projection
    that represents the projection map onto precisely one of the algebras
    mentioned above.
    It is straightforward to check that then $e_{0} \in J$,
    and $\text{dim} \, e_{0} + 1$ is equal to the dimension of the respective
    Jordan algebra associated to it.
    Hence, $e \in \mathcal{E}_{G_{3}}(e_{0})$.
\end{proof}

\begin{corollary}
\label{cor:q}
    For $e \in \mathcal{E}(\Lambda)$, since $e$ is a projection,
    $\text{dim} \, e 
    \in \left \{ 0,1,2,3,4,5,8 \right \}$.
\end{corollary}

We prove a useful decomposition of elements of $Q(e)$ in the following lemma.

\begin{lemma}
\label{lem:decomposition}
    Let $e \in \mathcal{E}(\Lambda)$.
    A matrix $x$ belongs to $\in Q(e)$, if and only if,
    $x = h + y$, where
    $h \in H(e)$, $H(e) y = y H(e) = \hat{0}$.
    and $\lim_{k \rightarrow \infty} y^{k} = \hat{0}$. 
    This decomposition is unique.
\end{lemma}
\begin{proof}
    Suppose that $h + y \in \Lambda$, $h \in H(e)$, $y^{k}  \overset{k}{\rightarrow}  \hat{0}$,
    and $hy = yh = \hat{0}$.
    Because $h \in H(e)$, there is a sequence $n_{k} \in \mathbb{N}$ for $k =1,2,3,\ldots$,
    such that $h^{n_{k}} \overset{k}{\rightarrow} e$.
    Then $x^{n_{k}}  = h^{n_{k}} + y^{n_{k}}  \overset{k}{\rightarrow}  e$, i.e.
    $x \in Q(e)$.
    On the other hand,
    let us suppose that $x \in Q(e) \subset \Lambda$.
    Let also $h = e x e$, and $y = e^{\perp} x e^{\perp}$,
    where $e^{\perp} = \mathbf{1}_{8} - e$.
    Then $h \in Q(e)$, and since
    $h =  h e = e h$,
    $h \in H(e)$, by \cite[Theorem 8]{schwarz1955hausdorff}.
    For $h' \in H(e)$, we have
    $h' y = h' e y = \hat{0} = y e h' = yh'$,
    i.e. $H(e) y = y H (e) = \hat{0}$.
    There is a sequence $n_{k}$ of natural numbers such that
    $x^{n_{k}}  \overset{k}{\rightarrow}  e$.
    Hence
    $h^{n_{k}} = e h^{n_{k}}  = e x^{n_{k}}  \overset{k}{\rightarrow} e^{2}  = e$.
    It follows that $y^{n_{k}}  \overset{k}{\rightarrow}  \hat{0}$,
    which is enough to say that $y^{k} \overset{k}{\rightarrow} \hat{0}$.
    Lastly,
    suppose that $x = h + y = h_{2} + y_{2}$,
    where $h_{2} \in H(e)$ and $y_{2}$,
    $h_{2} y_{2} = y_{2} h_{2} = \hat{0}$.
    For some sequence of natural numbers $m_{k}$, we have
    $e y_{2} = \lim_{k} h_{2}^{m_{k}} y_{2} = \hat{0}$.
    Then $h = e x = e h_{2} + e y_{2} = h_{2}$, and also $y = y_{2}$.
\end{proof}

For $x \in Q(e)$, since the decomposition described above is unique,
let us denote by $h_{x}$ and $y_{x}$ the matrices such that
$x = h_{x} + y_{x}$,
$h_{x} \in H(e)$, $H(e) y_{x} = y_{x} H(e)$, and
$y_{x}^{k} \overset{k}{\rightarrow} \hat{0}$.
The above lemma justifies the following definition.

\begin{definition}
    Let $e \in \mathcal{E}(\Lambda)$.
    We define:
    $Q_{0}(e) = \left \{ x \in Q(e) \: | \: || y_{x} || < 1 \right \}$, and
    $Q_{1}(e) = \left \{ x \in Q(e) \: | \: || y_{x} || = 1 \right \}$.
    For $i = 1, 2, 3, \ldots 8$,
    let $Q_{1}^{i}(e)$ be a set consisting of those $x = h_{x} + y_{x} \in Q_{1}(e)$,
    for which the multiplicity of the highest singular value of $y_{x}$
    is equal to $i$.
\end{definition}

It is obvious that
$Q(e) = \bigcup \left \{ Q_{1}^{i}(e), i=1,2,\ldots,8 \right \} \cup Q_{0}(e)$,
and these sets are disjoint (possibly empty).

\begin{proposition}
Let $e \in \mathcal{E}(\Lambda)$.
If $\mathrm{dim} \, e \in \left \{ 5,8 \right \}$,
then
$Q(e) = Q_{0}(e)$.
If $\mathrm{dim} \, e \leq 4$,
then
$Q_{1}^{i}(e) = \emptyset$ for $i \geq 5 - \mathrm{dim} \, e$.
\end{proposition}
\begin{proof}
    If $\mathrm{dim} \, e = 8$,
    then $e = \mathbf{1}_{8}$, and since by Remark \ref{rem:Qof1},
    $Q(\mathbf{1}_{8}) = H(\mathbf{1}_{8}) = \mathcal{G}(\Lambda)$,
    obviously $Q(\mathbf{1}_{8}) = Q_{0}(\mathbf{1}_{8})$.
    Suppose that $e \neq \mathbf{1}_{8}$
    and $x = h_{x} + y_{x} \in Q(e)$.
    Then $x^{t} x = h_{x}^{t} h_{x} + y_{x}^{t} y_{x} = e + y_{x}^{t} y_{x}$,
    by Proposition \ref{prop:htranspose};
    $y_{x} e = e y_{x} = \hat{0}$, 
    and because for any $k \in \mathbb{N}$, $(x^{t} x)^{k} \in \Lambda$,
    we have that $e + p = \lim_{k} x^{t} x \in \Lambda$,
    where $p$ is a orthogonal projection onto the space spanned by
    eigenvectors of $y_{x}^{t} y_{x}$ with eigenvalue 1.
    Of course, $e p = p e = \hat{0}$, so
    the matrix $e + p \in \mathcal{E}(\Lambda)$.
    It must be that $e + p \neq \mathbf{1}_{8}$,
    otherwise $x^{t} x = \mathbf{1}_{8}$, and hence
    $x \in \mathcal{G}(\Lambda) = Q(\mathbf{1}_{8})$, a contradiction.
    If $\mathrm{dim}\,e = 5$, then $p = \hat{0}$,
    because by Theorem \ref{thm:Idempotents},
    there are no idempotent elements of $\Lambda$
    with rank $6$ or $7$.
    Hence $||y|| < 1$ and $x \in Q_{0}(e)$.
    By the same argument, for $\textrm{dim}\, e \leq 4$,
    it is impossible that $\mathrm{dim} \, p  + \mathrm{dim}\, e \geq 5$,
    so $Q_{1}^{i}(e) = \emptyset$, for $i \geq 5 - \mathrm{dim} \, e$,
    because by definition, $i = \textrm{dim}\,p$.        
\end{proof}

For the sake of convenience, let us introduce the following sequence
of elements of $\mathcal{E}(\Lambda)$:
$p_{0} = \hat{0}, p_{1} = P_{8}$, $p_{2} = P_{38}$,
$p_{3} = P_{138}$, $p_{4} = P_{1238}$, $p_{5} = P_{13468}$.

\begin{theorem}
\label{thm:LowerIndices}
    Let $i,j$  be integers such that
    $1 \leq i \leq 4$, $0 \leq j \leq 4$ and $i+j \leq 5$.
    If $x \in Q_{1}^{i}(p_{j})$,
    then there exist $g_{1}, g_{2} \in G_{3}$ and $z \in Q_{0}(p_{i+j})$ 
    such that $x = g_{1} z g_{2}$.
\end{theorem}
\begin{proof}
    Let $x \in Q_{1}^{i}(p_{j})$ and
    $x = h_{x} + y_{x}$, as above.
    Since this decomposition is unique, we can write as in the proof of
    Lemma \ref{lem:decomposition}:
    $h_{x} = p_{j} x p_{j}$, and $y_{x} = p_{j}^{\perp} x p_{j}^{\perp}$,
    where $p_{j}^{\perp} = \mathbf{1}_{8} - p_{j}$.
    Because $p_{j} = p_{j}^{t}$, then
    $h y^{t} = y h^{t} = \hat{0}$, and thus
    $x x^{t}  = h h^{t} + y y^{t} = p_{j} + y y^{t}$,
    by \mbox{Proposition \ref{prop:htranspose}}.
    Let $R_{1} (p_{i} + y_{0}) R_{2}$ be the singular value decomposition of $y$,
    i.e. $R_{1}, R_{2}  \in \text{O}(8)$ are orthogonal matrices,
    and $y_{0}$ is diagonal with the only possible non-zero entries $s_{1}, s_{2}, \ldots, s_{8-i}$,
    such that $1 > s_{1} \geq s_{2} \geq \ldots \geq s_{8-i} \geq 0$,
    and $p_{i} y_{0} = y_{0} p_{i} = \hat{0}$.
    Then for $k \in \mathbb{N}$,
    $(x x^{t})^{k} = p_{j} + R_{1}( p_{i} + (y_{0} y_{0}^{t})^{k}) R_{1}^{t} \in \Lambda$,
    and  because $\Lambda$ is closed:
    $e_{1} = p_{j} + R_{1} p_{i} R_{1}^{t} = \lim_{k} (x x^{t})^{k} \in \Lambda$.
    Also, since $h y^{t} = y h^{t} = \hat{0}$,
    we have that $p_{j} R_{1} p_{i} R_{1}^{t} = R_{1} p_{i} R_{1}^{t} p_{j} = \hat{0}$.
    It follows that $e_{1}$ is an idempotent and $\text{rank} \, e_{1} = i+j$.
    By Theorem \ref{thm:Idempotents}, there is $g_{1} \in G_{3}$ such that
    $e_{1} = g_{1} p_{i+j} g_{1}^{t}$.
    A similar argument,
    applied this time to $x^{t} x$, shows that there is $g_{2} \in G_{3}$ 
    such that the idempotent $e_{2} = p_{j} + R_{2}^{t} p_{i} R_{2} \in \Lambda$
    could be written as $e_{2} = g_{2}^{t} p_{i+j} g_{2}$.
    Let $z = g_{1}^{t} x g_{2}^{t}$.
    What remains to show is that $z \in Q_{0}(p_{i+j})$.
    It is evident that $z \in \Lambda$.
    One can easily check that
    $p_{i+j} z = z p_{i+j}$,
    and hence $z = h_{z} + y_{z}$,
    where $h_{z} = p_{i+j} z p_{i+j}$,
    and $y_{z} = p_{i+j}^{\perp} z p_{i+j}^{\perp}$,
    $p_{i+j}^{\perp} = \mathbf{1}_{8} - p_{i+j}$.
    Because $h_{z} = p_{i+j} h_{z} = h_{z} p_{i+j}$,
    $h_{z} \in H(p_{i+j})$.
    Obviously, $h y_{z} = y_{z} h = \hat{0}$, for any $h \in H(p_{i+j})$.
    In addition, we have
    $y_{z} = g_{1}^{t}R_{1} y_{0} R_{2} g_{2}^{t}$,
    so $||y_{z}^{k}|| \leq ||y_{0}||^{k} = s_{1}^{k} \overset{k}{\rightarrow} \hat{0}$.
    Therefore, by Lemma \ref{lem:decomposition},
    $z \in Q(p_{i+j})$.
    Since $||y_{z}|| < 1$,
    $z \in Q_{0}(p_{i+j})$, which ends the proof.
\end{proof}

The main result of this paper could be captured in the following remark.
\begin{corollary}
\label{cor:q0}
    Suppose that $x \in \text{Ext}_{0}(\Lambda)$ and $S_{x}$ is not a Jordan isomorphism.
    Then there exist $g_{1}, g_{2} \in G_{3}$
    such that $g_{1}^{t} x g_{2}^{t} \in Q_{0}(\hat{0}) \cup Q_{0}(P_{8})$.
    In other words,
    either $||x|| < 1$,
    or
    $x = g_{1} (P_{8} + y) g_{2}$,  $y P_{8} = P_{8} y = \hat{0}$,
    and $|| y || < 1$.
\end{corollary}
\begin{proof}
Let $x \in  \text{Ext}_{0}(\Lambda)$.
    By Lemma \ref{lem:equivClassesOfQe} and Theorem \ref{thm:Idempotents}, there is $g \in G_{3}$ such that
    $g^{t} x g \in Q(e_{0})$, and $e_{0} \in J$.
    Because $S_{x}$ is an extremal positive map, then by
    \cite[Theorem 2]{miller2015stable},
    $e_{0} \in \left \{ \hat{0}, P_{8}, \mathbf{1}_{8} \right \}$.
    By Remark \ref{rem:Qof1}, $x \notin Q(\mathbf{1}_{8})$, hence
    $z = g^{t} x g \in Q(\hat{0}) \cup Q(P_{8})$,    
    and $z \in \text{Ext}_{0}(\Lambda)$
    (cf. \cite[Lemma 3.1.2]{stormer2013positive}).
    Suppose that $z \in Q(P_{8})$.
    By Theorem \ref{thm:LowerIndices},
    because $z \in \text{Ext}_{0}(\Lambda)$,
    $z \notin Q_{1}^{i}(P_{8})$
    for $i \geq 1$ (see  \cite[Theorem 2]{miller2015stable}).
    Hence $z \in Q_{0}(P_{8})$, and put $g_{1} = g$ and $g_{2} = g^{t}$.
    Then $x = g_{1} z g_{2}$,
    and by Lemma \ref{lem:decomposition}, because $H(P_{8}) = \left \{ P_{8} \right \}$
    (see Remark \ref{rem:GOfLambda}),
    $z = P_{8} + y$,  $y P_{8} = P_{8} y = \hat{0}$, and $||y|| < 1$.
    On the other hand, if $z \in Q(\hat{0})$,
    then by the same reasoning either $z \in Q_{0}(\hat{0})$,
    and $||z|| < 1$, because $H(\hat{0}) = \left \{ \hat{0} \right \}$, or
    $z \in Q_{1}^{1}(\hat{0})$ and then there are $g_{01}, g_{02} \in G_{3}$ such that
    $g_{01} z g_{02} \in Q_{0}(P_{8})$.
    Then put $g_{1} = g g_{01}^{t}$ and $g_{2} = g_{02}^{t} g^{t}$,
    and the assertion follows.
\end{proof}

We summarize the results presented above by saying that they allow
to narrow the task of finding positive extremal and bistochastic maps
on $M_{3}(\mathbb{C})$ to three specific groups;
and examples in all these groups have been found previously.
First, there are Jordan isomorphisms, represented by matrices from
$Q(\mathbf{1}_{8}) = \mathcal{G}(\Lambda)$.
Second, there are maps that could be called \emph{strongly ergodic}
\cite{miller2015stable},
belonging to the class represented by
$\left \{ x \in Q_{0}(\hat{0}) \:|\: ||x|| = \frac{1}{2} \right \}$,
of which the celebrated Choi map
\cite{choi1977extremal} is an example.
For a generalised Choi map $\Phi[a,b,c]$:
\begin{equation}\label{eq:choi}
\Phi[a,b,c](X)=\\ 
\frac{1}{2}
\begin{pmatrix}
ax_{11}+bx_{22}+cx_{33} & -x_{12} & -x_{13} \\
-x_{21} & cx_{11}+ax_{22}+bx_{33} & -x_{23} \\
-x_{31} & -x_{32} & bx_{11}+cx_{22}+ax_{33}
\end{pmatrix},
\end{equation}
where $X = (x_{ij})_{i,j = 1}^{3}$,
if we parametrise:
\begin{equation}
a(t)=\dfrac{(1-t)^2}{1-t+t^2},\quad b(t)=\dfrac{t^2}{1-t+t^2},\quad c(t)=\dfrac 1{1-t+t^2},
\end{equation}
$0 \leq t < 1$,
then $\Phi[a(t), b(t), c(t)]$ is an extremal bistochastic map on $M_{3}$.
For $t = 0$, it is the map proposed originally by Choi,
and for $t = 1$ it is a completely positive map.
See the paper by 
K.-C.\,Ha and S.-H.\,Kye \cite{ha2011entanglement} for more details.
The family of matrices $x_{t}$ such that
$S_{x_{t}} = \Phi[a(t), b(t), c(t)]$ is given by
\begin{equation}
    x_{t} = \begin{pmatrix}
        - \frac{1}{2} & 0 & 0 & 0 & 0 & 0 & 0 & 0 \\  
        0 & - \frac{1}{2} & 0 & 0 & 0 & 0 & 0 & 0 \\
        0 & 0 & \frac{1 - 4t + t^{2}}{4(1 - t + t^{2})} & 0 & 0 & 0 & 0 & - \frac{\sqrt{3}(1 - 4t + t^{2})}{4(1 - t + t^{2})} \\
        0 & 0 & 0 & - \frac{1}{2} & 0 & 0 & 0 & 0 \\
        0 & 0 & 0 & 0 & - \frac{1}{2} & 0 & 0 & 0 \\
        0 & 0 & 0 & 0 & 0 & - \frac{1}{2} & 0 & 0 \\  
        0 & 0 & 0 & 0 & 0 & 0 & - \frac{1}{2} & 0 \\  
        0 & 0 & \frac{\sqrt{3}(1 - 4t + t^{2})}{4(1 - t + t^{2})} & 0 & 0 & 0 & 0 & \frac{1 - 4t + t^{2}}{4(1 - t + t^{2})} \\
    \end{pmatrix}.
\end{equation}
It is worth noting that each
$x_{t} = \frac{1}{2} R_{t}$, where $R_{t} \in \text{O}(8)$,
as in Proposition \ref{prop:oneHalfofOrthogonal}.

Lastly, there are maps represented by elements of $Q_{0}(P_{8})$,
one example of which was proposed in \cite{miller2015stable}:
\begin{equation}
\label{eq:DefinitionOfS}
S_{0}(X) \:=\: \begin{pmatrix}
        \frac{1}{2}(x_{11} + x_{22}) & 0 & \frac{1}{\sqrt{2}} x_{13} \\
        0 & \frac{1}{2}(x_{11} + x_{22}) & \frac{1}{\sqrt{2}} x_{32} \\
        \frac{1}{\sqrt{2}} x_{31} & \frac{1}{\sqrt{2}} x_{23} & x_{33}
        \end{pmatrix}.
\end{equation}
The matrix $x \in \Lambda$ such that $S_{x} = S_{0}$ is given by
a diagonal matrix 
\begin{equation}
 x = 
\text{diag}(0,0,0, 
\frac{1}{\sqrt{2}}, \frac{1}{\sqrt{2}}, \frac{1}{\sqrt{2}}, \frac{-1}{\sqrt{2}}, 
1).
\end{equation}

We say that two elements $x, y \in \Lambda$
are equivalent if, and only if, there are $g_{1}, g_{2} \in G_{3}$
such that $x = g_{1} y g_{2}$.
Then the task of finding elements in $\text{Ext}_{0}(\Lambda)$,
up to this equivalence relation,
consists of:
\begin{enumerate}
\item finding all $R \in \mathrm{O}(8)$ such that
    $\frac{1}{2}R \in \text{Ext}_{0}(\Lambda)$;
\item determining which elements $x \in \Lambda$,
    such that $\frac{1}{2} < ||x|| < 1$, belong to $\text{Ext}_{0}(\Lambda)$;
\item finding all $y$, such that
    $y P_{8} = P_{8} y = \hat{0}$, $||y|| < 1$,
    and $P_{8} + y \in \text{Ext}_0(\Lambda)$.
\end{enumerate}

In the future,
we hope that the methods proposed,
together with some other, mostly geometric techniques,
will allow to bring the research closer to the final classification
of extremal positive maps on matrix algebras.

% \paragraph{Acknowledgements.}
% The authors would like to express their gratitude to...

%%%%%%%%%%%%%%%%%%%%%%%%
%%% End of the paper %%%
%%%%%%%%%%%%%%%%%%%%%%%%

% References
\bibliographystyle{abbrv}
\bibliography{./biblio}
% \nocite{*}

\end{document}
