\chapter{Zakończenie}
    \label{chap:Zakończenie}

Odwzorowania dodatnie stanowią ważny element teorii splątania złożonych
układów kwantowych.
Jak zostało pokazane, jedno z kluczowych dla teorii kryteriów splątania,
tzw. kryterium Peresa-Horodeckich,
polega na wskazaniu dla potencjalnego mieszanego stanu splątanego takiego
dodatniego odwzorowania pomiędzy algebrami reprezentującymi obserwable
poszczególnych podukładów, który ,,wykrywa’’ obecność splątania w tym stanie.
Pomimo że ogólna charakterystyka takich odwzorowań nie istnieje poza
najprostszym nietrywialnym przypadkiem odwzorowań dodatnich na algebrze
$M_{2}$ oraz pomiędzy algebrami $M_{2}$ i $M_{3}$, motywacja w postaci kryterium
splątania jest z pewnością wystarczająca,
aby kontynuować badania nad znalezieniem pełnego opisu odwzorowań dodatnich,
otwierającego drogę do dalszych zastosowań.
W niniejszej pracy przedstawiliśmy oryginalne próby scharakteryzowania
odwzorowań dodatnich w najprostszym nierozpoznanym dotąd przypadku odwzorowań
na algebrze $M_{3}$.
Poczynając od teoretycznego ugruntowania kryterium splątania dla złożonych
układów kwantowych,
słusznego nawet w najbardziej ogólnym przypadku układów
o nieskończonej liczbie stopni swobody,
przeszliśmy do wprowadzenia geometrycznych metod i ukazania ich przydatności do
analizy interesujących nas przekształceń,
poprzez podanie prostego dowodu znanego wcześniej rozkładu
odwzorowań dodatnich na $M_{2}$.
Dalej wskazaliśmy nowe przykłady dodatnich odwzorowań ekstremalnych na $M_{3}$,
między innymi dzięki wykorzystaniu wyników znanych
z teorii zwartych półgrup topologicznych.
Zebrana wiedza umożliwiła sformułowanie ogólnych ram programu dalszego
poszukiwania ekstremalnych bistochastycznych odwzorowań dodatnich
na algebrze $M_{3}$,
który bierze pod uwagę ich topologiczne własności.
Zrealizowanie takiego programu w przyszłości otworzyłoby drogę do lepszego
poznania własności zjawiska splątania układów kwantowych o skończonej ilości
stopni swobody,
a co za tym idzie -- do bardziej efektywnego wykorzystania tego zjawiska jako
zasobu odgrywającego kluczową rolę w kwantowych procesach przesyłu informacji.
