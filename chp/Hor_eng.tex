
\section{Horodeckis criterion in $C^*$-algebras}
In this section, we are going to present the result
analogous to the Theorem \ref{thm:main}, but this time in the C*-algebra setting.
It is expected that the previous assumption of injectivity
will have to be retained somehow. That leads to the choice of \emph{nuclear} C*-algebras,
i.e. algebras whose universal enveloping von Neumann algebras (the second dual spaces) are
injective. Not only it allows us to define a tensor product of two C*-algebras
uniquely, but also facilitates the usage of the previous result.
Nuclear C*-algebras, among others, share also the property of
local reflexivity, which will prove itself indispensable for
our line of reasoning.

Let $\mathcal{A}$ and $\mathcal{B}$ be C*-algebras.
By $\mathcal{A} \otimes_{max} \mathcal{B}$
and $\mathcal{A} \otimes_{min} \mathcal{B}$,
we denote respectively the \emph{maximal} and \emph{minimal}
C*-tensor product of $\mathcal{A}$ and $\mathcal{B}$.
It is known that if at least one of the algebras, say $\mathcal{A}$,
is nuclear, then those two tensor norms coincide and
the tensor product C*-algebra is in fact uniquely defined.
For a nuclear C*-algebra $\mathcal{A}$, we denote that tensor product
by $\mathcal{A} \bar{\otimes} \mathcal{B}$.
A C*-algebra $\mathcal{A}$ is by definition \emph{locally reflexive},
if for every finite-dimensional operator system $E \subset \mathcal{A}^{**}$,
i.e. a closed self-adjoint subspace containing the identity element,
there is a net $(T_{\lambda})$ of completely positive contractions,
$T_{\lambda}: E \rightarrow \mathcal{A}$,
which converges to the identity map on $E$ in the weak operator
topology. Every nuclear C*-algebra is locally reflexive
(see Brown and Ozawa\cite{Brown2008}, 9.3.1-9.3.3).
The algebras $\mathcal{A}$ and $\mathcal{B}$
can be treated as operator spaces in a natural way.
Thus, it is possible to introduce the operator space projective
tensor norm on the dual operator spaces $\mathcal{A}^{*}$
and $\mathcal{B}^{*}$, and the completion of the algebraic tensor product with
respect to that norm will be denoted by
$\mathcal{A}^{*} \hat{\otimes} \mathcal{B}^{*}$. Hence, we can introduce a cone of
separable states with respect to $\mathcal{A}$ and $\mathcal{B}$ in the same way as
in Section 2. Let  $\mathcal{S}(\mathcal{A})$ and $\mathcal{S}(\mathcal{B})$
denote the state spaces of $\mathcal{A}$ and $\mathcal{B}$, respectively.
We define a positive cone $C_{\mathcal{A},\mathcal{B}}$
of separable states with respect to $\mathcal{A}$ and $\mathcal{B}$ by
\begin{equation}
C_{\mathcal{A},\mathcal{B}} =
\overline{{\rm conv}}^{||\cdot||_{\wedge}}
\left \{\omega \otimes \varphi: \, \,\omega \in \mathcal{S}(\mathcal{A}),
\varphi \in \mathcal{S}(\mathcal{B})\right \}.
\end{equation}
It is clear that a state on $\mathcal{A} \bar{\otimes} \mathcal{B}$ is separable,
i.e. $\tilde{\phi} \in C_{\mathcal{A}, \mathcal{B}}$,
if and only if $\tilde{\phi}$ is a separable state
on the von Neumann algebra $\mathcal{A}^{**} \bar{\otimes} \mathcal{B}^{**}$.
\begin{theorem}
Let $\mathcal{A}$ and $\mathcal{B}$ be C*-algebras and suppose that
$\mathcal{A}$ is nuclear. A state $\tilde{\phi} \in
(\mathcal{A} \bar{\otimes} \mathcal{B})^{*}$ is separable
with respect to $\mathcal{A}$ and $\mathcal{B}$,
if and only if the functional $\tilde{\phi} \circ ( \mathbf{1} \otimes S )$
is positive for any positive and finite rank map $S: \mathcal{A} \rightarrow \mathcal{B}$.
\end{theorem}
{\bf Proof.} Let $\mathfrak{M}=\mathcal{A}^{**}$ and $\mathfrak{N}=\mathcal{B}^{**}$.
By the assumption, $\mathfrak{M}$ is injective. If $\tilde{\phi}$ is separable with respect to
$\mathcal{A}$ and $\mathcal{B}$, then it is also separable on
$\mathfrak{M} \bar{\otimes} \mathfrak{N}$. For any positive map
$S: \mathcal{A} \rightarrow \mathcal{B}$, the second dual map
$S^{**}:\mathfrak{M} \rightarrow \mathfrak{N}$ is positive and normal.
It follows from the first part of Theorem \ref{thm:main} that for any positive
$z \in \mathfrak{M} \bar{\otimes} \mathfrak{M}$,
$\langle (\mathbf{1} \otimes S^{**})(z), \tilde{\phi} \rangle \geq 0$.
Therefore, $\tilde{\phi} \circ (\mathbf{1} \otimes S)$ is a
positive functional on $\mathcal{A} \bar{\otimes} \mathcal{A}$.

Conversely, suppose that $\tilde{\phi} \circ (\mathbf{1} \otimes S) \geq 0$
for any positive and finite rank map $S: \mathcal{A} \rightarrow \mathcal{B}$.
We are going to show that the state $\tilde{\phi}$
is separable on $\mathfrak{M} \bar{\otimes} \mathfrak{N}$.
Suppose on the contrary that $\tilde{\phi}$ is not separable on
$\mathfrak{M} \bar{\otimes} \mathfrak{N}$.
By Theorem \ref{thm:main}, there is a positive normal and finite rank
map $\tilde{S}: \mathfrak{M} \rightarrow \mathfrak{N}$
and a positive element $z \in \mathfrak{M} \bar{\otimes} \mathfrak{M}$,
such that
\begin{equation}
\label{eq:Random6347}
\langle(\mathbf{1} \otimes \tilde{S})(z), \tilde{\phi}\rangle < 0.
\end{equation}
Moreover, we can choose the element $z$ from the algebraic tensor product
$\mathfrak{M} \otimes \mathfrak{M}$, i.e. $z = \sum_{j=1}^{N} x_{j} \otimes y_{j}$,
each $x_{j}$ and $y_{j} \in \mathfrak{M}$. Let $E \subset \mathfrak{M}$ be an
operator system generated by $\{ x_{j}, y_{j}\}_{j=1}^{N}$.
Since $\mathcal{A}$ is locally reflexive, there exists
a net $(T_{\lambda})$ of completely positive contractions
$T_{\lambda}: E \rightarrow \mathcal{A}$,
such that $T_{\lambda}$ converges to the identity map
on $E$ in the weak* operator topology, i.e.
for any $\omega \in \mathcal{A}^{*}$ and $x \in E$,
$ \langle \omega, T_{\lambda}x \rangle\stackrel{\lambda}{\rightarrow}
\langle x, \omega \rangle$. Let $a_{\lambda}$ denote the positive element
of $\mathcal{A} \otimes \mathcal{A}$,
$a_{\lambda} = (T_{\lambda} \otimes T_{\lambda}) (z) =
\sum_{j=1}^{N} T_{\lambda} x_{j} \otimes T_{\lambda} y_{j}$, and let
$i: \mathcal{A} \rightarrow \mathcal{A}^{**}=\mathfrak{M}$
be the canonical embedding. Then,
$i(a_{\lambda}) \stackrel{\lambda}{\rightarrow} z$ $\sigma$-weakly.
Since $\mathcal{B}$ is a Banach space, it has a local
reflexivity property in the following sense (see Ryan\cite{Ryan2002}, chapter 5.5).
For any finite-dimensional subspace $F \subset \mathcal{B}^{**}$,
there is an operator $P_{F}: F \rightarrow \mathcal{B}$
such that for any $y \in \mathcal{B}^{**} = \mathfrak{N}$
and $\varphi \in \mathcal{B}^{*}$, one has
\begin{equation}
\label{eq:Random24093}
\langle \varphi, P_{F}y \rangle =\langle y, \varphi \rangle.
\end{equation}
Let $F = \tilde{S}(\mathfrak{M})$.
It is obvious from equation \eqref{eq:Random24093} that $P_{F}$ is positive.
We define a positive map $S: \mathcal{A} \rightarrow \mathcal{B}$ by
$S = P_{F} \circ \tilde{S} \circ i$. Then,
for any $A \in \mathcal{A}$ and $\varphi \in \mathcal{B}^{*}$, we have
\begin{eqnarray}
\langle \varphi, SA \rangle =
\langle \varphi, P_{F} ( \tilde{S} \circ i (A)) \rangle =
\langle  \tilde{S} \circ i (A), \varphi \rangle \nonumber\\
=\langle i(A), \tilde{S}_{*} \varphi \rangle =
\langle \tilde{S}_{*} \varphi, A \rangle,
\end{eqnarray}
Suppose now that
$\tilde{\phi} = \lim \limits_{n} \sum_{i=1}^{n}\omega_{i} \otimes \varphi_{i}$,
where each $\omega_{i} \in \mathcal{A}^{*}$,
$\varphi_{i} \in \mathcal{B}^{*}$.
The limit is understood as the one in the projective
operator norm $|| \cdot ||_{\wedge}$, and hence also in the weak and weak* topology on
$(\mathcal{A} \bar{\otimes} \mathcal{B})^{*}$. Then, by the assumption,
$\langle \tilde{\phi},(\mathbf{1} \otimes S) (a_{\lambda}) \rangle \geq 0$. Moreover,
\begin{eqnarray}
\sum \limits_{i=1}^{n}\langle \omega_{i} \otimes \varphi_{i},
(\mathbf{1} \otimes S)(a_{\lambda}) \rangle =
\sum \limits_{i=1}^{n} \sum \limits_{j=1}^{N}
\langle \omega_{i} , T_{\lambda} x_{j} \rangle
\langle \tilde{S}_{*} \varphi_{i}, T_{\lambda} y_{j}\rangle \nonumber\\
\stackrel{\lambda}{\longrightarrow}\sum \limits_{i=1}^{n} \sum \limits_{j=1}^{N}
\langle  x_{j}, \omega_{i} \rangle\langle  y_{j},  \tilde{S}_{*} \varphi_{i} \rangle =
\langle (\mathbf{1} \otimes \tilde{S})(z),
\sum \limits_{i=1}^{n} \omega_{i} \otimes \varphi_{i} \rangle .    
\end{eqnarray}
Taking the limit with respect to $n$,
since $(T_{\lambda})$ is a net of contractions, we get that
$\langle (\mathbf{1} \otimes \tilde{S}) (z) ,
\tilde{\phi}\rangle \geq 0$, which contradicts \eqref{eq:Random6347},
and so $\tilde{\phi}$ is separable on
$\mathfrak{M} \bar{\otimes} \mathfrak{N}$. $\Box$ \\
That essentially generalizes the result of St{\o}rmer's, who proved a similar theorem under
additional conditions. Namely, he used all positive linear maps and assumed that $\mathcal{B}$
was an UHF algebra \cite{Stormer2009}.

