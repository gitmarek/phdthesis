\thispagestyle{empty}

\begin{center}
{ \large \bfseries
    Matematyczne metody algebr operatorów
    w analizie kryterium splątania
    złożonych układów kwantowych
}\\[1.5cm]

% Author and supervisor
{
\begin{tabular}{lcr}
\emph{Autor:} & \hspace{3cm} & \emph{Promotor:}\\
Marek \textsc{Miller} & & prof. dr hab. \\
    & & Robert \textsc{Olkiewicz}
\end{tabular}
}
\\[1cm]
Streszczenie\\[0.5cm]
\end{center}

{
Praca przedstawia wyniki dotyczące struktury dodatnich odwzorowań liniowych
na algebrach operatorowych, w szczególności na algebrach macierzy,
pod kątem ich zastosowań do zagadnień kwantowej teorii informacji,
ze szczególnym naciskiem na analizę jednego z kryteriów splątania złożonych
układów kwantowych.
Bezpośredni związek badanego kryterium z matematyczną teorią odwzorowań dodatnich
stanowi fizyczną motywację do poszukiwania przykładów odwzorowań ekstremalnych,
a także badania złożonej struktury tych obiektów.

Zaczynając od poprawnej definicji splątania stanów na algebrach von Neumanna
i C*-algebrach,
w pracy udowodniono twierdzenie analogiczne do kryterium splątania
znanego wcześniej w literaturze, jednak tym razem dla najbardziej
z fizycznego punktu widzenia ogólnego przypadku iniektywnych algebr
von Neumanna oraz tzw. jądrowych C*-algebr.
Następnie analizie została poddana struktura odwzorowań dodatnich na
nisko wymiarowych algebrach macierzy.
Pierwszym krokiem było pełne scharakteryzowanie odwzorowań na algebrze
$\mathcal{B}(\mathbb{C}^{2})$ za pomocą geometrycznych metod znanych z analizy
operatorów liniowych na stożkach w przestrzeniach rzeczywistych.
Następnie wykorzystane zostały tzw. stabilne podprzestrzenie półgrup
generowanych przez odwzorowania dodatnie na
$\mathcal{B}(\mathbb{C}^{3})$ do uzyskania
wstępnej klasyfikacji ekstremalnych odwzorowań dodatnich
zachowujących ślad i operator identyczności,
czyli tzw. odwzorowań \emph{bistochastycznych}.
Otrzymany rezultat posłużył do zaprezentowania oryginalnego przykładu
ekstremalnego odwzorowania dodatniego na algebrze macierzy
$\mathcal{B}(\mathbb{C}^{3})$, niebadanego wcześniej,
który stanowi swego rodzaju nowy element teorii.
Pracę kończy rozdział
ukazujący istotną rolę elementów idempotentnych w półgrupie bistochastycznych
odwzorowań dodatnich oraz wskazujący na dalsze perspektywy badań w tej dziedzinie.
}
\vfill
