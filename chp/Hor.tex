\chapter{Kryterium separowalności stanów kwantowych}
\label{chp:PHcrit}
\section{Kryterium Peresa-Horodeckich dla odwzorowań na algebrach von Neumanna}
W niniejszym rozdziale przedstawimy dowód twierdzenia
opublikowanego pierwotnie w pracy \cite{miller2014horodeckis}
i uogólniającego kryterium Peresa-Horodeckich na układy
kwantowe opisywane przez algebry von Neumanna,
spośród których jedna jest algebrą iniektywną.
Poczynając od najprostszego przypadku algebr skończenie wymiarowych,
będziemy w kolejnych krokach dążyć do coraz większego stopnia
ogólności dzięki znajomości struktury algebr kolejnych typów.

\vspace{0.5cm}
Dla ogólnych przestrzeni Banacha $X$ i $Y$, przez
$\mathcal{B}(X,Y)$
oznaczmy przestrzeń Banacha wszystkich ograniczonych odwzorowań
liniowych z $X$ w $Y$.
Przez $X^{*}$ wcześniej oznaczyliśmy przestrzeń funkcjonałów liniowych na $X$,
$X^{*} = \mathcal{B}(X, \mathbb{C})$.
Działanie funkcjonału $\varphi \in X^{*}$ na $x \in X$
zapiszemy: $\langle  \varphi , x \rangle = \varphi(x)$.
Powiemy, że net operatorów $(T_{\lambda})$,
$T_{\lambda} \in \mathcal{B}(X, Y^{*})$, zbiega do
$T \in \mathcal{B}(X, Y^{*})$ w \emph{$^{*}$-słabej topologii operatorowej}, jeśli
\label{page:weakstaroperatortop}
$\langle T_{\lambda} x, y \rangle \rightarrow \langle Tx, y \rangle$
dla każdego $x \in X$ i $y \in Y$.
Jeśli $X$ i $Y$ są również przestrzeniami operatorowymi,
(por. E.\,G.\,Effros, Z.-J.\,Ruan, \emph{Operator spaces} \cite{Effros2000})
przez $\mathcal{CB}(X,Y)$ oznaczymy przestrzeń operatorową
kompletnie ograniczonych odwzorowań pomiędzy $X$ i $Y$,
wyposażoną w normę
$|| \cdot ||_{\mathcal{CB}}$.
Rezygnując z drobiazgowego przytaczania definicji przestrzeni operatorowych
i odwzorowań kompletnie ograniczonych tych przestrzeni,
zaznaczmy jedynie, że zarówno algebry von Neumanna, jak i
ich przestrzenie predualne są przestrzeniami operatorowymi.
Podobnie rzecz ma się z C*-algebrami i ich przestrzeniami dualnymi.
Ponadto istotny będzie dla nas fakt, iż
przestrzeń $\mathcal{F}(X,Y)$
odwzorowań skończonego rzędu z $X$ w $Y$,
tzn. odwzorowań, których obraz jest podprzestrzenią liniową skończonego wymiaru
przestrzeni $Y$, zawiera się w $\mathcal{CB}(X,Y)$.
Pamiętajmy też, że dla przekształcenia $T$ skończonego rzędu
$||T||_{CB} = ||T||$.
Przez $\mathcal{CB}(X,Y)_{+}$
oznaczmy stożek kompletnie ograniczonych dodatnich odwzorowań
z $X$ w $Y$, a przez $\mathcal{CP}(X,Y)$ stożek odwzorowań kompletnie dodatnich.
Wiadomo, że $\mathcal{CP}(X,Y) \subset \mathcal{CB}(X,Y)_{+}$.

Przypomnijmy, że macierz gęstości na $\mathcal{H}\otimes\mathcal{K}$,
gdzie $\mathcal{H}$ i $\mathcal{K}$
są skończenie wymiarowymi przestrzeniami Hilberta,
jest \emph{separowalna},
jeśli jest kombinacją wypukłą prostych iloczynów tensorowych macierzy gęstości
z $\mathcal{H}$ i $\mathcal{K}$.
Ta definicja została potem uogólniona przez Wernera \cite{werner1989quantum}
na nieskończenie wymiarowe przestrzenie Hilberta poprzez
przejście do granicy w normie śladowej takich właśnie kombinacji wypukłych.
W tym samym duchu, zdefiniujmy wypukły zbiór stanów separowalnych.

Należy zaznaczyć, że przestrzeń predualna do iloczynu tensorowego algebr von Neumanna
$(\mathfrak{M} \bar{\otimes} \mathfrak{N})_{*}$
jest izometrycznie izomorficzna z
$\mathfrak{M}_*\hat{\otimes}\mathfrak{N}_*$, dopełnieniem
$\mathfrak{M}_*\otimes\mathfrak{N}_*$ w rzutowej normie operatorowej
\label{def:projectiveoperatornorm}
$\|\cdot\|_\wedge$,
a więc $(\mathfrak{M}_*\hat{\otimes}\mathfrak{N}_*)^*= \mathfrak{M} \bar{\otimes} \mathfrak{N}$
(por.: Effros, Ruan \cite{Effros2000}, twierdzenie 7.2.4;
a także wniosek 7.1.5).

\begin{Definition}
Zbiorem stanów separowalnych na $\mathfrak{M} \bar{\otimes} \mathfrak{N}$
nazwiemy podzbiór
$C_{1} \subset \mathcal{S}(\mathfrak{M} \bar{\otimes} \mathfrak{N})$
zdefiniowany przez
\begin{linenomath*}
 \begin{equation}
    \label{def:SeparableStates234}
C_{1} = \overline{{\rm conv}}^{||\cdot||_{\wedge}}
\left \{\omega \otimes \varphi: \, \,\omega \in \mathcal{S}(\mathfrak{M}),
\varphi \in \mathcal{S}(\mathfrak{N})\right \}.
 \end{equation}
\end{linenomath*}
Niech także $C_{1}^{*}$ oznacza stożek dualny:
$C_{1}^{*} = \{ x \in \mathfrak{M} \bar{\otimes} \mathfrak{N}:
 \langle x , \varphi \rangle \geq 0 \,\, \forall \varphi \in C_{1} \}$ oraz
niech $C_{0}^{*}$ będzie stożkiem zawartym w
$C_{1}^{*}$:
$C_{0}^{*} = \{ x \in \mathfrak{M} \otimes \mathfrak{N}:\langle x , C_{1} \rangle \geq 0 \}$.
\end{Definition}


Aby dowieść twierdzenia uogólniającego kryterium Peresa-Horodeckich
na stany na algebrach von Neumanna,
potrzebna nam będzie możliwość jednostajnego przybliżania elementów z algebry
za pomocą dodatnich operatorów skończonego rzędu.
To prowadzi do założenia iniektywności przynajmniej jednej z rozpatrywanych algebr.
\begin{Definition}
\label{def:injectivevNalgebras}
Algebra von Neumanna $\mathfrak{M}\subset B(\mathcal{H})$ jest \emph{iniektywna},
jeśli istnieje odwzorowanie rzutowe o normie jednostkowej (niekoniecznie normalne)
$\pi\!: B(\mathcal{H})\to\mathfrak{M}$ będące suriekcją.
\end{Definition}
Dla iniketywnych algebr von Neumanna
istnieje net $(A_{\lambda})$ kompletnie dodatnich
kontrakcji skończonego rzędu na  $\mathfrak{M}_*$,
zbiegających w silnej topologi do odwzorowania identycznościowego:
\begin{linenomath*}
 \begin{equation}
\label{RandomLabel:513684}
 || A_{\lambda} \omega - \omega || \stackrel{\lambda}{\longrightarrow} 0, \quad
 \end{equation}
\end{linenomath*}
dla każdego $\omega \in \mathfrak{M}_{*}$
(por. M.\,Takesaki, \emph{Theory of operator algebras}  \cite{Takesaki3},
\mbox{t.\,3.}, twierdzenie XV.3.1).
Ta właśnie cecha iniektywnych algebr von Neumanna okaże się bardzo przydatna
w dalszej części.
Należy tu zaznaczyć, że w zasadzie wszystkie rozpatrywane w ramach analizy
własności układów kwantowych algebry von Neumanna są iniektywne.
Dla przykładu: algebry przemienne (reprezentujące układy klasyczne),
skończenie wymiarowe,
algebra $B(\mathcal{H})$ dla dowolnej przestrzeni Hilberta $\mathcal{H}$,
faktory typu $II_1$ reprezentujące nieskończone łańcuchy spinów,
algebry Araki-Woodsa reprezentujące nierelatywistyczny gaz bozonów w nieskończonej objętości
\cite{araki1968classification},
faktory typu $III_1$ reprezentujące lokalne algebry $\mathcal{M}(O)$
w algebraicznej kwantowej teorii pola \cite{yngwason2005role}.
Ponieważ iloczyn tensorowy algebr iniektywnych pozostaje wciąż algebrą iniektywną,
więc również algebry o nietrywialnym centrum należą do klasy algebr iniektywnych.
Możemy więc powiedzieć, że zaproponowane poniżej uogólnienie kryterium Peresa-Horodeckich
obejmuje wszystkie, z fizycznego punktu widzenia interesujące, przypadki.

\paragraph{}
Zacznijmy od sformułowania twierdzenia, którego dowód przedstawimy poniżej.

\begin{Theorem}
    \label{thm:PHcrit}
    Niech $\mathfrak{M}$ i  $\mathfrak{N}$ będą algebrami von Neumanna.
    Niech także  $\mathfrak{M}$ będzie algebrą iniektywną.
    Wówczas stan $\tilde{\phi} \in \mathcal{S}(\mathfrak{M} \bar{\otimes} \mathfrak{N})$
    jest separowalny wtedy i tylko wtedy, gdy funkcjonał
    $\tilde{\phi} \circ (\mathbf{1} \otimes S)$
    jest dodatni dla każdego dodatniego normalnego odwzorowania skończonego rzędu
    $S\!: \mathfrak{M} \rightarrow \mathfrak{N}$.
\end{Theorem}

\begin{proof}
Załóżmy, że $\tilde{\phi}$ jest stanem separowalnym.
Niech $S: \mathfrak{M} \rightarrow \mathfrak{N}$ będzie
dodatnim normalnym odwzorowaniem, a $S_*$ odwzorowaniem predualnym.
Niech $z$ będzie dodatnim elementem
$\mathfrak{M} \bar{\otimes} \mathfrak{M}$.
Ponieważ
\begin{linenomath*}
 \begin{equation}
    \tilde{\phi} =\lim \limits_{n \rightarrow \infty}
\sum_{i=1}^{n}\alpha_i^{(n)} \omega_{i} \otimes \varphi_{i},
 \end{equation}
\end{linenomath*}
gdzie
$\alpha_i^{(n)}\geq 0$, $\sum_{i=1}^n\alpha_i^{(n)}=1$,
$\omega_{i}\in\mathcal{S}(\mathfrak{M})$
oraz
$\varphi_{i}\in\mathcal{S}(\mathfrak{N})$,
więc
\begin{linenomath*}
 \begin{equation}
 \langle (\mathbf{1} \otimes S)(z), \tilde{\phi} \rangle =
 \lim \limits_{n \rightarrow \infty}\sum \limits_{i = 1}^{n}\alpha_i^{(n)}
 \langle z, \omega_{i} \otimes S_{*} \varphi_{i} \rangle \geq 0.
 \end{equation}
\end{linenomath*}
Dowód przeciwnej implikacji rozłożymy na ciąg następujących po sobie kroków.
\end{proof}

Na początek przedstawmy kilka faktów, które okażą się użyteczne w dalszej
części wywodu.

Niech  $\mathfrak{M}$, $\mathfrak{N}$
będą dowolnymi algebrami von Neumanna,
$\mathfrak{M}\subset B(\mathcal{H})$ oraz $\mathfrak{N}\subset B(\mathcal{K})$,
gdzie  $\mathcal{H},\mathcal{K}$
są przestrzeniami Hilberta, niekoniecznie ośrodkowymi.
Ich zbiory stanów, tj. wypukłe zbiory normalnych, dodatnich i znormalizowanych
funkcjonałów, oznaczyliśmy odpowiednio przez
$\mathcal{S}(\mathfrak{M})$ i $\mathcal{S}(\mathfrak{N})$.

\begin{Lemma}
\label{thm:isometry}
Odwzorowanie $F: \mathcal{CB}(\mathfrak{M}_{*}, \mathfrak{N})
\rightarrow \mathfrak{M} \bar{\otimes} \mathfrak{N}$,
zadane przez
\begin{linenomath*}
 \begin{equation}
\langle F(T), \omega \otimes \varphi \rangle =\langle T\omega, \varphi \rangle,
 \end{equation}
\end{linenomath*}
gdzie $\omega \in \mathfrak{M}_{*}$, $\varphi \in \mathfrak{N}_{*}$,
rozszerza się do kompletnie izometrycznego izomorfizmu pomiędzy przestrzeniami
operatorowymi.
\end{Lemma}
\begin{proof}
  Por.: \cite{Effros2000}, wniosek 7.1.5.
\end{proof}

Z powyższego lematu w prosty sposób wynika, że
$F(\mathcal{CB}(\mathfrak{M}_{*}, \mathfrak{N})_{+}) = C_{1}^{*}$ oraz
$F(\mathcal{CP}(\mathfrak{M}_{*}, \mathfrak{N})) = (\mathfrak{M} \bar{\otimes} \mathfrak{N})_{+}$,
gdzie $(\mathfrak{M} \bar{\otimes} \mathfrak{N})_{+}$
jest stożkiem dodatnich elementów algebry.

Dla każdej algebry von Neumanna $\mathfrak{M}$,
istnieje kompletnie izometryczne zanurzenie
\begin{linenomath*}
 \begin{equation}
\theta: \mathfrak{M} \check{\otimes} \mathfrak{M}_{*}\rightarrow
\mathcal{CB}(\mathfrak{M}_{*}, \mathfrak{M}_{*}),
 \end{equation}
\end{linenomath*}
gdzie $\check{\otimes}$ oznacza iniektywny iloczyn tensorowy przestrzeni operatorowych.
(por. \cite{Effros2000}, stwierdzenie 8.1.2).
Dla przykładu, jeśli
$u \in \mathfrak{M} \otimes \mathfrak{M}_{*}$, $u = \sum_{i = 1}^{n} a_{i} \otimes \omega_{i}$,
wówczas dla $\omega \in \mathfrak{M}_{*}$,
\begin{linenomath*}
 \begin{equation}
\theta(u)\omega = \sum_{i=1}^{n} \langle a_{i} , \omega \rangle \, \omega_{i}.
 \end{equation}
\end{linenomath*}
Posiłkując się izomorfizmem $F$,
udowodnijmy następujący
\begin{Lemma}
\label{prop:34523}
Niech $\omega\in \mathcal{S}(\mathfrak{M} \bar{\otimes} \mathfrak{N})$.
Wówczas $\langle F(A),\omega\rangle\geq 0$ $\forall A\in
\mathcal{CB}(\mathfrak{M}_*,\mathfrak{N})_+$,
wtedy i tylko wtedy, gdy $\omega$ jest separowalne.
\end{Lemma}
\begin{proof}
$\Rightarrow$ Załóżmy, że $\omega\notin C_1$.
Ponieważ $C_1$ jest zbiorem normowo domkniętym i wypukłym,
istnieje $x \in \mathfrak{M} \bar{\otimes} \mathfrak{N}$,
takie że $\langle x, \omega\rangle<0$ oraz $\langle x,C_1\rangle\geq 0$.
Zatem $x\in C_1^*$, a więc istnieje $S \in \mathcal{CB}(\mathfrak{M}_*,\mathfrak{N})_+$,
dla którego $\langle F(S),\omega\rangle<0$, co pociąga za sobą sprzeczność.\\
$\Leftarrow$ Dowód przeciwnej implikacji jest oczywisty.
\end{proof}

Przejdźmy teraz do dalszej części dowodu twierdzenia \ref{thm:PHcrit}.

{\it Krok 1.}
\begin{Lemma}
\label{lem:netinvNalg}
Załóżmy, że algebra $\mathfrak{M}$ jest iniektywna oraz
$(A_{\lambda})$ to net kompletnie dodatnich kontrakcji skończonego rzędu na
$\mathfrak{M}_{*}$, takich że
$A_{\lambda} \stackrel{\lambda}{\rightarrow} \mathbf{1}$ w silnej topologii.
Jeżeli $z \in \mathfrak{M} \bar{\otimes} \mathfrak{N}$, wówczas
$z = {\rm w}^*\mbox{-}\lim_{\lambda} \, (\mathbf{1} \otimes F^{-1}(z) )
(\theta^{-1} (A_{\lambda}))$.
\end{Lemma}
\begin{proof}
Ponieważ każda z norm $||A_{\lambda}||_{\mathcal{CB}} = || A_{\lambda} ||  \leq 1$,
zatem wystarczy pokazać, że dla dowolnego $\omega \in \mathfrak{M}_{*}$
i $\varphi \in \mathfrak{N}_{*}$,
mamy
$\langle z_{\lambda}, \omega \otimes \varphi\rangle{\longrightarrow}
\langle z , \omega \otimes \varphi\rangle$,
gdzie przez $(z_{\lambda})$ oznaczyliśmy net $z_{\lambda} = (\mathbf{1} \otimes \
F^{-1}(z))(\theta^{-1}(A_{\lambda})) \in \mathfrak{M} \otimes \mathfrak{N}$.
Kładąc
$\theta^{-1}(A_{\lambda}) = \sum_{i=1}^{N_{\lambda}} \
a_{i}^{\lambda} \otimes \omega_{i}^{\lambda}$, dostajemy
\begin{linenomath*}
 \begin{multline}
\langle z_{\lambda}, \omega \otimes \varphi\rangle =
\sum \limits_{i=1}^{N_{\lambda}}
\langle a_{i}^{\lambda} \otimes F^{-1}(z) \omega_{i}^{\lambda} ,\
\omega \otimes \varphi \rangle
= \sum \limits_{i=1}^{N_{\lambda}}
\langle  a_{i}^{\lambda}, \omega \rangle
\langle z,  \omega_{i}^{\lambda} \otimes \varphi \rangle = \\
= \langle z , (A_{\lambda} \omega) \otimes \varphi \rangle.
 \end{multline}
\end{linenomath*}
Ponieważ $A_{\lambda} \stackrel{\lambda}{\rightarrow} \mathbf{1}$ w silnej topologii,
otrzymujemy:
\begin{linenomath*}
 \begin{equation}
| \langle z - z_{\lambda}, \omega\otimes \varphi \rangle | \
\leq || A_{\lambda} \omega - \omega || \, ||\varphi|| \, ||z|| \
\stackrel{\lambda}{\longrightarrow} 0.
 \end{equation}
\end{linenomath*}
\end{proof}

\begin{Corollary}
\label{cor:denseness}
Powyższy lemat pokazuje, że $C_0^*$ jest $^{*}$-słabo gęsty w $C_1^*$.
Rzeczywiście, dla
$z \in C_{1}^{*}$, skoro $(A_{\lambda})$ jest netem kompletnie dodatnich
odwzorowań skończonego rzędu, każde $z_{\lambda} \in C_{0}^{*}$.
\end{Corollary}

{\it Krok 2: Faktory skończone, iniektywne i ośrodkowe.}\\
Załóżmy, że $\mathcal{R}$ jest iniektywnym ośrodkowym faktorem typu
II$\phantom{}_{1}$ \cite{Takesaki3};
niech $\tau$ będzie znormalizowanym śladem na $\mathcal{R}$,
a więc normalnym funkcjonałem, dla którego
$\tau(xy) = \tau(yx)$, dla każdych $x,y \in \mathcal{R}$
oraz $\tau(\mathbf{1}) = 1$.
Algebra von Neumanna jest z definicji \emph{ośrodkowa},
jeśl jej przestrzeń predualna jest ośrodkową przestrzenią Banacha.
Ponieważ $\mathcal{R}$ jest algebrą skończoną, więc istnieje zanurzenie
$i: \mathcal{R} \rightarrow
\mathcal{R}_{*}$, zadane przez $i(x) = \omega_{x}$, $\omega_{x}(y) = \tau(xy)$,
dla każdego $x,y \in \mathcal{R}$.
Jasne jest, że $i(\mathcal{R})$
jest zbiorem gęstym normowo w $\mathcal{R}_{*}$.
Warto przy tym zauważyć, że to zanurzenie jest odwzorowaniem dodatnim i kompletnie
ograniczonym, ale nie kompletnie dodatnim.
Następująca, jawna konstrukcja reprezentacji $\mathcal{R}$
pochodzi z podręcznika autorstwa D.\,Evansa  i Y.\,Kawahigashi
\cite{Evans1998}, przykład 5.16.

Niech $\mathcal{A}_{0}$ będzie sumą mnogościową wstępującego ciągu algebr macierzowych:
\begin{linenomath*}
 \begin{equation}
M_{2}(\mathbb{C}) \subset
M_{2}(\mathbb{C}) \otimes M_{2}(\mathbb{C}) \subset
M_{2}(\mathbb{C}) \otimes M_{2}(\mathbb{C}) \otimes M_{2}(\mathbb{C}) \subset
\ldots\, .
 \end{equation}
\end{linenomath*}
Dla $n$-tej potęgi tensorowej $M_{2}(\mathbb{C})$,
oznaczonej $M_{2}(\mathbb{C})^{\otimes n}$,
definiujemy $\tau(x) = \frac{1}{2^{n}} {\rm Tr}(x)$,
oraz zanurzenie
$M_{2}(\mathbb{C})^{\otimes n}$ w $\mathcal{A}_{0}$,
$x \mapsto x \otimes \mathbf{1} \otimes \mathbf{1} \otimes \ldots$.
Wówczas $\tau$ jest dobrze zdefiniowanym stanem śladowym na $\mathcal{A}_{0}$.
Niech $\mathcal{H}$ będzie ośrodkową przestrzenią Hilberta, powstałą jako uzupełnienie
$\mathcal{A}_{0}$ z użyciem iloczynu skalarnego
$\langle x , y \rangle = \tau(y^{*} x )$.
Możemy potraktować $\mathcal{A}_{0}$ jako
\mbox{*-podalgebrę} $B(\mathcal{H})$,
pozwalając $\mathcal{A}_{0}$ działać na $\mathcal{H}$ poprzez mnożenie z lewej strony.
Następnie zdefiniujmy algebrę von Neumanna
$\mathcal{R}$ jako domknięcie w słabej topologii algebry
$\mathcal{A}_{0} \subset B(\mathcal{H})$.
Wiadomo, że
$\mathcal{R}$ jest ośrodkowym faktorem typu II$\phantom{}_{1}$
oraz że rozszerzenie funkcjonału $\tau$ na $\mathcal{R}$,
które będziemy oznaczać tym samym symbolem,
jest jedynym unormowanym śladem na $\mathcal{R}$.
Następnie zdefiniujmy liniowe odwzorowanie
$\beta_{0}$ na $\mathcal{A}_{0}$, takie że
\begin{linenomath*}
 \begin{equation}
\beta_{0}(x^{1} \otimes \ldots\otimes x^{n} \otimes \mathbf{1} \otimes \ldots ) =
(x^{1})^{t} \otimes \ldots\otimes (x^{n})^{t} \otimes \mathbf{1} \otimes \ldots,
 \end{equation}
\end{linenomath*}
gdzie $(x^{i})^{t}$ oznacza transpozycję $x^{i} \in M_{2}(\mathbb{C})$,
$i = 1,2,\ldots,n$.
Jasne jest, że $\beta_{0}$ to *-antyautomorfizm na $\mathcal{A}_{0}$
oraz że istnieje jego rozszerzenie do $\sigma$-słabo ciągłego (normalnego)
*-antyautomorfizmu $\beta$ na algebrze $\mathcal{R}$.
Pomimo faktu, że ani odwzorowanie $i$, ani $\beta$ nie są kompletnie dodatnie,
$\beta$ nie jest nawet kompletnie ograniczone,
pokażemy, że ich złożenie $i \circ \beta$ jest kompletnie dodatnią kontrakcją.
Rzeczywiście, niech $x_{1}, x_{2}, \ldots x_{n}$ oraz
$y_{1}, y_{2}, \ldots y_{n}$ będą elementami $\mathcal{R}$
dla $n \in \mathbb{N}$.
Wówczas
\begin{linenomath*}
 \begin{multline}
\sum \limits_{i, j =1}^{n}
\langle x_{i}^{*} x_{j}, i \circ \beta(y_{i}^{*} y_{j}) \rangle =
\sum \limits_{i, j =1}^{n}
\langle x_{i}^{*} x_{j}, i ( \beta( y_{j}) \beta(y_{i}^{*})) \rangle = \\
=\sum \limits_{i, j =1}^{n}
\tau \left(\beta( y_{j}) \beta(y_{i}^{*}) x_{i}^{*} x_{j}\right) =
\sum \limits_{i, j =1}^{n} \tau \left( (x_{i} \beta(y_{i}) )^{*} \,  x_{j} \beta( y_{j})
\right) =\\
= \tau \left( \
\left( \sum \limits_{i=1}^{n} x_{i} \beta(y_{i}) \right)^{*}
\left( \sum \limits_{j=1}^{n} x_{j} \beta(y_{j}) \right)\right ) \geq 0,
 \end{multline}
\end{linenomath*}
co dowodzi, że $i \circ \beta$ jest kompletnie dodatnie.
Ponieważ oba odwzorowania, $i$ oraz $\beta$ są kontraktywne,
zatem również $i\circ\beta$ ma normę mniejszą od jedności.
Od tego momentu będziemy używać skrótowej notacji: $i\beta$.

Załóżmy obecnie, że
$T: \mathcal{R} \rightarrow M_{2}(\mathbb{C})^{\otimes n}\subset \mathcal{R}$
jet normalnym odwzorowaniem z $\mathcal{R}$ na $n$-tą potęgę tensorową algebry
$M_{2}(\mathbb{C})$, którą w naturalny sposób traktujemy jako podalgebrę $\mathcal{R}$.
Stosując pewną wersję izomorfizmu Choi-Jamiołkowskiego
\cite{choi1975completely,jamiolkowski1974effective},
możemy popatrzeć na odwzorowanie predualne $T_{*}$ jako na działające na
iloczynie tensorowym $\mathcal{R}\otimes\mathcal{R}$.
Mianowicie, niech $e_{ij}$ będą macierzami z $M_{2}(\mathbb{C})$
o zerowych elementach, poza elementem $ij$, równym 1.
Rozważmy elementy bazy w
$M_{2}(\mathbb{C})^{\otimes n}$:
$e_{i_{1} j_{1}} \otimes e_{i_{2} j_{2}} \
\otimes \ldots \otimes e_{i_{n} j_{n}} \otimes \mathbf{1} \otimes \ldots$,
gdzie każdy z indeksów $i_{1}, j_{1}, \ldots, i_{n}, j_{n}$
przybiera wartości 1 lub 2.
Wówczas, dla dowolnego $x \in \mathcal{R}$ i
$\omega \in \mathcal{R}_{*}$, mamy
\begin{linenomath*}
 \begin{multline}
\langle x, T_{*} \omega \rangle =\langle T x,  \omega \rangle
=  2^{n} \sum \limits_{i_{1}, j_{1}\ldots i_{n}, j_{n}}
\tau \left((Tx) e_{j_{1} i_{1}} \otimes \ldots \otimes e_{j_{n} i_{n}}
\otimes \mathbf{1} \otimes \ldots \right) \times \\
\langle e_{i_{1} j_{1}} \otimes \ldots \otimes e_{i_{n} j_{n}}
\otimes \mathbf{1} \otimes \ldots ,\omega \rangle\\
= 2^{n} \sum \limits_{i_{1}, j_{1}\ldots i_{n}, j_{n}} \tau \left((Tx)
\beta (e_{i_{1} j_{1}} \otimes \ldots \otimes e_{i_{n} j_{n}}
\otimes \mathbf{1} \otimes \ldots ) \right) \times \\
\langle e_{i_{1} j_{1}} \otimes \ldots \otimes e_{i_{n}j_{n}}
\otimes \mathbf{1} \otimes \ldots  , \omega \rangle = \\
=  2^{n} \sum \limits_{i_{1}, j_{1}\ldots i_{n}, j_{n}}
\langle Tx ,i \beta (e_{i_{1} j_{1}} \otimes \ldots \otimes e_{i_{n} j_{n}}
\otimes \mathbf{1} \otimes \ldots) \rangle\times\\
\langle e_{i_{1} j_{1}} \otimes \ldots \otimes e_{i_{n} j_{n}}
\otimes \mathbf{1} \otimes \ldots,\omega \rangle.
 \end{multline}
\end{linenomath*}

Stąd, jeśli zdefiniujemy dodatni element $\mathcal{R}\otimes\mathcal{R}$ poprzez
\begin{linenomath*}
 \begin{multline}
m_{n} = 2^{n} \sum \limits_{i_{1}, j_{1}\ldots i_{n}, j_{n}}
\left( e_{i_{1} j_{1}} \otimes e_{i_{2} j_{2}} \otimes \ldots \otimes
e_{i_{n} j_{n}} \otimes \mathbf{1} \otimes \ldots \right) \otimes \\
\left( e_{i_{1} j_{1}} \otimes e_{i_{2} j_{2}}\otimes \ldots \otimes
e_{i_{n} j_{n}} \otimes \mathbf{1} \otimes \ldots\right),
 \end{multline}
\end{linenomath*}
możemy napisać
\begin{linenomath*}
 \begin{equation}
\label{eq:thetainverse}
\theta^{-1}(T_{*}) =(\mathbf{1} \otimes (T_{*} \circ i \beta)) (m_{n}).
 \end{equation}
\end{linenomath*}

\begin{Lemma}
\label{lem:netininjectivefactor}
Niech $\mathcal{R}$
będzie ośrodkowym iniektywnym faktorem typu II$\phantom{}_{1}$;
niech także $z \in C_{1}^{*} \subset \mathcal{R} \bar{\otimes} \mathfrak{N}$.
Wówczas istnieje net $S_{\lambda}$ normalnych dodatnich odwzorowań skończonego rzędu,
$S_{\lambda}: \mathcal{R} \rightarrow \mathfrak{N}$,
oraz ciąg $(m_{n(\lambda)})$ dodatnich elementów $\mathcal{R} \otimes \mathcal{R}$,
dla których
\begin{linenomath*}
 \begin{equation}
z = {\rm w^*-} \lim \limits_{\lambda}(\mathbf{1} \otimes S_{\lambda})(m_{n(\lambda)}).
 \end{equation}
\end{linenomath*}
\end{Lemma}
\begin{proof}
Niech $(A_{\lambda})$ będzie netem kompletnie dodatnich kontrakcji
skończonego rzędu na $\mathcal{R}_{*}$,
zbiegających silnie do odwzorowania identycznościowego.
Możemy założyć, że dualne odwzorowania $A_{\lambda}^{*}$
mają obraz zawarty $M_{2}(\mathbb{C})^{\otimes n(\lambda)}$.
Wtedy, ze wzoru \eqref{eq:thetainverse} wynika, że
\begin{linenomath*}
 \begin{equation}
\theta^{-1} (A_{\lambda}) =
( \mathbf{1} \otimes (A_{\lambda} \circ i\beta)) (m_{n(\lambda)}).
 \end{equation}
\end{linenomath*}
Zdefiniujmy net $(S_{\lambda})$ poprzez
$S_{\lambda} = F^{-1}(z) \circ A_{\lambda} \circ i\beta$.
Skoro $z \in C_{1}^{*}$,
więc każde $S_{\lambda}$ jest dodatnim, normalnym odwzorowaniem skończonego rzędu
na $\mathcal{R}$.
Ponieważ
\begin{linenomath*}
 \begin{equation}
(\mathbf{1} \otimes S_{\lambda}) (m_{n(\lambda)}) =
(\mathbf{1} \otimes F^{-1}(z)) (\theta^{-1}(A_{\lambda})),
 \end{equation}
\end{linenomath*}
teza wynika wprost z Lematu \ref{lem:netinvNalg}.
\end{proof}
\begin{Lemma}
\label{lem:step2}
Niech $\tilde{\phi}$ będzie stanem na
$\mathfrak{M} \bar{\otimes} \mathfrak{N}$,
gdzie $\mathfrak{M}$ jest skończonym, iniektywnym i ośrodkowym faktorem.
Jeżeli $\tilde{\phi} \circ ( \mathbf{1} \otimes S)$
jest dodatnim funkcjonałem na
$\mathfrak{M} \bar{\otimes} \mathfrak{M}$
dla każdego normalnego dodatniego odwzorowania skończonego rzędu
$S: \mathfrak{M} \rightarrow \mathfrak{N}$,
wówczas $\tilde{\phi}$ jest stanem separowalnym.
\end{Lemma}
\begin{proof}
Jeżeli $\mathfrak{M}$ jest skończonym faktorem typu I, tzn. $\mathfrak{M}$ jest
w rzeczywistości skończenie wymiarową algebrą macierzy,
wówczas dowód tezy został podany w pracy Horodeckich
\cite{horodecki1996separability}.
W takim przypadku, zamiast ciągu $(m_{n(\lambda)})$,
tak jak powyżej
wystarczy rozważyć jedynie pojedynczy,
,,maksymalnie splątany'' element iloczynu tensorowego.
Przypuśćmy zatem, że $\mathfrak{M} = \mathcal{R}$.
Wystarczy dowieść, że
$\tilde{\phi}(z) \geq 0$ dla każdego
$z \in C_{1}^{*} \subset \mathfrak{M} \bar{\otimes} \mathfrak{N}$.
Z Lematu \ref{lem:netininjectivefactor} wynika, że
$z = \lim_{\lambda} (\mathbf{1} \otimes S_{\lambda})(m_{n(\lambda)})$,
gdzie każde $S_{\lambda}$
jest dodatnie, normalne i skończonego rzędu,
a każde $m_{n(\lambda)}$ jest dodatnim elementem $\mathfrak{M} \otimes \mathfrak{M}$
oraz granica dotyczy *-słabej topologii.
Korzystając z założeń, otrzymujemy
\begin{linenomath*}
 \begin{equation}
\langle z , \tilde{\phi} \rangle =\lim \limits_{\lambda}
\langle ( \mathbf{1} \otimes S_{\lambda}) (m_{n(\lambda)}), \tilde{\phi}
\rangle \geq 0.
 \end{equation}
\end{linenomath*}
\end{proof}

{\it Krok 3: Skończone iniektywne algebry ośrodkowe}.\\
W tym kroku pozbędziemy się założenia, że $\mathfrak{M}$ jest faktorem.
Jeśli $\tilde{\phi}$ jest stanem na $\mathfrak{M} \bar{\otimes} \mathfrak{N}$,
wówczas $\tilde{\phi}$ jest separowalny,
wtedy i tylko wtedy, gdy
dla każdego centralnego projektora $p \in \mathfrak{M}$
funkcjonał
$\tilde{\phi}_{p} (z) = \langle (p \otimes \mathbf{1}) z, \tilde{\phi}\rangle$
jest separowalny na $p\mathfrak{M} \bar{\otimes} \mathfrak{N}$.
Zatem, możemy rozpatrzeć oddzielnie algebry von Neumanna
typu I i II$\phantom{}_{1}$.
\begin{Lemma}
\label{lem:step3}
Teza Lematu \ref{lem:step2} jest prawdziwa, gdy $\mathfrak{M}$
jest skończoną iniektywną ośrodkową algebrą von Neumanna.
\end{Lemma}
\begin{proof}
Jeśli $\mathfrak{M}$ jest skończoną algebrą von Neumanna typu I,
wówczas $\mathfrak{M}$ jest izomorficzna z algebrą
$M_{n}(\mathbb{C}) \bar{\otimes} \mathcal{A}$, dla pewnego $n \in \mathbb{N}$,
gdzie $\mathcal{A}$ jest algebrą abelową
(por. Blackaddar \cite{Blackadar2006}, III.1.5.12).
Jeżeli zaś $\mathfrak{M}$ jest typu II$\phantom{}_{1}$,
wówczas $\mathfrak{M} \cong  \mathcal{R}\bar{\otimes}\mathcal{C}$,
gdzie $\mathcal{C}$ jest centrum $\mathfrak{M}$
(por. Takesaki \cite{Takesaki3}, twierdzenie XVI.1.5).
Dalsza część dowodu bierze pod uwagę jedynie fakt, że $\mathfrak{M}$
może zostać rozbita na iloczyn tensorowy skończonego ośrodkowego faktora i
abelowej algebry von Neumanna.
Pokażemy zatem dowód jedynie dla przypadku algebry typu II$\phantom{}_{1}$.

Zacznijmy od przedstawienia abelowej algebry $\mathcal{C}$ jako
$L^{\infty}(\Gamma, \mu)$, czyli algebry funkcji istotnie ograniczonych
na lokalnie zwartej przestrzeni topologicznej $\Gamma$,
wraz z dodatnią miarą Radona $\mu$
(por. Takesaki \cite{Takesaki1}, twierdzenie IV.7.17, s. 263).
Co więcej, ponieważ $\mathcal{C}$ jest przemienna,
$\mathfrak{M} \cong L^{\infty}_{\mathcal{R}}(\Gamma, \mu)$,
gdzie $L^{\infty}_{\mathcal{R}}(\Gamma, \mu)$  oznacza
funkcje istotnie ograniczone na $\Gamma$, przyjmujące wartości w $\mathcal{R}$.
Ponadto $\mathfrak{M}_{*} \cong L^{1}_{\mathcal{R}_{*}}(\Gamma, \mu)$,
gdzie ostatni symbol oznacza
przestrzeń funkcji całkowalnych na $\Gamma$, przyjmujących wartości w $\mathcal{R}_{*}$.
Dla każdego $x \in \mathfrak{M}$ i $\omega \in \mathfrak{M}_{*}$,
istnieją funkcje $x:\Gamma\to\mathcal{R}$ oraz
$\omega:\Gamma\to\mathcal{R}_{*}$, takie że
\begin{linenomath*}
 \begin{equation}
\langle x, \omega \rangle =\int \limits_{\Gamma}
\langle x(\gamma), \omega(\gamma) \rangle \, d \mu(\gamma),
 \end{equation}
\end{linenomath*}
przy czym $||\omega|| = \int_{\Gamma} ||\omega(\gamma)|| \, d \mu(\gamma)$
i $||x|| = \sup || x(\gamma) ||$.
Należy zaznaczyć, że z definicji przestrzeni $L^{1}_{\mathcal{R}_{*}}(\Gamma, \mu)$
wynika, iż dla każdego zwartego zbioru
$K \subset \Gamma$ i $\epsilon >0$,
istnieje zwarty podzbiór $K^{'} \subset K$,
taki że $\mu (K \setminus K^{'}) < \epsilon$
oraz funkcja $\gamma \mapsto \omega(\gamma)$ jest ciągła na $K^{'}$.

Dla każdego mierzalnego podzbioru $E \subset \Gamma$
zdefiniujmy odwzorowanie
$i_{E}: \mathcal{R} \rightarrow \mathfrak{M}$ poprzez
$i_{E}(x_{0})(\gamma) = \chi_{E}(\gamma) \, x_{0}$,
gdzie $x_{0} \in \mathcal{R}$
oraz $\chi_{E}$ jest funkcją charakterystyczną zbioru $E$.
Odwzorowanie to jest kompletnie dodatnie i normalne,
a nawet jest *-homomorfizmem.
Niech więc $\tilde{\phi}_{E} = \tilde{\phi} \circ (i_{E} \otimes \mathbf{1})$,
dla każdego $E \subset \Gamma$, $0<\mu(E)<\infty$.
Wówczas $\tilde{\phi}$ jest separowalny, wtedy i tylko wtedy, gdy
każdy funkcjonał $\tilde{\phi}_{E}$ jest z dokładnością do stałej
normującej separowalnym stanem na $\mathcal{R} \bar{\otimes} \mathfrak{N}$.
Aby nie zaciemniać aktualnego rozumowania,
odłóżmy dowód tego raczej technicznego faktu do następnego lematu.

Niech $S_{0}: \mathcal{R} \rightarrow \mathfrak{N}$ będzie dodatnim normalnym
odwzorowaniem skończonego rzędu.
Zdefiniujmy
$S_{E}: \mathfrak{M} \rightarrow \mathfrak{N}$ przez
\begin{linenomath*}
 \begin{equation}
S_{E}(x(\cdot)) = \frac{1}{\mu(E)}
\int \limits_{E} S_{0}x(\gamma) \, d\mu(\gamma),
 \end{equation}
\end{linenomath*}
rozumiane jako całkę Bochnera w $\mathfrak{N}$.
Całka taka istnieje na mocy następującego argumentu.
Ponieważ $S_{0}$ ma skończony rząd, funkcja
$\gamma \mapsto S_{0}x(\gamma)$ jest słabo mierzalna.
Rzecz jasna obraz $S_{0}$ jest przestrzenią ośrodkową i dlatego,
z twierdzenia Pettisa o mierzalności,
funkcja podcałkowa jest mierzalna w sensie Bochnera.
Jej całkowalność jest teraz oczywista.
Ponadto, $S_{E}$ jest dodatnie, normalne i również posiada skończony rząd.
Zauważając, że dla każdego $x_{0} \in \mathcal{R}$,
$S_{0}x_{0} = S_{E} \circ i_{E} (x_{0})$ oraz
biorąc dodatni element $z_{0} \in \mathcal{R} \otimes \mathcal{R}$,
na mocy głównego założenia dostajemy
\begin{linenomath*}
 \begin{equation}
\langle (\mathbf{1} \otimes S_{0})(z_{0}) , \tilde{\phi}_{E} \rangle =
\langle (\mathbf{1} \otimes S_{E})(i_{E} \otimes i_{E})(z_{0}),
\tilde{\phi} \rangle \geq 0.
 \end{equation}
\end{linenomath*}
Z Lematu 3.3, $\tilde{\phi}_{E}$ jest separowalny i dlatego również
$\tilde{\phi}$ jest separowalny,
pod warunkiem, że następny lemat jest prawdziwy.
\end{proof}

\begin{Lemma}
Stan $\tilde{\phi}$ na $\mathfrak{M} \bar{\otimes} \mathfrak{N}$
jest separowalny, wtedy i tylko wtedy, gdy każdy funkcjonał $\tilde{\phi}_{E}$
jest, z dokładnością do stałej normującej, separowalnym stanem na
$\mathcal{R} \bar{\otimes} \mathfrak{N}$
dla każdego mierzalnego zbioru $E \subset \Gamma$,
takiego że $\mu(E) < \infty$.
\end{Lemma}
\begin{proof}
Z faktu, że $\tilde{\phi}$ jest separowalny, w łatwy sposób wynika, że również
$\tilde{\phi}_{E}$ jest separowalny.
Odwrotnie, załóżmy, że każdy funkcjonał $\tilde{\phi}_{E}$ jest separowalny.
Niech $x(\cdot) \in \mathfrak{M}$.
Na mocy elementarnego rozumowania,
podobnego do konstrukcji całki Lebesgue'a,
można pokazać, że ponieważ $\mathfrak{M}$ jest ośrodkową algebrą von Neumanna,
istnieje ciąg funkcji prostych
\begin{linenomath*}
 \begin{equation}
x_{n}(\cdot) =     \sum \limits_{k=1}^{n} \chi_{B_{k}^{(n)}}(\cdot)\, x_{k}^{(n)},
 \end{equation}
\end{linenomath*}
gdzie
\begin{linenomath*}
 \begin{equation}
\label{eq:integral}
x_{k}^{(n)} = \frac{1}{\mu(B_{k}^{(n)})}
\int \limits_{B_{k}^{(n)}} x(\gamma) \, d\mu(\gamma)\quad \in \mathcal{R},
 \end{equation}
\end{linenomath*}
i każde $B_{k}^{(n)} \subset \Gamma$ jest zbiorem mierzalnym oraz
$0< \mu(B_{k}^{(n)}) < \infty$;
co więcej $\langle x_{n}, \omega \rangle\stackrel{n}{\rightarrow}\langle x,
\omega \rangle$ dla każdego $\omega \in \mathfrak{M}_{*}$.
Całka \eqref{eq:integral} odgrywa tu rolę wartości średniej $x(\cdot)$
na zbiorze $B_{k}^{(n)}$ i jest rozumiana w następujący sposób.
Ponieważ dla każdego $\omega_{0} \in \mathcal{R}_{*}$,
funkcja $\gamma \mapsto \langle x(\gamma), \omega_{0} \rangle$
jest mierzalna, więc jeśli $E \subset \Gamma$ jest zbiorem mierzalnym,
$\mu(E) < \infty$,
wówczas całka $\int_{E} \langle x(\gamma), \omega \rangle \, d \mu(\gamma)$
definiuje ograniczony funkcjonał liniowy na przestrzeni $\mathcal{R}_{*}$,
czyli element \mbox{z $\mathcal{R}$}.

Niech teraz $\tilde{\phi} =\lim_{n\to\infty}\sum_{i=1}^{n}\alpha_i(\omega_{i}\otimes\varphi_{i})$,
gdzie granicę bierze się w projektywnej tensorowej normie przestrzeni operatorowych
$|| \cdot ||_{\wedge}$ na $\mathfrak{M}_{*} \hat{\otimes} \mathfrak{N}_{*}$,
oraz niech $z \in C_{0}^{*}$, $z = \sum_{j=1}^{M} x_{j} \otimes y_{j}$.
Jasne jest, że rodzina zbiorów $B_{k}^{(n)}$ we wzorze \eqref{eq:integral}
może zostać wybrana dla każdego $x_{j}$, $j = 1,2, \ldots, M$; tzn.
istnieją ciągi $(x_{j,n})_{n=1}^{\infty}$, zadane przez
\begin{linenomath*}
 \begin{equation}
x_{j,n}(\cdot) =     \sum \limits_{k=1}^{n} \chi_{B_{k}^{(n)}}(\cdot)
\, x_{j,k}^{(n)},
 \end{equation}
\end{linenomath*}
gdzie
\begin{linenomath*}
 \begin{equation}
\label{eq:integralxes}
x_{j,k}^{(n)} = \frac{1}{\mu(B_{k}^{(n)})}
\int \limits_{B_{k}^{(n)}} x_{j}(\gamma) \, d\mu(\gamma)
\quad \in \mathcal{R},
 \end{equation}
\end{linenomath*}
takie że $x_{j,n} \stackrel{n}{\longrightarrow} x_{j}$
w topologii *-słabej na  $\mathfrak{M}$.

Zdefiniujmy następnie odwzorowanie
$\pi_{E}: \mathcal{R}_{*} \rightarrow \mathfrak{M}_{*}$
poprzez $\pi_{E}(\omega_{0})(\gamma) =\frac{1}{\mu(E)} \chi_{E}(\gamma)\omega_{0}$.
Łatwo sprawdzić, że $\pi_{E}$ jest kompletnie dodatnią kontrakcją dla każdego
mierzalnego zbioru $E \subset \Gamma$, takiego że $\mu(E) < \infty$;
a przez to, z założenia, każdy funkcjonał
\begin{linenomath*}
 \begin{equation}
\tilde{\phi}_{n} = \sum \limits_{k=1}^{n}
(\pi_{B_{k}^{(n)}} \otimes \mathbf{1})
( \tilde{\phi}_{B_{k}^{(n)}} )
 \end{equation}
\end{linenomath*}
jest separowalny na $\mathfrak{M} \bar{\otimes} \mathfrak{N}$.
Pokażemy, że $\langle z , \tilde{\phi}_{n} \rangle
\stackrel{n}{\rightarrow}\langle z , \tilde{\phi} \rangle$.
Niech $\epsilon > 0$.
Wybierzmy liczbę naturalną $L$, taką że
$|| \sum_{i=L+1}^{\infty}\alpha_i(\omega_{i} \otimes \varphi_{i}) ||_{\wedge}
\leq \frac{\epsilon}{4 ||z||}$.
Wtedy
\begin{linenomath*}
 \begin{multline}
\label{eq:longineq}
|\langle z, \tilde{\phi} - \tilde{\phi}_{n} \rangle| =
|\langle z, \tilde{\phi} -\sum \limits_{k=1}^{n} \tilde{\phi}
\circ (i_{B_{k}^{(n)}} \pi^{*}_{B_{k}^{(n)}} \otimes \mathbf{1})
\rangle | \leq \\
\leq \Big |\langle z,\sum \limits_{i=1}^{L} \omega_{i} \otimes \varphi_{i} -
\sum \limits_{i'=1}^{L}\sum \limits_{k=1}^{n}
\omega_{i'} \circ (i_{B_{k}^{(n)}} \pi^{*}_{B_{k}^{(n)}})
\otimes \varphi_{i'}
\rangle \Big | + \frac{\epsilon}{2},
 \end{multline}
\end{linenomath*}
ponieważ $\sum_{k=1}^{n} i_{B_{k}^{(n)}} \pi^{*}_{B_{k}^{(n)}}$
jest kontrakcją na $\mathfrak{M}$.
Odwzorowanie dualne
$\pi^{*}_{E}: \mathfrak{M} \rightarrow \mathcal{R}$
działa na $x(\cdot) \in \mathfrak{M}$ przez
\begin{linenomath*}
 \begin{equation}
\label{eq:intergralpi}
\pi^{*}_{E}(x(\cdot)) = \frac{1}{\mu(E)}
\int \limits_{E} x(\gamma) \, d\mu(\gamma).
 \end{equation}
\end{linenomath*}
Niech $N$ będzie liczbą naturalną, taką że dla każdego $i = 1, 2, \ldots, L$
oraz $j = 1,2, \ldots M$, jeśli tylko $n \geq N$,
to $|\langle x_{j} - x_{j,n}, \omega_{i} \rangle| \leq
\frac{\epsilon}{2 L M c_{L} }$, gdzie
$c_{L} = \max \{ |\langle y_{j}, \varphi_{i} \rangle |$,
$i = 1,2, \ldots, L$; $j = 1,2, \ldots, M \}$.
Korzystając ze wzorów \eqref{eq:intergralpi} i \eqref{eq:integralxes},
obliczmy
\begin{linenomath*}
 \begin{multline}
\sum \limits_{k=1}^{n} i_{B_{k}^{(n)}} \pi^{*}_{B_{k}^{(n)}} (x_{j})
=\sum \limits_{k=1}^{n} \chi_{B_{k}^{(n)}}(\cdot) \,
\pi^{*}_{B_{k}^{(n)}} (x_{j}) = \\
=\sum \limits_{k=1}^{n} \chi_{B_{k}^{(n)}}(\cdot) \,
\frac{1}{\mu(B_{k}^{(n)})}
\int \limits_{B_{k}^{(n)}} x_{j}(\gamma) \, d\mu(\gamma)
=\sum \limits_{k=1}^{n} \chi_{B_{k}^{(n)}}(\cdot) \,
x_{j,k}^{(n)}=x_{j,n}.
 \end{multline}
\end{linenomath*}
Kontynuując nierówność \eqref{eq:longineq}, otrzymujemy dla $n \geq N$:
\begin{linenomath*}
 \begin{multline}
   \nonumber
|\langle z, \tilde{\phi} - \tilde{\phi}_{n} \rangle|\leq \Big |\langle z,
\sum \limits_{i=1}^{L} \omega_{i} \otimes \varphi_{i} -\sum \limits_{i'=1}^{L}
\sum \limits_{k=1}^{n}\omega_{i'} \circ (i_{B_{k}^{(n)}} \pi^{*}_{B_{k}^{(n)}})
\otimes \varphi_{i'}\rangle \Big | + \frac{\epsilon}{2} =\\
= \Big |  \sum \limits_{j=1}^{M}\left( \sum \limits_{i =1}^{L}
\langle x_{j}, \omega_{i} \rangle \langle y_{j}, \varphi_{i} \rangle -
\sum \limits_{i'=1}^{L} \sum \limits_{k=1}^{n}
\langle i_{B_{k}^{(n)}} \pi^{*}_{B_{k}^{(n)}} (x_{j}) ,
\omega_{i'} \rangle \langle y_{j}, \varphi_{i'} \rangle
\right) \Big |  + \frac{\epsilon}{2} =\\
= \Big | \sum \limits_{j=1}^{M}\left( \sum \limits_{i=1}^{L}
\langle x_{j}, \omega_{i} \rangle \langle y_{j}, \varphi_{i} \rangle  -
\sum \limits_{i'=1}^{L}\langle x_{j,n} , \omega_{i'} \rangle \langle y_{j}, \varphi_{i'} \rangle
\right) \Big |  + \frac{\epsilon}{2} =\\
= \Big | \sum \limits_{j=1}^{M} \sum \limits_{i=1}^{L}
\left(\langle x_{j}, \omega_{i} \rangle \langle y_{j}, \varphi_{i} \rangle  -
\langle x_{j,n} , \omega_{i} \rangle \langle y_{j}, \varphi_{i} \rangle
\right) \Big |  + \frac{\epsilon}{2} \leq\\
\leq \sum \limits_{j=1}^{M} \sum \limits_{i=1}^{L}
c_{L}  \, \left | \langle    x_{j}, \omega_{i} \rangle -
\langle x_{j,n} , \omega_{i} \rangle\right | + \frac{\epsilon}{2} \leq
\end{multline}
\end{linenomath*}
\begin{linenomath*}
\begin{equation}
\leq \sum \limits_{j=1}^{M} \sum \limits_{i=1}^{L}
c_{L}  \frac{\epsilon}{2 L M c_{L}}     + \frac{\epsilon}{2}
=  \frac{\epsilon}{2} +  \frac{\epsilon}{2} =  \epsilon.
 \end{equation}
\end{linenomath*}
Zatem  $\langle z , \tilde{\phi} \rangle  =
\lim_{n} \langle z , \tilde{\phi}_{n} \rangle \geq 0$;
a skoro $z$ było dowolnym elementem z $C_{0}^{*}$,
na mocy wniosku \ref{cor:denseness}, $\tilde{\phi}$ stanem separowalnym.
\end{proof}

{\it Krok 4: Skończone algebry iniektywne}.\\
Aby nie zakładać dłużej ośrodkowości algebry $\mathfrak{M}$,
posłużymy się następującym rozumowaniem.
Niech $\mathfrak{M}$ będzie skończoną iniektywną algebrą von Neumanna oraz
niech $\tilde{\phi}$ będzie stanem na $\mathfrak{M} \bar{\otimes} \mathfrak{N}$,
takim że $\tilde{\phi} \circ (\mathbf{1} \otimes S) \geq 0$
dla każdego dodatniego odwzorowania skończonego rzędu
$S:\mathfrak{M} \rightarrow \mathfrak{N}$.
Wiadomo (por. \cite{Takesaki3}, s. 178), że
dla każdego skończonego zbioru
$\{ \mathbf{1}, x_{1}, x_{2}, \ldots, x_{n} \} \subset \mathfrak{M}$,
istnieje skończona ośrodkowa algebra iniektywna
$\mathfrak{S} \subset \mathfrak{M}$,
taka że $x_{i} \in \mathfrak{S}$, $i =1, 2, \ldots n$.
Co więcej, $\mathfrak{S}$
jest obrazem normalnego odwzorowania projektywnego
(\emph{conditional expectation}) $\mathcal{E}$ na $\mathfrak{M}$.
Pokażmy, że $\tilde{\phi}$ jest w słabej topologii granicą ciągu stanów separowalnych.
Rzeczywiście, niech wszystkie skończone podzbiory $\mathfrak{M}$,
częściowo uporządkowane ze względu na mnogościowe zawieranie,
stanowią indeksy netu $(\mathcal{E}_{\mu})$
odwzorowań projektywnych $\mathfrak{M}$
na ośrodkowe podalgebry $\mathfrak{S}_{\mu}$,
zawierające właśnie te skończone podzbiory.
Wówczas obcięcie $S|_{\mathfrak{S}_{\mu}}$ odwzorowania $S$
do $\mathfrak{S}_{\mu}$ jest również normalne i dodatnie.
Z \mbox{lematu \ref{lem:step3}}, obcięcie $\tilde{\phi}$ do
$\mathfrak{S}_{\mu} \bar{\otimes} \mathfrak{N}$
jest funkcjonałem separowalnym, a zatem każdy stan
$\tilde{\phi}_{\mu} = \tilde{\phi}|_{\mathfrak{S}_{\mu}
\bar{\otimes} \mathfrak{N}} \circ (\mathcal{E}_{\mu} \otimes \mathbf{1})$
jest również separowalny.
Łatwo dowieść, że net $(\tilde{\phi}_{\mu})$ zbiega słabo do $\tilde{\phi}$.
Skoro zbiór stanów separowalnych jest wypukły,
$\tilde{\phi}$ jest także granicą w normie stanów separowalnych,
a zatem samo $\tilde{\phi}$ jest stanem separowalnym.

\vspace{0.3cm}
{\it Krok 5: Iniektywne algebry półskończone}.

W tym kroku osłabimy założenie, że $\mathfrak{M}$ jest algebrą skończoną.
\begin{Lemma}
Teza lematu \ref{lem:step2} jest słuszna, gdy
$\mathfrak{M}$ jest iniektywną półskońcozną algebrą von Neumanna.
\end{Lemma}
\begin{proof}
Dla każdego skończonego niezerowego projektora $p \in \mathfrak{M}$,
przez $\mathfrak{M}_{p}$ oznaczmy skończoną algebrę von Neumanna
$p \mathfrak{M} p$.
Niech $S : \mathfrak{M}_{p} \rightarrow \mathfrak{N}$ będzie odwzorowaniem
dodatnim i normalnym.
Definiujemy $S_{p} : \mathfrak{M} \rightarrow \mathfrak{N}$
przez $S_{p}(x) = S(pxp)$, $x \in \mathfrak{M}$,
które jest również dodatnie i normalne.
Niech $T_{p}x = p x p$, $x \in \mathfrak{M}$.
Ponieważ odwzorowanie $T_{p}$ jest kompletnie dodatnią normalną kontrakcją,
posiada rozszerzenie do dodatniej kontrakcji
$T_{p} \otimes \mathbf{1}$ na $\mathfrak{M} \bar{\otimes} \mathfrak{N}$.
Zatem możemy zdefiniować dodatni funkcjonał $\tilde{\phi}_{p}$ na
$\mathfrak{M} \bar{\otimes} \mathfrak{N}$ poprzez
$\tilde{\phi}_{p} = \tilde{\phi} \circ (T_{p} \otimes \mathbf{1})$.
Niech $z$ będzie dodatnim elementem $\mathfrak{M}_{p} \otimes \mathfrak{M}_{p}$.
Ponieważ istnieje dodatni element $z' \in \mathfrak{M} \otimes \mathfrak{M}$,
taki że $z = (T_{p} \otimes T_{p})(z')$, więc na mocy założenia
\begin{linenomath*}
 \begin{equation}
\tilde{\phi}_{p} \big|_{\mathfrak{M}_{p} \bar{\otimes} \mathfrak{N}}
\circ (\mathbf{1} \otimes S)(z) =
\tilde{\phi} \circ (\mathbf{1} \otimes S_{p})
\left( (T_{p} \otimes \mathbf{1})(z') \right ) \geq 0.
 \end{equation}
\end{linenomath*}
A zatem, tak jak w dowodzie kroku 4.,
$\tilde{\phi}_{p} \big|_{\mathfrak{M}_{p} \bar{\otimes} \mathfrak{N}}$
jest funkcjonałem separowalnym.
Stąd wynika, że separowalny jest również funkcjonał $\tilde{\phi}_{p}$.
Rzeczywiście, dla $z\in C_{0}^{*}$, mamy
$(T_{p} \otimes \mathbf{1})(z) \in
\mathfrak{M}_{p} \bar{\otimes} \mathfrak{N}$, a więc
\begin{linenomath*}
 \begin{equation}
\tilde{\phi}_{p}(z) = \tilde{\phi}_{p}((T_{p} \otimes \mathbf{1}) z) =
\tilde{\phi}_{p} \big|_{\mathfrak{M}_{p} \bar{\otimes} \mathfrak{N}}
((T_{p} \otimes \mathbf{1}) z) \geq 0,
 \end{equation}
\end{linenomath*}
co dowodzi, że $\tilde{\phi}_{p}$ jest separowalny.

Niech znów $z \in C_{0}^{*} \subset \mathfrak{M} \otimes \mathfrak{N}$
będzie dowolne.
Pokażemy, że $\langle z, \tilde{\phi} \rangle \geq 0$.
Niech $p_{\nu}$ będzie netem skończonych projektorów
półskończonej algebry $\mathfrak{M}$,
takich że $\lim_{\nu} p_{\nu} = \mathbf{1}$,
w *-słabej topologii.
Ponieważ $\mathfrak{M}$ jest algebrą operatorów działającą
na przestrzeni Hilberta,
możemy powiedzieć, że net $p_{\nu}$ zbiega do jedności w
$\sigma$-słabej topologii operatorowej.
To z kolei implikuje, że
$p_{\nu} \stackrel{\nu}{\rightarrow} \mathbf{1}$ w topologii $\sigma$-silnej,
ponieważ $p_{\nu} = p_{\nu}^2= p_{\nu} p_{\nu}^{*}$.
Stąd dla każdego $x \in \mathfrak{M}$ oraz $\omega \in \mathfrak{M}_{*}$,
$\langle p_{\nu} x p_{\nu}, \omega \rangle\stackrel{\nu}{\rightarrow}
\langle x, \omega \rangle$, a ponieważ
$(T_{p_{\nu}} \otimes \mathbf{1})(z) =
(p_{\nu} \otimes \mathbf{1}) z (p_{\nu} \otimes \mathbf{1})$,
dostajemy $\langle z, \tilde{\phi} \rangle
= \lim_{\nu} \langle z, \tilde{\phi}_{p_{\nu}} \rangle \geq 0$,
co kończy dowód.
\end{proof}

{\it Krok 6: Iniektywne algebry typu III}.

\begin{Lemma}
Teza lematu \ref{lem:step2} jest prawdziwa dla iniektywnych algebr von Neumanna
typu III.
\end{Lemma}
\begin{proof}
Ponieważ $\mathfrak{M}$ jest iniektywną algebrą typu III,
istnieje półskończona iniektywna algebra von Neumanna
$\mathfrak{M}_{0}$ oraz odwzorowanie rzutowe o normie jednostkowej
$\mathcal{E}: \mathfrak{M}_{0} \rightarrow \mathfrak{M}$.
Z konstrukcji $\mathfrak{M}_{0}$, możemy potraktować
$\mathfrak{M}$ jako podalgebrę $\mathfrak{M}_{0}$.
Co więcej, $\mathcal{E}$ jest granicą w *-słabej topologii operatorowej
netu $(\mathcal{E}_{\lambda})$ normalnych, jednostkowych i kompletnie dodatnich
odwzorowań $\mathcal{E}_{\lambda} : \mathfrak{M}_{0} \rightarrow \mathfrak{M}$,
(por. Brown, Ozawa \cite{Brown2008}, lemat 9.3.6).
Ponadto
$\mathfrak{M}_{0} = \mathfrak{M} \rtimes_{\alpha} \mathbb{R}$,
gdzie chodzi o iloczyn krzyżowy względem działania modularnego
(\emph{modular action}) $\alpha$ na $\mathfrak{M}$.
Ponieważ $\mathfrak{M}$ jest typu III, więc
$(\mathfrak{M} \rtimes_{\alpha} \mathbb{R}) \rtimes_{\hat{\alpha}} \mathbb{R}
\cong \mathfrak{M} \bar{\otimes} B(L^{2}(\mathbb{R})) \cong \mathfrak{M}$,
gdzie $\hat{\alpha}$ jest działaniem sprzężonym do $\alpha$.
Stąd istnieje projektor o normie jednostkowej
$\tilde{\mathcal{E}} : \mathfrak{M} \rightarrow \mathfrak{M}_{0}$.
Analogicznie, niech $(\tilde{\mathcal{E}}_{\mu})$
będzie netem normalnych, jednostkowych i kompletnie dodatnich odwzorowań
$\tilde{\mathcal{E}}_{\mu}: \mathfrak{M} \rightarrow \mathfrak{M}_{0}$,
zbiegających do $\tilde{\mathcal{E}}$ w *-słabej topologii operatorowej.

Dla każdego $\lambda$, niech $\tilde{\phi}_{\lambda}$ oznacza stan na
$\mathfrak{M}_{0} \bar{\otimes} \mathfrak{N}$, zadany przez
$\tilde{\phi}_{\lambda} = \tilde{\phi} \circ(\mathcal{E}_{\lambda}
\otimes \mathbf{1})$.
Załóżmy, że każde $\tilde{\phi}_{\lambda}$ jest separowalne.
Wówczas, obcięcie do $\mathfrak{M} \bar{\otimes} \mathfrak{N}$ jest
z dokładnością do stałej normującej stanem separowalnym; ponadto
$\tilde{\phi}_{\lambda} \stackrel{\lambda}{\rightarrow}
\tilde{\phi}$ w słabej topologii.
Na mocy standardowego argumentu o wypukłości,
$\tilde{\phi}$ jest również granicą stanów separowalnych w silnej topologii,
co z definicji oznacza, że $\tilde{\phi}$ jest separowalny.
A zatem pozostaje jedynie pokazać, że  $\tilde{\phi}_{\lambda}$
jest netem stanów separowalnych na $\mathfrak{M}_{0} \bar{\otimes} \mathfrak{N}$.
W tym celu ustalmy $\lambda$ oraz załóżmy, że $\tilde{\phi}_{\lambda}$
nie jest separowalny.
Na mocy tego, co zostało powiedziane powyżej,
istnieje dodatnie normalne odwzorowanie
$S_{0}: \mathfrak{M}_{0} \rightarrow \mathfrak{N}$
oraz dodatni element
$z_{0} \in \mathfrak{M}_{0} \bar{\otimes} \mathfrak{M}_{0}$,
taki że
\begin{linenomath*}
 \begin{equation}
\label{eq:Random20203}
\langle(\mathbf{1} \otimes S_{0})(z_{0}),\tilde{\phi}_{\lambda}\rangle < 0.
 \end{equation}
\end{linenomath*}
Co więcej, $z_{0}$ można wybrać tak, aby należało do algebraicznego
iloczynu tensorowego $\mathfrak{M}_{0} \otimes \mathfrak{M}_{0}$.
Stąd, ponieważ $\tilde{\mathcal{E}}$ jest suriekcją,
istnieje dodatnie
$z \in \mathfrak{M} \otimes \mathfrak{M}$, takie że
$(\tilde{\mathcal{E}} \otimes \tilde{\mathcal{E}})(z) = z_{0}$.
Niech $S_{\mu}: \mathfrak{M} \rightarrow \mathfrak{N}$
będzie dodatnim normalnym odwzorowaniem, zadanym przez
$S_{\mu} = S_{0} \circ \tilde{\mathcal{E}}_{\mu}$.
Wtedy z założenia oraz ponieważ oba odwzorowania
$\mathcal{E}_{\lambda}$ i $\tilde{\mathcal{E}}$
są kompletnie dodatnie, mamy
\begin{linenomath*}
 \begin{multline}
\langle(\mathbf{1} \otimes S_{0})(z_{0}),\tilde{\phi}_{\lambda} \rangle =
\lim \limits_{\mu} \langle(\mathbf{1} \otimes S_{\mu})(\tilde{\mathcal{E}}
\otimes \mathbf{1})(z) ,\tilde{\phi}_{\lambda}\rangle = \\
= \lim \limits_{\mu} \langle
(\mathbf{1} \otimes S_{\mu})(\mathcal{E}_{\lambda}\circ
\tilde{\mathcal{E}}\otimes \mathbf{1})(z) ,\tilde{\phi}\rangle \geq 0,
 \end{multline}
\end{linenomath*}
co stoi w sprzeczności z \eqref{eq:Random20203}.
Zatem, $\tilde{\phi}_{\lambda}$
jest separowalne.
\end{proof}

Fakt, że dla każdej iniektywnej algebry $\mathfrak{M}$
istnieje centralny projektor $p$, taki że $p\mathfrak{M}$
jest półskończona i $({\bf 1}-p)\mathfrak{M}$ jest typu III,
kończy dowód twierdzenia \ref{thm:PHcrit}.


\section{Kryterium Peresa-Horodeckich dla odwzorowań na C$^*$-algebrach}
\label{sec:HorCstar}
Rezultat analogiczny do przedstawionego powyżej twierdzenia \ref{thm:PHcrit}
można otrzymać również dla odwzorowań na C$^*$-algebrach.
Należy się spodziewać, że poprzednie założenie o iniektywności jednej z
algebr będzie musiało zostać w jakiś sposób zachowane i w tym przypadku.
Stąd potrzeba wprowadzenie tzw. algebr jądrowych (\emph{nuclear algebras}).
Z definicji $\mathcal{A}$ jest jądrową C$^{*}$-algebrą,
jeżeli $\mathcal{A}^{**}$, druga przestrzeń dualna,
jest izomorficzna z iniektywną algebrą von Neumanna.
Dla algebr jądrowych iloczyn tensorowy jest zdefiniowany jednoznacznie.
Ważną własnością jądrowych C$^{*}$-algebr jest również lokalna rekfleksywność,
która okaże się pomocna na dalszym etapie dowodu.

Niech  $\mathcal{A}$ oraz $\mathcal{B}$ będą C$^{*}$-algebrami.
Wiadomo, że na algebraicznym iloczynie tensorowym
$\mathcal{A} \otimes \mathcal{B}$ można wprowadzić wiele norm,
dla których przestrzeń ta staje się C*-algebrą.
Dwie spośród nich (norma \emph{minimalna} i \emph{maksymalna})
zajmują wyróżnioną pozycję, ograniczając wszystkie pozostałe normy z dołu i z góry.
Jeśli przez $\mathcal{A} \otimes_{max} \mathcal{B}$
i $\mathcal{A} \otimes_{min} \mathcal{B}$,
oznaczyć odpowiednio
\emph{maksymalny} i \emph{minimalny}
iloczyn tensorowy C$^{*}$-algebr $\mathcal{A}$ i $\mathcal{B}$,
wówczas wiadomo, że gdy jedna z nich, powiedzmy algebra $\mathcal{A}$,
jest jądrowa, normy tensorowe na obu iloczynach są sobie równe,
a co za tym idzie -- iloczyn tensorowy tych algebr jest zdefiniowany
jednoznacznie (por. Brown, Ozawa \cite{Brown2008}, twierdzenie 3.8.7).
Taki iloczyn będziemy oznaczać przez
$\mathcal{A} \bar{\otimes} \mathcal{B}$.
C$^{*}$-algebra $\mathcal{A}$ jest ponadto z definicji \emph{lokalnie refleksywna},
jeśli dla każdego skończenie wymiarowego układu operatorowego
$E \subset \mathcal{A}^{**}$,
tj. domkniętej przestrzeni zawierającej operator jednostkowy i zamkniętej
na sprzężenie hermitowskie operatorów,
istnieje net $(T_{\lambda})$ kompletnie dodatnich kontrakcji,
$T_{\lambda}: E \rightarrow \mathcal{A}$,
zbieżny do odwzorowanie identycznościowego na $E$
w $^{*}$-słabej topologii operatorowej.
Każda jądrowa C$^{*}$-algebra jest lokalnie refleksywna
(por. \cite{Brown2008}, 9.3.1-3).
Algebry $\mathcal{A}$ i $\mathcal{B}$
można potraktować jako przestrzenie operatorowe w naturalny sposób.
Można więc wprowadzić dla nich projektywny iloczyn tensorowy
przestrzeni operatorowych na przestrzeniach dualnych
$\mathcal{A}^{*}$ i $\mathcal{B}^{*}$.
Dopełnienie algebraicznego iloczynu tensorowego w tej normie oznaczmy przez
$\mathcal{A}^{*} \hat{\otimes} \mathcal{B}^{*}$.
Stąd już prosta droga do definicji stanów separowalnych względem
algebr $\mathcal{A}$ i  $\mathcal{B}$.
Niech
$\mathcal{S}(\mathcal{A})$ i $\mathcal{S}(\mathcal{B})$
oznaczają przestrzenie stanów odpowiednio dla $\mathcal{A}$ i $\mathcal{B}$.
Zdefiniujmy dodatni stożek $C_{\mathcal{A},\mathcal{B}}$
stanów separowalnych względem  $\mathcal{A}$ i $\mathcal{B}$ poprzez
\begin{linenomath*}
 \begin{equation}
C_{\mathcal{A},\mathcal{B}} =
\overline{{\rm conv}}^{||\cdot||_{\wedge}}
\left \{\omega \otimes \varphi: \, \,\omega \in \mathcal{S}(\mathcal{A}),
\varphi \in \mathcal{S}(\mathcal{B})\right \}.
 \end{equation}
\end{linenomath*}
Jasne jest, że stan na $\mathcal{A} \bar{\otimes} \mathcal{B}$ jest separowalny,
tzn. $\tilde{\phi} \in C_{\mathcal{A}, \mathcal{B}}$,
wtedy i tylko wtedy, gdy $\tilde{\phi}$ jest separowalnym stanem na
algebrze von Neumanna $\mathcal{A}^{**} \bar{\otimes} \mathcal{B}^{**}$.

\begin{Theorem}
Niech $\mathcal{A}$ i $\mathcal{B}$ będą C$^{*}$-algebrami; załóżmy ponadto, że
$\mathcal{A}$ jest algebrą jądrową.
Stan $\tilde{\phi} \in
(\mathcal{A} \bar{\otimes} \mathcal{B})^{*}$ jest separowalny
względem $\mathcal{A}$ i $\mathcal{B}$,
wtedy i tylko wtedy, gdy
funkcjonał $\tilde{\phi} \circ ( \mathbf{1} \otimes S )$
jest dodatni dla każdego dodatniego odwzorowania skończonego rzędu
$S: \mathcal{A} \rightarrow \mathcal{B}$.
\end{Theorem}
\begin{proof}
Niech $\mathfrak{M}=\mathcal{A}^{**}$ i $\mathfrak{N}=\mathcal{B}^{**}$.
Z założenia $\mathfrak{M}$ jest algebrą iniektywną.
Jeżeli stan $\tilde{\phi}$ jest separowalny względem
$\mathcal{A}$ i $\mathcal{B}$, wówczas jest on również separowalny na
$\mathfrak{M} \bar{\otimes} \mathfrak{N}$.
Dla każdego dodatniego odwzorowania
$S: \mathcal{A} \rightarrow \mathcal{B}$, drugie dualne odwzorowanie
$S^{**}:\mathfrak{M} \rightarrow \mathfrak{N}$ jest dodatnie i normalne.
Stąd na mocy twierdzenia \ref{thm:PHcrit}, dla każdego dodatniego
$z \in \mathfrak{M} \bar{\otimes} \mathfrak{M}$,
$\langle (\mathbf{1} \otimes S^{**})(z), \tilde{\phi} \rangle \geq 0$.
Zatem, $\tilde{\phi} \circ (\mathbf{1} \otimes S)$ jest dodatnim funkcjonałem
na $\mathcal{A} \bar{\otimes} \mathcal{A}$.

Przypuśćmy przeciwnie, że $\tilde{\phi} \circ (\mathbf{1} \otimes S) \geq 0$
dla każdego dodatniego odwzorowania skończonego rzędu
$S: \mathcal{A} \rightarrow \mathcal{B}$.
Pokażemy, że stan $\tilde{\phi}$
jest separowalny na $\mathfrak{M} \bar{\otimes} \mathfrak{N}$.
Załóżmy nie wprost, że $\tilde{\phi}$ nie jest separowalny.
Z twierdzenia \ref{thm:PHcrit},
istnieje dodatnie normalne odwzorowanie skończonego rzędu
$\tilde{S}: \mathfrak{M} \rightarrow \mathfrak{N}$
oraz dodatni element $z \in \mathfrak{M} \bar{\otimes} \mathfrak{M}$,
takie że
\begin{linenomath*}
 \begin{equation}
\label{eq:Random6347}
\langle(\mathbf{1} \otimes \tilde{S})(z), \tilde{\phi}\rangle < 0.
 \end{equation}
\end{linenomath*}
Co więcej, $z$ możemy wybrać z algebraicznego iloczynu tensorowego
$\mathfrak{M} \otimes \mathfrak{M}$,
tzn. $z = \sum_{j=1}^{N} x_{j} \otimes y_{j}$,
oraz każde $x_{j}$ i $y_{j} \in \mathfrak{M}$.
Niech $E \subset \mathfrak{M}$ będzie układem operatorowym generowanym przez
$\{ x_{j}, y_{j}\}_{j=1}^{N}$.
Ponieważ $\mathcal{A}$ jest przestrzenią lokalnie refleksywną,
istnieje net $(T_{\lambda})$ kompletnie dodatnich kontrakcji
$T_{\lambda}: E \rightarrow \mathcal{A}$,
taki że $T_{\lambda}$ zbiega do odwzorowania identycznościowego na $E$
w *-słabej topologii, tzn.
dla każdego $\omega \in \mathcal{A}^{*}$ i $x \in E$,
$ \langle \omega, T_{\lambda}x \rangle\stackrel{\lambda}{\rightarrow}
\langle x, \omega \rangle$.
Niech $a_{\lambda}$ oznacza dodatni element $\mathcal{A} \otimes \mathcal{A}$,
$a_{\lambda} = (T_{\lambda} \otimes T_{\lambda}) (z) =
\sum_{j=1}^{N} T_{\lambda} x_{j} \otimes T_{\lambda} y_{j}$,
oraz niech
$i: \mathcal{A} \rightarrow \mathcal{A}^{**}=\mathfrak{M}$
będzie kanonicznym zanurzeniem.
Wtedy $i(a_{\lambda}) \stackrel{\lambda}{\rightarrow} z$
$\sigma$-słabo.
Ponieważ $\mathcal{B}$ jest przestrzenią Banacha,
posiada ona własność lokalnej refleksywności w następującym sensie
(por. Ryan \cite{Ryan2002}, rozdział 5.5).
Dla każdej skończenie wymiarowej podprzestrzeni $F \subset \mathcal{B}^{**}$,
istnieje operator $P_{F}: F \rightarrow \mathcal{B}$,
taki że dla każdego $y \in \mathcal{B}^{**} = \mathfrak{N}$
oraz $\varphi \in \mathcal{B}^{*}$, mamy
\begin{linenomath*}
 \begin{equation}
\label{eq:Random24093}
\langle \varphi, P_{F}y \rangle =\langle y, \varphi \rangle.
 \end{equation}
\end{linenomath*}
Niech $F = \tilde{S}(\mathfrak{M})$.
Z równania \eqref{eq:Random24093} wynika w oczywisty sposób, że
$P_{F}$ jest dodatnie.
Zdefiniujmy dodatnie odwzorowanie $S: \mathcal{A} \rightarrow \mathcal{B}$ przez
$S = P_{F} \circ \tilde{S} \circ i$.
Wówczas dla każdego $A \in \mathcal{A}$ oraz $\varphi \in \mathcal{B}^{*}$,
mamy
\begin{linenomath*}
 \begin{multline}
\langle \varphi, SA \rangle =
\langle \varphi, P_{F} ( \tilde{S} \circ i (A)) \rangle =
\langle  \tilde{S} \circ i (A), \varphi \rangle = \\
=\langle i(A), \tilde{S}_{*} \varphi \rangle =
\langle \tilde{S}_{*} \varphi, A \rangle,
 \end{multline}
\end{linenomath*}
Załóżmy ponadto, że
$\tilde{\phi} = \lim \limits_{n} \sum_{i=1}^{n}\omega_{i} \otimes \varphi_{i}$,
gdzie każde $\omega_{i} \in \mathcal{A}^{*}$,
$\varphi_{i} \in \mathcal{B}^{*}$.
Granicę rozumie się w projektywnej normie operatorowej $|| \cdot ||_{\wedge}$,
a więc również w słabej i *-słabej topologii na
$(\mathcal{A} \bar{\otimes} \mathcal{B})^{*}$.
Z głównego założenia wynika, że
$\langle \tilde{\phi},(\mathbf{1} \otimes S) (a_{\lambda}) \rangle \geq 0$.
Dodatkowo
\begin{linenomath*}
 \begin{multline}
\sum \limits_{i=1}^{n}\langle \omega_{i} \otimes \varphi_{i},
(\mathbf{1} \otimes S)(a_{\lambda}) \rangle =
\sum \limits_{i=1}^{n} \sum \limits_{j=1}^{N}
\langle \omega_{i} , T_{\lambda} x_{j} \rangle
\langle \tilde{S}_{*} \varphi_{i}, T_{\lambda} y_{j}\rangle
\stackrel{\lambda}{\longrightarrow}\\
\sum \limits_{i=1}^{n} \sum \limits_{j=1}^{N}
\langle  x_{j}, \omega_{i} \rangle\langle  y_{j},  \tilde{S}_{*} \varphi_{i} \rangle =
\langle (\mathbf{1} \otimes \tilde{S})(z),
\sum \limits_{i=1}^{n} \omega_{i} \otimes \varphi_{i} \rangle .
 \end{multline}
\end{linenomath*}
Biorąc granicę względem $n$,
ponieważ $(T_{\lambda})$ jest netem kontrakcji, dostajemy
$\langle (\mathbf{1} \otimes \tilde{S}) (z) ,
\tilde{\phi}\rangle \geq 0$,
co przeczy równaniu \eqref{eq:Random6347},
a zatem stan $\tilde{\phi}$ jest separowalny na
$\mathfrak{M} \bar{\otimes} \mathfrak{N}$.
\end{proof}

Udowodnione powyżej twierdzenie uogólnia analogiczny wynik opublikowany przez
E.\,St{\o}rmera; pozwala jednak opuścić wcześniejsze założenie, że występujące
w tezie twierdzenia odwzorowania to wszystkie odwzorowania dodatnie
(nie tylko skończonego rzędu) oraz że jedna z algebr jest tzw. algebrą
UHF \cite{stormer2009separable}.

\paragraph{}
Powyższa analiza pokazuje, że nawet w najogólniejszym przypadku złożonych
układów kwantowych na nieskończenie wymiarowych algebrach von Neumanna
splątanie stanów kwantowych zależy ściśle od własności dodatnich odwzorowań
na tych algebrach.
W następnym rozdziale zajmiemy się szczegółową analizą dodatnich odwzorowań
na skończenie wymiarowych algebrach macierzy, skupiając się przede wszystkim
na najprostszym, do tej pory nierozpoznanym przypadku algebry macierzy
$M_{3}$.
