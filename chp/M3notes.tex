\chapter{Odwzorowania dodatnie na algebrach macierzowych}
\label{chp:M3notes}

\section{Charakterystyka odwzorowań na algebrze $M_{2}$}
\label{sec:M2notes}

W tym podrozdziale przestawimy główny rezultat opublikowany w pracy
\cite{miller2015stable},
który stanowi alternatywny dowód twierdzenia udowodnionego przez
E\,St{\o}rmera \cite{stormer1963positive}
i S.\,L.\,Woronowicza \cite{woronowicz1976positive},
podającego pełną charakterystykę odwzorowań dodatnich na algebrze $M_{2}$.


\begin{Theorem}
\label{thm:PositiveMapsOnM2}
    Niech $S:M_{2} \rightarrow M_{2}$ będize odwzorowaniem dodatnim.
    Wówczas $S = \Lambda_{1}  + \Lambda_{2} \circ t$,
    gdzie odwzorowania
    $\Lambda_{1}, \Lambda_{2}:M_{2} \rightarrow M_{2}$
    są kompletnie dodatnie, a
    $t: M_{2} \rightarrow M_{2}$ oznacza transpozycję macierzy na $M_{2}$.
\end{Theorem}


\begin{proof}
\label{RandomLabel:875919}
  Odwzorowanie $S$ działa na wektory bazowe $\sigma_{\mu}$ poprzez
\begin{equation}
 S \sigma_{\mu} = \sum_{\nu = 0}^{3} \pi(S)_{\mu \nu} \sigma_{\nu}.
\end{equation}
Załóżmy na początek, że $\pi(S) \in \Theta(L_{4})$. 
Jeżeli $\pi(S) = \rho(V) \in \text{SO}^{+}(1,3)$,
$V \in SL_{2}(\mathbb{C})$,
wówczas
$S \sigma(x) = \sum_{\mu = 0}^{3}  x_{\mu} V^{*} \sigma_{\mu} V =
 V^{*} \sigma(x) V$,
 $x \in \mathbb{R}^{4}$,
a więc $S$ jest kompletnie dodatnie,
skoro każda macierz $A \in M_{2}$ jest kombinacją liniową macierzy hermitowskich.
Z drugiej strony,
jeśli $\pi(S) = \rho(V) J$,
to
\begin{multline}
\label{RandomLabel:830200}
S \sigma(x) =
  x_{0} \, V^{*}V + x_{1} \, V^{*} \sigma_{1} V -
  x_{2} \, V^{*} \sigma_{2} V +
  x_{3} \, V^{*} \sigma_{3} V = \\
  = x_{0} \, V^{*}V + x_{1} \, V^{*} \sigma_{1}^{t} V +
  x_{2} \, V^{*} \sigma_{2}^{t} V +
  x_{3} \, V^{*} \sigma_{3}^{t} V =
  V^{*} \sigma(x)^{t} V,
\end{multline}
ponieważ $\sigma_{2}^{t} = - \sigma_{2}$.
Zatem, $S$ jest złożeniem transpozycji i odwzorowania kompletnie dodatniego
(tzn. $S$  jest kompletnie \emph{ko}dodatnie).
Z kolei załóżmy, że $\pi(S) \in \delta(L_{4})$,
gdzie
$
 \delta(L_{4}) = \left \{ u w^{t}:
 \, u, w \in \partial L_{4}  \right \}
$,
jest zbiorem operatorów rzedu 1, zachowujących brzeg stożka $\partial L_{4}$.
Z \cite[Lematu 3.2]{loewy1975positive}, 
$S$ jest ektremalnym odwzorowaniem dodatnim, takim że $\text{rank} S = 1$,
co implikuje, że $S$ musi być kompletnie dodatnie.

Jeżeli $\pi(S)$ jest macierzą zachowującą stożek Lorentza,
z \cite[Corollary 4.6]{loewy1975positive} wnioskujemy, że
$\pi(S) \in \text{conv} \left ( \Theta(L_{4}) \cup \delta(L_{4}) \right)$. 
Innymi słowy każdy operator zachowujący stożek Lorentza $L_{4}$
jest wypukłą kombinacją tych operatorów, które dodatkowo zachowują brzeg stożka.
Z tego, co zostało pokazane powyżej, wynika, iż
\begin{equation}
\label{RandomLabel:587827}
    S = \Lambda_{1} + \Lambda_{2} \circ t,
\end{equation}
gdzie $\Lambda_{1}$, $\Lambda_{2}$
są odwzorowaniami kompletnie dodatnimi.
\end{proof}

Twierdzenie \ref{thm:PositiveMapsOnM2} podaje alternatywny dowód
znanego rezultatu \cite{stormer1963positive, woronowicz1976positive}.
Jednak, o ile metody wykorzystywane przez autorów tamtych prac
polegają w głównej mierze na algebraicznych manipulacjach macierzami,
o tyle wprowadzenie odwzrorowania $\pi$ pozwoliło nam
na podkreślenie istotnych dla problemu geometrycznych własności stożków
zawartych w  $\mathbb{R}^{4}$.
Dzięki temu udało się otrzymać pełną charakteryzację odwzorowań dodatnich
na $M_{2}$,
której źródeł należy się doszukiwać w szczególnej symetrii stożka
$L_{4}$, reprezentującego macierze z $M_{2}$.
W tym najprostszym przypadku mogliśmy wykorzystać głęboki wynik z
\cite{loewy1975positive}, dotyczący
opisu operatorów zachowujących stożek $L_{4}$.
Jak zobaczymy poniżej, jeśli przejdziemy do kolejnego przypadku
odwzorowań dodatnich na $M_{3}$,
stożek wektorów $x \in \mathbb{R}^{9}$ takich że
$\lambda(x) =  \sum_{\mu=0}^{8} x_{\mu} \lambda_{\mu}$
jest dodatnią macierzą z $M_{3}$,
dla dowolnej bazy $(\lambda_{\mu})_{\mu=0}^{8}$ przestrzeni liniowej $M_{3}$,
nie jest już stożkiem Lorentza $K_{9}$ \cite{goyal2011geometry}.
Dlatego też, kiedy przyjdzie nam zająć się tym przypadkiem,
będziemy musieli poszerzyć metody analizy o dodatkowe narzędzia.


Naturalne pytanie, które pojawia się w kontekście Twierdzenia
\ref{thm:PositiveMapsOnM2},
to kiedy dodatnie odwzorowania na $M_{2}$ jest również kompletnie dodatnie.

\begin{Theorem}
\label{thm:MapsPreservingIdentity}
Niech $S: M_{2} \rightarrow M_{2}$ będzie dodatnim odwzorowaniem zachowującym
macierz jednostkową. Rozważmy następujące warunki:
\begin{enumerate}

\item
\label{lem:condProj}
$S$ jest odwzorowaniem Schwarza oraz istnieje projektor ortogonalny $P$, rzędu 1, dla którego $S(P)^{2} = S(P)$.

\item
\label{lem:condCommut}
Istnieje projektor ortogonalny P rzędu 1, taki że dla każdego $X \in M_{2}$,
$S([P,X]) = [S(P), \, S(X)]$.

\item
\label{lem:condCP}
$S$ jest kompletnie dodatnie.
\end{enumerate}
Wówczas pomiędzy warunkami powyżej zachodzi relacja:
$
\ref{lem:condProj} \Rightarrow
    \ref{lem:condCommut} \Rightarrow \ref{lem:condCP}.
$
\end{Theorem}

\begin{proof}
$\ref{lem:condProj}) \Rightarrow \ref{lem:condCommut})$.

Niech $A_{k} = kP + iX$, dla pewnego $X = X^{*} \in M_2$
oraz $k \in \mathbb{N}$.
Kładąc $A_{k}$ w nierówności Kadisona-Schwarza
\eqref{eq:SchwarzInequality}, otrzymujemy
\begin{equation}
[S(P), \,  S(X)] - S([P, \, X]) \leq - \frac{i}{k} (S(X^{2}) - S(X)^{2}),
\end{equation}
dla każdego $k \in \mathbb{N}$.
Zatem
\begin{equation}
\label{ineq:Commutators}
[S(P), \, S(X)] - S([P, \, X]) \leq 0.
\end{equation}
Biorąc tym razem $A'_{k} =kP - i X$ i powtarzając rachunek,
dostajemy nierówność odwrotną do \eqref{ineq:Commutators},
a stąd
\begin{equation}
\label{eq:Commutators}
S([P, X]) = [S(P), S(X)],
\end{equation}
dla każdej hermitowskiej macierzy $X$.
Ponieważ równanie \eqref{eq:Commutators} jest liniowe w $X$,
zachodzi również dla każdej macierzy $X \in M_{2}$.

$\ref{lem:condCommut}) \Rightarrow \ref{lem:condCP})$.
Niech $P$ będize projektorem ortogonalnym rzędu 1, takim że
\begin{equation}
\label{eq:Commutator}
S([P,X]) = [S(P), \, S(X)],
\end{equation}
dla każdego $X \in M_{2}$.
Jeśli $S$ jest kompletnie dodatnim odwzorowaniem, wówczas dla każdej pary
unitarnych macierzy $U, V \in M_{2}$,
odwzorowanie $\tilde{S}: A \mapsto U^{*} S(V^{*} A V) U$
jest również kompletnie dodatnie.
Możemy zatem założyć bez straty ogólności, że
$P = P_{1} =
\left(
\begin{smallmatrix} 1 & 0 \\ 0 & 0 \end{smallmatrix}
\right)$,
i $S(P)$ jest diagonalne.
Niech $P_{2} = \left(
\begin{smallmatrix} 0 & 0 \\ 0 & 1 \end{smallmatrix}
\right)$,
$E_{12} =
\left(
\begin{smallmatrix} 0 & 1 \\ 0 & 0 \end{smallmatrix}
\right)$,
and
$E_{21} =
\left(
\begin{smallmatrix} 0 & 0 \\ 1 & 0 \end{smallmatrix}
\right)$.
Załóżmy, że
$S(P_{1}) = \lambda_{1} P_{1} + \lambda_{2} P_{2}$,
$0 \leq \lambda_{2} \leq \lambda_{1} \leq 1$.
Jeśli $S(E_{12}) = 0$,
wówczas oczywiście również $S(E_{21}) = 0$ i w rezultacie
$S$ odwzorowuje $M_{2}$
w przemienną algebrę macierzy diagonalnych.
Stąd, $S$ jest kompletnie dodatnie.
Z drugiej strony, jeśli
$S(E_{12}) =
\left(
\begin{smallmatrix} w & z \\ x & y \end{smallmatrix}
\right) \neq 0$,
$w, z, x, y \in \mathbb{C}$
i gdy położymy $X = E_{12}$ we wzorze \eqref{eq:Commutator},
pamiętając jednocześnie, że $\lambda_{1} \geq \lambda_{2}$,
otrzymujemy $x = y = w = 0$ i $\lambda_{1} - \lambda_{2} = 1$.
To oznacza, że $\lambda_{1} = 1$ and $\lambda_{2} = 0$ oraz
$S(E_{12}) = z E_{12}$.
A zatem $S$ działa na macierz
$X \in M_{2}$, $X = (x_{ij})_{i,j=1,2}$ jak
\begin{equation}
S(X) = \begin{pmatrix}
 x_{11} & z \, x_{12} \\
\overline{z} \, x_{21} & x_{22}
\end{pmatrix}.
\end{equation}
Stąd wynika już w prosty sposób, że $S$ jest kompletnie dodatnie.
Rzeczywiście,
z dodatniości $S$ mamy $|z| \leq 1$.
Jeśli napiszemy
\begin{equation}
S(X) =
\begin{pmatrix}
1 & z \\ \overline{z} & 1
\end{pmatrix} \circ X,
\end{equation}
gdzie symbol $\circ$ w tym przypadku oznacza iloczyn Hadamarda macierzy,
to jest iloczyn, w którym mnoży się ze sobą macierze tego samego wymiaru
element po elemencie.
Ponieważ $|z| \leq 1$, macierz
$
\left( 
\begin{smallmatrix} 1 & z \\ \overline{z} & 1 \end{smallmatrix}
\right)$
jest dodatnia.
Ogólnie znany jest fakt, że odwzorowanie $X \mapsto A \circ X$
jest kompletnie dodatnie wtedy i tylko wtedy, gdy $A$ 
jest dodatnią macierzą.
\end{proof}

\begin{Example}
Można bez trudu pokazać, że warunek \ref{lem:condProj} w
Twierdzeniu \ref{thm:MapsPreservingIdentity} jest w istotny sposób silniejszy
niż warunek \ref{lem:condCommut}.
Niech $S$ będzie odwzorowaniem bistochastychnym na $M_{2}$, takim że
\begin{equation}
\label{RandomLabel:514431}
    \pi(S) = \begin{pmatrix}
    1 & 0 & 0 & 0 \\
    0 & b & 0 & 0 \\
    0 & 0 & 0 & 0 \\
    0 & 0 & 0 & 0
    \end{pmatrix},
    \quad 0 < b < 1,
\end{equation}
czyli $S(\sigma_{1}) = b \sigma_{1}$ oraz
$S(\sigma_{2}) = S(\sigma_{3}) = 0$.  
Wówczas $S\left( [P_{0}, \, X ] \right) = [ S(P_{0}), \, S(X) ]$,
gdzie
$P_{0}= \frac{1}{2} \left( \begin{smallmatrix}
 1 & 1 \\ 1 & 1
 \end{smallmatrix} \right)$ i $X \in M_{2}$.
Jednakże nie istnieje projektor ortogonalny $P$ rzędu 1,
dla którego $S(P)^{2} = S(P)$.
Rzeczywiście, skoro
$P_{0} = \frac{1}{\sqrt{2}}(\sigma_{0} + \sigma_{1})$,
$S(P_{0}) = \frac{1}{\sqrt{2}}(\sigma_{0} + b \sigma_{1})$.
Niech $X$ będzie macierzą hermitowską, $X = \sigma(x)$, $x \in \mathbb{R}^{4}$,
$S(X) = x_{0} \sigma_{0} + b x_{1} \sigma_{1}$.
Wtedy $[S(P_{0}), \, S(X) ] =0$. 
Co więcej,
$[P_{0}, \, \sigma(x)] = 
i(x_{2} \sigma_{3} - x_{3} \sigma_{2})$,
a stąd 
$S([P_{0}, \, \sigma(x)]) = 0$.
Warunek \ref{lem:condCommut} w Twierdzeniu \ref{thm:MapsPreservingIdentity}
jest prawdziwy dla każdej macierzy hermitowskiej $X$,
a zatem również dla każdej macierzy $X \in M_{2}$.  
Z drugiej strony, 
$P = \sigma(a)$, $a \in \mathbb{R}^{4}$, jest projektorem ortogonalnym
rzędu 1,
wtedy i tylko wtedy, gdy $a_{0} = ||\vec{a}|| = \frac{1}{\sqrt{2}}$.
Zatem $S(P) \neq \mathbf{1}$, a ponieważ $b<1$,
więc $S(P)$ nie może być projektorem ortogonalnym \mbox{rzędu 1},
tzn. $S(P)^{2} \neq S(P)$.    
\end{Example}

Dla odwzorowań ektremalnych warunek \ref{lem:condProj} z Twierdzenia
\ref{thm:MapsPreservingIdentity} okazuje się również wystarczający
do tego, aby takie odwzorowanie było kompletnie dodatnie dla każdego wymiaru.
Bardziej ogólny problem, postawiony przez Robertsona
\cite{robertson1983schwarz},
czy istniej odwzorowanie Schwarza, które jest ekstremalne, ale
nie 2-dodatnie, pozostaje otwarty.
Zauważmy, że z \cite[Theorem 3.3]{marciniak2008extremal} wynika, iż
każde ektremalne odwzorowanie 2-dodatnie jest kompletnie dodatnie.

\begin{Theorem}
\label{thm:ExtremalSchwarz}
Niech $S: M_{n} \rightarrow M_{n}$
będzie odwzorowaniem Schwarza, ekstremalnym w zbiorze wszystkich odwzorowań
dodatnich. Załóżmy dodatkowo, że
istnieje para projektorów ortogonalnych $P, Q \in M_{n}$, rzędu 1,
dla których $S(P) = Q$.
Wówczas $S$ jest kompletnie dodatnie. 
\end{Theorem}
\begin{proof}
Niech $(e_{i})_{i=1}^{n}$
będzie standardową bazą ortonormalną w $\mathbb{C}^{n}$.  
Ponieważ chcemy pokazać, że $S$ jest kompletnie dodatnie,
możemy założyć, że $P = Q = P_{n}$,
gdzie $P_{n} = e_{n} e_{n}^{*}$ jest projektorem ortogonalnym na jednowymiarową
przestrzeń rozpiętą przez wektor $e_{n} = (0,0,\ldots,0,1) \in \mathbb{C}^{n}$. 
Z faktu, że $S$ zachowuje macierz jednostkową, wynika
$S(P_{n}^{\perp}) = P_{n}^{\perp}$,
gdzie $P_{n}^{\perp} = \mathbf{1}_{n} - P_{n}$. 
Dla macierzy $X \in M_{n}$,
zapisanej w postaci blokowej
$X = \left( \begin{smallmatrix} A & u \\ w^{t} & z \end{smallmatrix} \right)$,
gdzie $A \in M_{n-1}$, $u, w \in \mathbb{C}^{n-1}$ są kolumnami, a
$z \in \mathbb{C}$, mamy
\begin{equation}
\label{eq:SBlockForm}
    S (X) = \begin{pmatrix}
        \hat{S}_{0}(A) & S_{1} u \\
        (\overline{S}_{1} w)^{t} & z
    \end{pmatrix},
\end{equation}
dla $\hat{S}_{0}: M_{n-1} \rightarrow M_{n-1}$ będącego z konieczności
odwzorowaniem Schwarza na $M_{n-1}$, and $S_{1} \in M_{n-1}$.
Rzeczywiście, ponieważ $S$ jest odwzorowaniem Schwarza, 
z \cite[Proposition 2.1.5]{stormer2012positive} mamy, że
$S(P_{n} X) = P_{n} S(X)$.
Stąd
$S(P_{n} X P_{n}) = P_{n} S(X) P_{n}$ i podobnie
$S(P_{n}^{\perp} X P_{n}^{\perp}) = P_{n}^{\perp} S(X) P_{n}^{\perp}$,
$S(P_{n}^{\perp} X P_{n}) = P_{n}^{\perp} S(X) P_{n}$,
$S(P_{n} X P_{n}^{\perp}) = P_{n} S(X) P_{n}^{\perp}$,
co dowodzi, że $S$ musi mieć postać jak w równaniu \eqref{eq:SBlockForm}.

Jeśli weźmiemy jednowymiarowy projektor ortogonalny 
$P_{u} = u u^{*}$, gdzie
$u = (u_{1}, u_{2}, \ldots, u_{n}) = (\vec{u}, u_{n})$,
$\vec{u} = (u_{1}, u_{2}, \ldots, u_{n-1}) \in \mathbb{C}^{n-1}$,
$u_{n} \in \mathbb{C}$,
dostaniemy
\begin{equation}
 SP_{u} \:=\:  S \begin{pmatrix}
    \vec{u} \vec{u}^{\,*} & \overline{u}_{n} \vec{u} \\
    u_{n} \vec{u}^{\,*}   & |u_{n}|^{2}
 \end{pmatrix} \: = \: 
 \begin{pmatrix}
    \hat{S}_{0}(\vec{u} \vec{u}^{*}) &
         \overline{u}_{n} S_{1} \vec{u} \\
    u_{n} ( S_{1} \vec{u} )^{*} &
        |u_{n}|^{2}
 \end{pmatrix}.
\end{equation}
W przypadku, gdy $u_{n} \neq 0$,
biorąc dopełnienie Schura
(por. \cite[Twierdzenie 1.12, p.34]{zhang2006schur}),
mamy, że $SP_{u} \geq 0$, wtedy i tylko wtedy, gdy
\begin{equation}
\label{ieq:SchurForS}
  S_{1} \vec{u} \vec{u}^{*} S_{1}^{*} \leq \hat{S}_{0}(\vec{u}\vec{u}^{*}).
\end{equation}
To oznacza, że $S$ jest sumą dwóch dodatnich odwzorowań działających na
jednowymiarowy projektor ortogonalny jak poniżej:
\begin{equation}
 SP_{u} \:=\:  
 \begin{pmatrix}
      S_{1} \vec{u} \vec{u}^{*} S_{1}^{*}  &
         \overline{u}_{n} S_{1} \vec{u} \\
    u_{n} ( S_{1} \vec{u} )^{*} &
        |u_{n}|^{2}
 \end{pmatrix} +   
 \begin{pmatrix}
    \hat{S}_{0}(\vec{u} \vec{u}^{*})  -  S_{1} \vec{u} \vec{u}^{*} S_{1}^{*} & 0 \\
    0 & 0
 \end{pmatrix}.
\end{equation}
Skoro $S$ jest z założenia ekstremalne i  $S(P_{n}) = P_{n}$,
\begin{equation}
 SP_{u} \:=\:  
 \begin{pmatrix}
      S_{1} \vec{u} \vec{u}^{*} S_{1}^{*}  &
         \overline{u}_{n} S_{1} \vec{u} \\
    u_{n} ( S_{1} \vec{u} )^{*} &
        |u_{n}|^{2}
 \end{pmatrix} = 
    \begin{pmatrix}
    S_{1} & 0 \\ 0 & 1
    \end{pmatrix}
    \, P_{u} \,
    \begin{pmatrix}
    S_{1} & 0 \\ 0 & 1
    \end{pmatrix}^{*}.
\end{equation}
Jeśli położymy $U = \left( \begin{smallmatrix} S_{1} & 0 \\ 0 & 1 
\end{smallmatrix} \right) \in M_{n}$,
z faktu, że $S(P_{n}^{\perp}) = P_{n}^{\perp}$,
dostaniemy $U U^{*} = \mathbf{1}_{n}$, tj. $U$ macierzą unitarną oraz
$S(X) = U X U^{*}$, $X \in M_{n}$, co kończy dowód.
\end{proof}

Przedstawiwszy geometryczny dowód Twierdzenia \ref{thm:PositiveMapsOnM2}
o rozkładzie odwzorowań dodatnich na $M_{2}$, możemy pokusić się o 
o przedstawienie dodatkowych faktów dotyczących odwzorowań bistochastycznych.
Odznaczmy przez $\Delta_{0}$ zbiór wszystkich bistochastycznych odwzorowań na
$M_{2}$.
Niech także $\Delta_{1} \subset \Delta_{0}$ oznacza zbiór tych odwzorowań
bistochastycznych,
które są jednoczenie kompletnie dodatnie i kompletnie kododatnie.
Korzystając z tego samego geometrycznego przedstawienia zbioru
odwzorowań dodatnich jako stożka Lorentza $L_{4}$,
możemy pokazać, że $\Delta_{1}$ jest zbiorem zaskakująco dużym.

Każda macierz $Y \in M_{n}(\mathbb{R})$ daje się rozłożyć względem wartości singularnych:
$Y = O_{1} D O_{2}$, gdzie
$O_{1}, O_{2}$ to macierze ortogonalne tego samego wymiaru, co $Y$, a
$D = \text{diag}(t_{1}, t_{2}, \ldots, t_{n})$ jest macierzą diagonalną,
taką że wartości singularne $Y$ dają się uporządkować:
$t_{1} \geq t_{2} \geq \ldots t_{n} \geq 0$.
Innymi słowy, wartości singularne $Y$ to wartości własne macierzy
$|Y| = (Y^{t} Y)^{1/2}$.
W przestrzeni operatorów $M_{n}(\mathbb{R})$ możemy wyróżnić wiele norm;
jedną z nich jest norma operatorowa: $|| \cdot ||$;
drugą nazwiemy normą śladową i zdefiniujemy przez
$||Y||_{1} = \sqrt{\text{Tr} \, Y^{t} \, Y }$.
domkniętą kulę o jednostkowym promieniu w normie operatorowej oznaczmy:
$K_{0} = \left \{ Y \in M_{3}(\mathbb{R}): \, ||Y|| \leq 1 \right \}$;
natomiast kulę w normie śladowej jako
$K_{1} = \left \{ Y \in M_{3}(\mathbb{R}): \, ||Y||_{1} \leq 1 \right \}$.

\begin{Theorem}
\label{thm:Ball}
Istnieje wypukły izomorfizm $F$, taki że $K_{0}$, $F(\Delta_{0}) = K_{0}$. 
Ponadto, $F(\Delta_{1}) = K_{1}$.    
\end{Theorem}
\begin{proof}
Niech $S$ będzie odwzorowaniem bistochastycznym na $M_{2}$.
Macierz $\pi(S)$ możemy zapisać w postaci blokowej
\begin{equation}
\label{eq:Sbistochastic}
\pi(S) = \begin{pmatrix}
    1  &  \vec{0}^{t} \\
    \vec{0} & Y
    \end{pmatrix},
\end{equation}
$Y \in M_{3}(\mathbb{R})$,
a $\vec{0} \in \mathbb{R}^{3}$ jest zerowym wektorem-kolumną.
Połóżmy $F(S) = Y$. 
Z definicji \eqref{RandomLabel:365828} odwzorowania $\pi$,
jasne jest, że dla  $S_{1}, S_{2} \in \Delta_{0}$,
mamy $F \left( \lambda S_{1} + (1-\lambda) S_{2} \right) = 
 \lambda F(S_{1}) + (1-\lambda) F(S_{2})$,
$0 \leq \lambda \leq 1$
oraz $F(S_{1} S_{2}) = F(S_{1}) F(S_{2})$.
Odwzorowanie $S$ jest dodatnie wtedy i tylko wtedy, gdy
$\pi(S)$ zachowuje stożek $L_{4}$,
co w tym wypadku jest równoważne temu, iż $F(S) = Y \in K_{0}$.
Co więcej, łatwo zauważyć, że oba przekształcenia, $F$ i $F^{-1}$,
są ciągłe w topologii indukowanej przez normy na
$\Delta_{0}$ and $K_{0}$.
Zatem $F(\Delta_{0}) = K_{0}$, a ponadto $F$ jest homeomorfizmem.

W dalszej części załóżmy, że $S \in \Delta_{1}$.
Ponieważ $S$ jest kompletnie dodatnie,
z \cite[Twierdzenia 1]{landau1993birkhoff} wiemy, że
$S$ wypukłą kombinacją odwzorowań unitarnych:
$X \mapsto U^{*} X U$, $U \in \text{U}(2)$.
Z dowodu  \mbox{Twierdzenia \ref{thm:PositiveMapsOnM2}},
dla kontkretnej postaci $\pi(S)$, takiej jak w równaniu \eqref{eq:Sbistochastic},
oznacza to, że $Y$ należy do wypukłej otoczki generowanej przez elementy grupy $\text{SO}(3)$.
Z  pracy \cite{miranda1994group} (Corollary na str. 139),
wnioskujemy, że jeśli $t_{1} \geq t_{2} \geq t_{3} \geq 0$
to wartości singularne macierzy $Y$,
wówczas $Y \in \text{conv} \, \text{SO}(3)$,
wtedy i tylko wtedy, gdy
\begin{equation}
\label{eq:ineqForSingVals}
t_{1} \leq 1
\quad \text{and} \quad
t_{1} + t_{2} - \text{sig} (\text{det} Y) \, t_{3} \leq 1.
\end{equation}
Zatem $S$ jest kompletnie kododatnie, tzn.
$S' = S \circ t$ jest kompletnie dodatnim bistochastycznym odwzorowaniem.
Niech $F(S') = Y'$.
Wówczas $Y' = Y \, \text{diag} (1,-1,1)$.
Obie macierze, $Y$ i $Y'$, mają te same wartości singularne oraz
$\text{det} \, Y' = - \text{det} \, Y$.
Ponieważ $Y' \in  \text{conv} \, \text{SO}(3)$, mamy
$t_{1} + t_{2} + \text{sig} (\text{det} Y) \, t_{3} \leq 1$.
Łącząc ten fakt z równaniem \eqref{eq:ineqForSingVals},
skoro $\text{det} \, Y = 0$, wtedy i tylko wtedy, gdy  $t_{3} = 0$,
otrzymujemy, że $S$ jest jednocześnie kompletnie dodatnie i kompletnie kododatnie,
wtedy i tylko wtedy, gdy
$||Y||_{1} = t_{1} + t_{2} + t_{3} \leq 1$,
co w rezultacie oznacza, że $F(\Delta_{1}) = K_{1}$
i kończy dowód.
\end{proof}

Podsumowując; przedstawiliśmy powyżej geometryczne podejście do problemu
analizy odwzorowań dodatnich, przynajmniej w najprostszym przypadku odwzorowań
algebry macierzy $M_{2}$.
W szczególności,
pokazaliśmy, że dzięki tym metodom można w zwarty i elegancki sposób dowieść
znanego twierdzenia o rozkładzie odwzorowań dodatnich na sumę przekształceń
kompletnie dodatnich i kompletnie kododatnich,
które okazuję się być bezpośrednio zależne od faktu, że każdy operator
zachowujący stożek Lorentza w $\mathbb{R}^{n}$ jest wypukłą kombinacją tych
operatorów, które dodatkowo zachowują brzeg stożka.
Jak już wspomnieliśmy, twierdzenie dowiedzione powyżej
przekłada się na kryterium sperowalności stanów kwantowych dla układu
złożonego dwóch qubitów i otrzymuję formę znanego kryterium PPT
\cite{peres1996separability,horodecki1996separability}.
Dowód kolejnego faktu, Twierdzenia \ref{thm:MapsPreservingIdentity},
chociaż nie posługuje się tymi samymi metodami geometrycznymi w sposób jawny,
bazuje na wiedzy dotyczącej operatorów zachowujących stożek Lorentza i posiadających
dodatkowo normę równą jedności.
Stąd pochodzi założenie \ref{lem:condProj}. w Twierdzeniu \ref{thm:MapsPreservingIdentity}.
Wreszcie, Twierdzenie \ref{thm:Ball} pokazuje, że zbiór odwzorowań bistochastycznych,
które są jednocześnie kompletnie dodatnie i kompletnie kododatnie jest
topologicznie tożsamy z kulą jednostkową w $M_{3}(\mathbb{R})$ w normie śladowej.

W dalszej części rozwiniemy przedstawione metody na kolejny nietrywialny przypadek
odwzorowań algebry $M_{3}$.


\section{Stabilne podalgebry jako metoda analizy własności odwzorowań na $M_{3}$}
\label{sec:M3notes}

